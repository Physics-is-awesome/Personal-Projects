
\chapter{*Language}
"The limits of my language mean the limits of my world" | Ludwig Wittgenstein
\section{New Words}
\subsection{*Why Konie Greek}
\subsection{Philotimic}
\section{*Nature of Meaning and Semantics vs Pragmatics}
\section{*Language and Thought}
\section{*Mathematics and Language}
\section{*Language and the Ineffable}
\section{*Language and LLM}
%%%%%%%%%%%%%%%%%%%%%%%
\chapter{Personal Analysis}
"The unexamined life is not worth living" | Socrates
\section{Disgrace and Pride}
(if anyone else reads this, don't pay too much mind. I use words differently than their actual meaning. I don't word things very well. I was simply trying to capture my own indescribable and esoteric and possibly failable thoughts in this moment on this topic, my actual feelings are very different than what can be interpreted through the word choice.)(Additional update: upon later analysis, I have found that these ideas of pride and disgrace are for the most part the idea of seeing my earlier mentioned "personal values" in a light very similar to traditional morality. This doesn't fully explain it and I will continue to work on refining these ideas beyond the esoteric concentration currently presented.)
\par Why is it that you are so accustomed to the use of disgrace in your speech and why does it effect you in such a manner. The idea of 'disgracing' yourself is so vital to your worldview, it effects everything; morals, interactions with other, and so much more. It isn't even like you are all that effected by the thoughts of others. In fact the most abnormal and extreme feelings of self-disgust and disgrace are related to a refection of the thoughts of other. Why is it that I feel like to see other, value their opinions, and let them influence men to disgrace myself. It is an odd thing, yet it is so very natural. It influences everything, and it is so much more extreme when I am in isolation as I am now. What is it about the concept of being influenced by others to be so repulsive, so disgusting that I won't just let it influence my actions but I will try to persuade others my way is best with pride. With this eternal pride that locks me in, says that I am always all of the way in all things. There are so many things to think about in this discussion so I will try to take it piece by piece and hopefully add some things.(also, what it with the strange switching from dialogue with an external, internal, and this explanation style?)"
\par Well, disgust is a natural motivating force. It motivates far better than most, it is consuming. It involves fear, pride, righteous anger, annoyance, and so many more extreme and strong emotions. It motives far more than most, and it comes so easily, especially for someone like me. There is no greater fear than falling in ones own eyes. My disgust of others is just who I am, it is what makes me who I am. It disconnects me from the petty emotions around me. I mean, can people really say that those are better; to be insecure, jealous, vapid, empty, etc? Is it really better? Take insecurity, it is such a strong emotion that consumes people so very easily, making them bicker and fight, making them claw at each other with their broken paws that do very little besides hurt themselves and hurt others by their own volition of nonsensical blight. My 'pride' can rise me above that. If I do not see them as anything, then for what reason do I have to be insecure about this vapid collection of dust. Now I am not truly some raging narcissist who sees no one beyond and object for my own use. I see people as they are, I care for others, value them. I just simply do not value them in the way others prescribe them. I don't see them as threats, competition, just broken monkeys in need of assistance(and that I am the same way and can occasionally use assistance myself, just less than most.) I still see them, I make friends just as easy, if not easier than most due to this.
\par This disgust is a strengthening endeavor that assistance me in many ways. My self-disgust pushes me further. It is what lead my to teach myself calculus in elementary, and has brought me here know with graduate level knowledge, research, and so much more. It has disconnected my from the vapid desires of others in their looks and whatever other nonsense.(though I will add, this disgust is not some obscene emotional self-distrain but rather a more abstract understanding that doesn't truly make me feel bad, just concous of 'wrong doing.' More so like a passion against, I feel no negative emotions about myself)
\par Though I must acknowledge that this does not make me better, it makes me better within my own eyes, my own values, not others. My values have no greater truth than that they are my own. So my disgust should always remain abstract, looking down at the ideas not the people. I have done great at this, never prescribing habits or concepts to people. Though as of late I have had trouble with this, seeing people beyond the moment. Being conscious of their deniers and thoughts, it can create some discontent. I should remain as I was, seeing people only in the instant. Nothing more. I am the only conscious being within the confines of my own mind, for it is better this way. You cannot accuse a rock of moral failure, only a man. For which I am the only man that I see, the only man that I know, I am to constituent for the moral blame. The disgust should be within me and me alone seeing only me. For I am the only one held to my values and the only one who could suffer the breaking of them. For they are my values alone, that is the purpose of the disconnect. Why my values must be mine alone, for I am the only one to be connected to them, the only man in my eyes.
\par For this I also hold the burden of thought and reason. The thought behind my morals, my values, my beliefs, my religion, my reality. I see the world as my land for the conquest of my own knowledge to build up it all. That is the whole purpose of this book, to fully disconnect, to find myself fully and fully alone. Because, in the end that is all that matters in the abstract. For sure, I love my family, my friends, and my fellow man, but I love myself in the way that one can only love oneself, the expectation of my own measurement, the pride in my achievements, the disgust in my faults, the understanding of my beliefs. For one cannot love others without loving oneself, for love is derived from ones understanding of their own values, who they are in all of reality.
\par Now why are these feelings and thoughts all the more present in isolation.For when with others, those are my real feelings. In isolation I attempt to derive and find, deriving and finding the strange and unwieldy emotions of the mind does not come truly and with the same accuracy of that of physics. It comes in strange botches of thought that don't mean what they literally mean but can be described through thoughtful examination of the words, other words, and the actions of the person.
\par When did I become this, so thoughtful behind it all. Seeing the world beyond the material, seeing my thoughts beyond the exact. Maybe I really am chaining, in a way that must lead to the changing of even my most basic assumptions.(this last paragraph is really stupid)






\section{On Ayn Rand}
\par It would be obvious to admit that this section would be an analysis and critique of Ayn Rand's ideas, but like the more erudite among you will notice that this would go against the structure of this chapter. This is exactly correct, this will rather be an analysis about my propensity towards Ayn Rand's ideas.
\par While this may seem useless and without purpose, it isn't. It has been a strange psychological question about my enjoyment of Ayn Rand's novels even though I starkly disagree with actual philosophy and the fact that her books lack the many of the general characteristics that generally allow me to enjoy such novels. So what is it?
\par There are several reasons, though I will start with the most fundamental. Simply put, I see myself within the characters. Most particularly Howard Roark. Many of the descriptions of his won emotions and others description of me mirror is a strange way. While not absolutely similar, it builds off in a way more closely connected than any other fictional character. How he exists as an independent entity, not noticing others but living by their moral code not out of other means but as its own mean. Because of integrity above all. The way he is an act of moral striving rather than a disgusting abstraction for those even more disgusting to connect to. He lacks those pathetic neurotic tendencies that those around me let control and give authority over them.
\par Beyond that, he gives me ways to articulate my own personal feelings in a way that I have never seen before. Being "to proud to boast" by seeing both the criticism and complements of the world to be equally insignificant because it does not come from myself. To find pride, value, truth, within myself not within the pathetic world beneath me. Seeing another person see how pathetic the social validation games are, not is disdain for others but as seeing it beneath myself.
\par The way happiness is the his natural order rather than some far flung ideal that is beyond, that negative emotions seem dulled by their pathetic attempt. That he is truly happy at all times, just as I am.
\par The way compromise, even in the slightest way seems to be evil. That this concept has been a driving force in my life thoroughly. Because it is a self-betrayal, a betrayal that can't even be thought about. An evil beyond belief and idea. Even a white lie, a broken ideal without real backing, a principle made as a child, and so much more seem evil and I can't figure out how others live with it. Though these compromises are never entertained long enough to feel anything beyond the knowledge of it.
\par The way others see him as cold and arrogant despite him clearly not being, due to there misunderstanding of him. Because he is independent, because he doesn't care about friends and what insignificant interactions they had, what their friend's did to annoy them. That they don't care about philosophy, physics, mathematics. 
\par How he is both happy in isolation and with others. Equally, because the existence of others doesn't have that effect on him.
\par The way his creative and logical thoughts of Philosophy, physics, mathematics, and other intellectual topics are all that matter, well beyond the existence so often people confine themselves to.
\par Finally, he lacks all neurotic tendencies. No desire for complements, praise, people to soften their words, bend down. These neurotic tendencies are found everywhere, in everything. People, fictional characters, and others. Though I feel such thoughts so rarely. It is a impeded idea in almost all of fictional by its own virtue, but I never see it within my own mind. He like me is truly beyond these petty neuroticism, no capacity for them at all. No vulnerabilities, no anxieties, no insecurities, no of it.
\section{On the Pursuit of Thought}
\par NFC, or Need for Cognition is a psychological concept seen clearly in this book. Though it many not be obvious the the extreme extent it is true.
\par Much of my ideas of philosophy, ethics, physics, and more may seem like a desire to know reality in its greatest extent(and this is certainly true), but in its most basic and primate way, it is my need to think.
\par I love thinking, it is my favorite thing. This is what drive me, my desire to satisfy that part of me. I think about anything complex enough; philosophy, ethics, physics, mathematic, coding, economics, political philosophy, literary analysis, world building, international relations theory, geopolitics, formal analysis, psychology, meta-cognition, and so much more. The more thinking required the better.
\par In fact, I love it the most when it is beyond me. When it takes me weeks to not even fully understand what questions to ask, when it feels beyond my comprehension, when I think for days and go no-where. I love it, the scavenger to knowledge, then to actually get it. For it to all fall into place just as it should.
\par Now very little does this, in fact most things just come. They are understood intuitively. Seem to basic to even consider. Even some of my other hobbies like psychology, politics, and such seem to basic, and most of the other things beyond my intellectual hobbies seem so basic as to not even give it time at all.
\par Another funny consequence, is despite my almost compulsion to efficiency, I still am drawn to complexity. Despite my normally physicalistic and literal tendencies I am also drawn to the abstract. I have found this most easily seen in my obsession with using higher order math. I use tensor calculus, hamiltonain mechanics, geometric identities, when similar methods can do it. I obsess over these ideas when there are more practical matters. 
\section{Nietzsche's Sovereign Man and Morality}
\par Upon further reflection, my chapter on disgrace and pride has much connection with Nietzsche's "sovereign man." Here I will first explain what that is, its connection, and divergence. 
\par Before I begin, one thing with Nietzsche's writing is that it is up to controversial interpretation. Some say the sovereign man is sincere, other ironic, others meta-ironic, others see it simply as a literary device, and there are still other interpretations out there. Luckily though, for the purposes of this exercise it doesn't matter Nietzsche's intentions, only my intentions and ideas. For instance, Nietzsche's reasoning for presenting the sovereign individual differ from mine; which I will present later.
\par Now to actually begin with the analysis. Who is the sovereign individual? The sovereign individual is the archetype of Nietzsche's will to power(the Ubermensch later takes its place as a further extreme but I care little for this idea.) It is an implied ideal in which a person creates mastery over their own life. To live their life in accordance with their values, that they themselves had created rather than inherited. They do this with their mastery over their own impulses; a mastery so extreme that they eventually shape their impulses into however they consciously wish. Further more, these values and 'morals' are created though anesthetic ideas, creativity, future-bound, and affirmation to the power of life rather than traditional fear. Finally another key part in is surprisingly forgiveness, though not in the traditional sense. Rather than forgiveness caused by God or some other values it is a combination of the acknowledgment of the fact that most people are incapable of true moral thought, self-mastery, and to let go of their resentments and impulses; another key idea is simply the fact that not forgiving hurts the sovereign individual, Nietzsche's suggests that it is better to simply forget about the trespasses rather than hold on to resentment and call it responsibility to forgive when it is hidden resentment. Essentially the sovereign individual is the archetype of the promise made flesh(or word)
\par Now how does this relate? Well before I get toe pride and disgrace, lets explore the philotimic virtue. The philotimic virtue highly mirrors the concept of creating your own values, setting your life and soul in pursuit of them, and finally holding yourself to them in a way that can be seen as fanatic to the outside. Also, creating these values individual of the existence of others(though the philotimic virtue includes the will and thought of God to yet be higher than the will to power of myself.) To live your life as you will it so. T
\par Now disgrace, the idea that my personal values that I have willed hold the same good and evil. That to betray them is evil. But both ideas have a similar them, by not hiding from it but rather carrying the burden is guilt erased. By taking the conscious action of purifying oneself one becomes worthy of forgiveness(not God's forgiveness but my personal one) and because I have achieved this, while I carry momentary understandings of mistakes unlike the sovereign man, I hold no returning 'guilt' or resentment of any kind. No neurotic tendencies, no projections onto other, none.
\par Pride, this as mentioned earlier is the same as the earlier mentioned philotimic virtue. The idea of creation and happiness as a norm that has been willed by me as an action of my pursuit of virtue and creation. This is where the disgrace truly comes in, it is reabsorbed, not as pain but as energy for the recreation of the self in the form of higher values.
\par Another key idea is that of forgiveness. Once again there is the connection between my idea and his. The fact that for the self; moral, ethical, value-driven, etc is not a same thing. It is a rapture of the very existence, but rather than waste energy on being sorry for oneself, one must use that energy on self-perfection. Though, forgiveness of others is another story. Other people are not capable of moral action, they are slaves to impulses, temperament, social constraints and much more. While people claim to be moral, much of there actions disgrace their ideals, their intentions align with physiological temperaments rather than values, and they abandon so much so easily. Their transgressions should be forgotten, or better yet not noticed at all.
\par Though, I am sure you are thinking about the differences(I mean I have been sprinkling them about these section) and while they are important to some extent. I don't want to go though a point by point refutation of Nietzsche ideas, especially when this analogy of the sovereign man is meant more an a psychological analogy to help explain intuitively my own personality rather than philosophical affirmation.(I disagree with Nietzsche on much)
\par One thing I will say though is the difference in end goal. For me, the sovereign man in a more refined sense(axiomatically driven, recursively made, God fearing, etc) is the final form. The Ubermensch is useless. The chaotic form of disparaging logic, reason, and God in forming morality disbanded the entire project and is just as disgusting as the slave morality of others.
\section{Half-Growth and Half-Death}
\par While many proclaim that my ideas of seeing people not as thinking being, not as moral agents is misanthropic and anti-social. That I should have more respect for others, that this lack of respect can lead to later personal problems with others. Though I disagree, first I will go over how I have seen others interact that makes me disagree, and my own personal evidence from my falterations with these values.(one thing I will add, is once again these are not true literal beliefs but analogies to explore esoteric identities.)
\par Think of how people treat children, then people they deem as equals, then themselves, and then those they deem as higher than them when they perceive a slight. 
\par The child's slight is either ignored or the person takes conscious action to help them, not by expecting moral action as they would with others but by understanding the child's abilities and inabilities and working around them in a kind manner.(Bar extreme causes or grotesque individuals)
\par With others there exist a range of reactions to slights, but their is a clear differentiator. They expect others that are 'equals' to exist in a semi-moral manner(this is because most people hardly understand moral manner to begin with). They try to act in kindness and understand others faults and temperaments but when push comes to shove they resent others, they expect from others. This can come in a wide variety of ranges: resentment, anger, fear, annoyance, instability, and so much more. These reactions are almost always unproductive and hurt relationships, people and such. They also use the perceived moral responsibility to negate their own moral responsibility in many cases(under my analysis, this is where most pain received by the affliction of others comes from, either consciously or unconsciously)
\par Then from here it is easy to see that the extremes of these go up and up. 
\par These are where a majority of the petty, vapid, and pathetic emotions I wrote about earlier come from.
\par Now what I suggest isn't as radical as it seems, it is just shifting the average person closer to where others see children, no one other than God as above me. That I intentionally analysis and 'handle' other people. I understand them both to better see them as pathetic not in a disgusting way but like a child. Also, to better handle them for interactions with them.

\par Though, one thing I didn't include in my earlier writing is that some people do come closer. Select people I know that I am close to enjoy(or don't depending on how you look at it) the responsibility of a human and conscious life within my own ideas.
\par Now, where does my growth and death come from? Well in recent time I have faltered, I have subconsciously found that I do expect things from others. These is so terrible in my eyes fro several reasons. One, as I have mentioned earlier, this 'equality' is the source of much of our troubles in the modern age void of true horrors. Second, for my personality and moral theory this effects me in a much more extreme sense than it does other. From a shallow perspective I am an extreme puritanical person who is obviously very prone to disdain. Beyond that I hold some values in high regards that many don't hold at all. Though, more truly than that, my in my moral theory intentions mean everything and so very few people have truly pure intentions. Now I don't mean everyone is a selfish jerk(though many are) but beyond that many are husks of people that only follow moral theory because they don't comprehend any other way, others hedonists that find minor moral action as easier, others do it because of insecurities, and so on and so forth. The concept of doing things for the mere fact they are moral/logical is lost on almost every person you will ever meet. 
\par Here I will give an illustration. Last week(as of writing this section) I had an interaction. Someone had very condescendingly given me 'advice.' Telling me in a clearly condescending and antagonistic tone to remember to unroll my selves I had rolled up to wash my hands.(this is also after of many other similar interactions with this person) Now because I knew that to explain the ethical ideas of attempting to 'assert' fake authority on another in such efforts was immoral due to the fact that on a psychological basis that such things could very easily be comprehended as attacks on their own self-determination and competency. Also, that such actions could easily be interpreted(and likely truly) as things like need for superiority, low self-esteem, extreme lack of social awareness, control issues, projection of their own incompetence, desire for a reaction, or malignant/grandiose/competitive narcissism. Though I instead let it slide and said thank you, I didn't feel it would be worth it or would give any results on the matter. Now I didn't expect them to realize their fault or anything of that extreme. I expected them to say your welcome either as a empty social connection or an understanding of the fact that I understood their game and they would either give up or hopefully be filled with self-disgust over the conscious acknowledgment of why they did what they did. Instead they didn't make eye contact and instead "hhmf" at me. I still had the rational ability to understand that this person was either too stupid or immoral to understand whatever I wanted to say on the matter, so I decided not to. Though regardless I was filled with disdain. I was disgusted by such extreme evil. I know for many this doesn't seem like evil, but in the eyes of intentions it can only be logically assumed as. Those of lacking ability of intelligence to react in moral manners wouldn't make such an extreme mistake, those powered by resentment or anger would likely be filled with shame of such actions, and so on and so forth. Likely the only remaining interpretations would be that of malignant/competitive narcissism, desire for the reaction/pain of others, and other similar grotesque ideas. Now while I see the earlier stated ideas as immoral, these take special places, especially for someone like her who has the capacity of living in dignity. 
\par Now this isn't some one shot, lately I have been more and more likely to to feel disdain for others. I don't know if it is puberty, social connection, increases in closeness with others, or whatever. All I know is that while this is something that so many other have proclaimed as what would be the greatest thing to my self-perfection is rather the worst thing.
\section{Lack of Aesthetic}
\par When I am referring to aesthetics in this section, I do not mean philosophical aesthetics, as in valuing something over another; I mean traditional ideas of beauty in music, art, and natural affairs.
\par In this sense of the word, I lack almost all aesthetic values. I have no favorite color, no concept on beauty in most things, and no care for musics, visual art, or other 'artistic' concepts.
\par This doesn't mean I have none. I have some interest in the artistic understanding in complex and intriguing literature, I can  also be temporary incapsed by complex art/music(though only as long as it takes for me to understand it, and if it is too abstract as to be useless I have little or no care for it); beyond that I can find beauty in mathematics, logic, efficiency, ideas, plans.
\par In relation with that I have no feelings of sentimentality, meek emotions, and other similar emotions.
\par Now, I have always naturally found all of these things as beneath me and childish(and still do to some extent) though I have been taking time to try to expand my horizons. 
\par As I have been doing this, I have noticed some changes. I have thought in more lofty ways than before, been more open to some experiences, and even engage with emotions in a way previously unheard of. Now here I will leave with this, but later I will explore this in more detail. What is actually causing this change, Explore what the change actually is rigorously, and whether is it actually a good or bad thing?
\section{Boredom}
\par Just as NFC is a primary driver in my life, so is aversion to boredom. 
\par The most obvious connection is simply for my NFC is grown and in a feedback loop with my aversion to boredom. I satisfy my boredom with challenging cognitive tasks. This is one of if not the largest driver in my self-education, research projects, writing this book, and so much more are driven by aversion to boredom. 
\par A little beyond that is ambition. Just as I choice challenging cognitive tasks to satisfy my boredom, so do I choice other challenging tasks. Leading, planning, mentorship, responsibility, work, physical activity, and more. This is what pushes me in SVA, Slack, JROTC, Scouts, OA, and much more. These activities satisfy my boredom.
\par I will go even further and say this desire for challenge is primary from aversion to boredom, not fear, perfectionism, and other traditional psychological reasons for people pushing themselves further than most.
\par Now, my aversion to boredom isn't just in ambition and other productive means. The most obvious is in watching TV or non-educational videos. Though this is largely to simple to take any time analyzing.
\par Though there are other things to analysis, relations. First I will go over casual relations. 
\par For some reason, as I will discuss in more detail later on, I switch between 'extroversion' and 'introversion' in a sense. When I am away from others I hold no desire to be with them, I satisfy my boredom through abstract thought, arguments with myself, physics, and other similar things. That this feels the most natural thing. Though when I am with others this switches. Getting lost in thought no longer feels natural but rather challenging and hard to focus in the same way. I naturally feel the way to satisfy my boredom is through conversation with others. It is an interesting development that I will likely look into further.


\section{Why Physics}
\par While I have many interest; math, coding, economics, psychology, philosophy, logic, and many more. Though, one interest stands above and beyond the others. Physics. Ever since I was a kid I have been obsessed with physics.
\par Now I am sure I can come up with some lofty reason on how physics is the nature of the universe in its purest form, or the most fundamental science of them all. While I am sure that it is part of the reason I like physics so much, it is not one of the primary.
\par The most easily observed is its balance of abstract thinking and practicality. It is one of the most abstract and purest forms of logic other than pure maths and some forms of philosophy, but unlike the others it has a more direct connection to real-world application. For instance, my work is working towards fusion reactors, something I think once working will be one of the most influential and greatest creations of the modern age. To combine both reduction to axioms and analytics and contructionism of creating something practical.
\par Next is the challenge. Not only is physics itself extremely challenging, with some aspects of it taking months of studying to even comprehend it, it also combines many other hard disciplines; math, coding, engineering, and even metaphysics as times. This challenge is exhilarating and as mentioned elsewhere is one of the things that i most desire.
\par On combining other things, physics is a multi-disciplinary subject. Combining and using many of my other hobbies.

\par There are many more reasons but these are the main ones.
\section{Lack of Resistance}
\par As developed throughout this book, especially in "On the Pursuit of Thought" the idea of challenge as a goal. 
\par The joy of an intellectual challenge specifically. To mull over a topic for hours. To have something beyond current comprehension. Something that doesn't make sense. Then all of a sudden explodes, not only to explain itself, but making connections all over. Like a flood. 
\par This is my greatest joy, what I live for. Though, even as a teenager I am coming to limits. I will examine this in parts, first through my self-education and then through external world interactions.
\par What is going on personally is I fear that I won't feel this feeling. Many topics like economics, psychology, geopolitics, history, philosophy, literature, and more don't give this rush anymore.
\par As an example I will examine literature. In the past year I have found interest in literature, a topic I had previously missed. It was enticing, learning about the usages of symbolic characters(and figuring out who was and what was), setting as a character, philosophy through stories, and so much more. I had that excitement, though lately I don't. While there is still more to learn, it isn't the same. Everything new, is obvious. No requirements of complex thought when the workings are already there, only new information to be feed into the pre-made algorithm.
\par Now, I still have some exiting things in the above fields and much, much more in abstract mathematics and physics, but I am only 16 at the moment. If I am already this far along now, what is to say in 10 years there will be much left, what about 40? I have already surpassed all traditional classes and all that is left is research articles and manuscripts, these will leave me for quite some time, but who knows how long that will be.
\par This also continues in my personal life, school has the intellectual engagement of watching paint dry, talking to people is a burden, leadership roles still have excitement but have less and less return,
\section{Am I Misanthropic?*}
\par My misanthropic like tendencies are clear throughout this book, but am I really misanthropic? No, well... maybe a little.
\par I do have many friends, friends that I enjoy. I like talking to people. I am truly an extrovert in nature. 
\par Though, beneath that there is a disconnect. I don't truly enjoy talking but rather staving off boredom. While in isolation I generate distrust and disgust of others. Many people I dislike more than I dislike boredom and everyday the percentage of people within that camp grows. 
\par 
\section{Depth of Feeling}
\par All of this discussion on emotions and psychology can leave a reader with the thoughts of a sensitive and emotional boy, but this is far from the case. 
\par While I use extreme words to convey my ideas, the extremity is not there. All my life my feelings have been dulled. While others are overwhelmed by feelings, for me they are hard to observe. Like they don't fully touch me. 
\par This isn't true for all emotions; my passion, excitement, fanaticism, and need for cognition are all real. This doesn't take away from my thesis. 
\par Beyond these, I have always required conscious effort to hold on to emotions. While this may sound like lamentations, it is not. It is a great thing to forget anger and pain because of there insignificance. To never feel overwhelmed or anxious, to have control over my thoughts and actions, I am in control. I love that I am this way. 
\par There was once a time that this made me feel inhuman. While others talked about there emotions, though that feeling like all has passed. Though, don't take this too far, I am by no means some robot devoid of all emotions, just simply that for any population with variance some people will be above or below average at things. For me I am simply far enough that observing others like me is a rare enough occurrence to think it truly is rare(but it really isn't.)
\section{Language Usage*}
\par Before I actually begin upon the main idea of this section I feel the best way to introduce this is to explain the inspiration of this section.
\par The inspiration is the oddest place place possible; Ted Kaczynski, or the Unabomber.
\par I remember when I was a kid I always hated the idiom "Have your cake and eat it too." The reason for this is due to the fact that the have your cake is placed first, and this fit(in a way that doesn't fit with the idiom.), because you must have your cake to eat it. So, I started to switch it around to "Eat your cake and have it to" though never really liked that. Until I read that part of the reason the Unabomber was caught was due to his strange phrase "Eat your cake, and have it all the same." When I had read this my first thought was insult that I hadn't thought of that myself.
\par Once I thought of that I realized more connection. Some more 'normal.' Starting with a thoughtfully researched and observed effect being the fact that highly educated members of fields like physics, mathematics, logisticians, computer science, and philosophy use mathematical and logic terms in regular speech. This is because their thorough and clear definitions have clear applications. Axioms, manifolds, prior, bias(in a mathematical sense), fallacy, paradox, induction, inference, deduction abduction, ambiguity, equivocation, disjunction, mapping, inversion, duality, convergence, divergence, topology, entropy, singularity, resonance, phase transition, field, fractal, lattice, span, gradient, symmetry breaking, null, second order, and more. This isn't that odd considering that it is more wide spread, I mean a couple of those terms described above have even introduced themselves into colloquial speech(though used slightly differently.)
\par Another is using words that were created in academia, then introduced into regular speech that then changed it, in their original usage.
\par Though some are a bit stranger, mainly the fact that I simply create my own definitions for words or even create my own words.
\\
\par So lets go through these one by one.
\par The first is very simple, these words are very useful in regular speech, convey abstract ideas easily, and can be applied easily. The mere fact that I understand them makes it only logical that I use it.
\par Next, the usage of academic definitions has a similar requirement. The fact that these academic definitions are generally more completely and logically defined than generally used word. 
\par Finally, the changing of words and creation has a similar efficiency and logic.
\par Though, this brings new questions, why do I feel so comfortable with this, what criteria do I use for greater 'logic,' what is with this obsession with logic, why don't I just pick either change current words or create new ones instead of do a bit of both, and much more.
\par ...
\section{*Infinite Responsibility}
\section{*Handling Others}
\section{*Need For Coherence}

\section{*The Will to Power}
\section{*Rigidity \& Natural Change}
\section{*Need for Coherence}
\section{*On the Internal Promise and Dignity}
\section{*Zealousness in all Things}
\section{*Duality of Extroversion and Introversion}
\section{*On Ambiguity} 
\section{*Cognitive Style}
\section{*Internal Authority and External Power}
\section{*The Ideal within}
\section{*Internal Control and Fear}
\section{*External Understanding}
\section{*Transcendence and Delusion}
\section{Euphesus*}
\par As I mentioned earlier, there has been change in me as of recent, and it has recently come to head with the advent of 'dating'(not actually dating the girl formally[will explain more later]{Also not completely sure it is 'coming to head' just more extreme than before}).
\par What I mean, is for the first time in a long time I feel conflicted. As strange as it may sound, I have  been complete for a while now. I never feel internal conflict, neuroticism, second guessing myself, or anything of that such. I have for a very long time clearly defined my morals and values and thus never needed to second guess myself. While I am not perfect, the few times I make mistakes, I clearly identify them after the fact and rectify them. There is no lingering guilt nor do I go back and forth between what I should have done or should do for the future. 
\par Now, in the last year or so, I have had some rare conflicts; many of whom I have talked about in earlier sections. Though with this ambiguous relationship, it has become an almost daily thing.
\par I went back and forth between whether or not I should ask her out in the first place. When I did I had a lingering regret about doing it over text rather than in-person. Later when texting her after the fact, I felt the inexplicable desire to talk to her, without any real end-desire other than the conversation. First, I felt conflicted because I have never felt the such a desire, in fact I often was disgusted by others for this desire. It always seemed beneath me, while it wasn't against my values, for someone who has always been so 'complete' and whose entire personality and temperaments are recursively defined by myself the mere fact of something new is insane upon the face of it. It is like a self-betrayal which it took me a while to figure out what the betrayal was in the first place. Furthermore, this caused me to start questioning thing, why I didn't desire to talk to my friends about nothing in particular, should I start doing it, etc. It even lead me to try it. Then this started to make me think about how some random girl I am not even formally dating, has caused me to alter my own behavior. Just slightly, but nobody alters me even slightly.
\par Once I had finally gone through all of that and moved on, I decided I would talk to her over text. During that time I neurotically tried to when would be a good time and what to say. Once again, I don't do that. I don't care what other people think of me, I don't care if the message wasn't perfect, I don't feel neurotic over anything, certainly not a person. Yet I did, I thought about it. Finally, I texted her, it went nice.
\par I decided I should keep in touch, because I wouldn't see her in person for a month. So I decided to reduce ambiguity I would simply have a schedule of when to initiate a conversation. The number of days between initiations was meaningless and I knew I just needed some arbitrary number, yet I thought about what would be an embarrassingly long amount of time.
\par Later, came the day for my next initiated conversation. and once again I spent way to long figuring out what to say, but when I did, she hadn't responded. I didn't think much of it, due to the fact there were dozens of possible explanations for the lack of communication that were fine. That was until the next day, when she responded on a group text, but continued to ignore the individual one. Now even still, hours later I am still thinking of it. Still conflicted, wondering if asking her out was the right choice, or my number, or what I said. Wondering what my next steps should be. 
\par While yes, I understand these emotions are normal for an adolescent boy of my age, but they aren't for me. Not only am I rarely conflicted about anything. I am much less about people. I am the kind of person who interacts with people for amusement or out of responsibility; deep down I don't care other people in the way required to feel these neurotic thoughts and emotions.
\par It is so strange, having the last couple years of my identity and foundation so clearly defined for everything. Only to bring out something new, something that doesn't fit. Now I don't know what to do; stay rigid or change(but change what) 
\\
\par Over the last month or so I have gotten over the visceral reaction. Now that I am back from my trip we can see each other in person, but haven't very much. 


\chapter{Politics}
\section{*Political Philosophy }
\section{*Politics}
\section{*Economics}
\section{*International Relations Theory}
\section{*Geopolitics}


%%%%%%%%%%%%%%%%%%%%
\chapter{Plans}
"Unless you try to do something beyond what you have already mastered, you will never grow" | Ralph Waldo Emerson
\section{High School}
\subsection{Introduction}
\par Right now I am in high school and for it I have many ambitions, both for their own sake and to set myself up for future success in my other plans. Here I will exclaim my current ambitions. Their reasoning, why I plan to pursue them, and how to achieve them.


\subsection{Self-Education}
\par Self-education has always been a great value of mine. As mentioned earlier, I became obsessed with learning Einstein's field equation in 3-4th grade. Beyond that I have spent much of my time reading books, watching YouTube videos, and more about topics that interest me. Though, late in 8th grade, this desire to teach myself grew a great deal. Before I can go about my future goals for teaching myself I must first go over what I have already down.(This list is not full, mainly lectures finished and books from lectures, this does not include books not finished, non-lecture based video education, research papers read, projects where I learned things for the project alone, and other similar programs)
\\
Mathematics:
\begin{itemize}
    \item AP Calculus AB and BC - Khan Academy
    \item Multivariable Calculus 
    \item Khan Academy
    \item 18.03 Differential Equations - MIT OpenCourseWare
    \item Gilbert Strang on linear algebra - MIT OpenCourseWare
    \item Vector Calculus - Trevor Bazett
    \item Tensor Analysis - eigenchris
    \item Tensor Calculus - eigenchris
    \item Crash Course in Complex Analysis - Steve Burton 
    \item Introduction to Applied Numerical Analysis - Richard W. Hamming
    \item Symplectic geometry \& classical mechanics - Tobias Osborne 
\end{itemize}
Physics:
\begin{itemize}
    \item 8.02 Physics II - MIT OpenCourseWare
    \item 8.03 Physics III - MIT OpenCourseWare
    \item 8.033 Relativity - MIT OpenCourseWare
    \item General Relativity - Stanford Online
    \item Introduction to plasma physics - USYD senior plasma physics lectures
    \item Introduction to electromagnetism - Griffiths
    \item Computational physics - Mark Newman
    \item 8.224 Exploring Black Holes: General Relativity and astrophysics - MIT OpenCourseWare
    \item Introduction to cosmology - Stanford Online
    \item Introduction to fusion energy and plasma physics course - PPPL
    \item Seminar: Fusion and plasma physics - MIT OCW
    \item Statistical Mechanics - Stanford Online
    \item Hamiltonian description for magnetic field lines in fusion plasmas: A tutorial - AIP
    \item Fusion economics: power density, materials and maintenance - PPPL Frontiers Colloquia
    \item Flash-X code tutorial, a users perspective - University of Chicago 
    \item Flash-X user guide - Flash-x
    \item Flash4 User support
    \item Radiative Procuresses in Astrophysical Phenomena- George B. Rybicki, Alan P. Lightman
    \item Goldstien Classical Mechanics Lectures - Prof. Jacob Linder
    \item PiTP 2016 - Institute of Advanced study
    \item 2024 PPPL Graduate Summer School
    \item Goldstein Classical Mechanics Lectures - Jacob Linder
\end{itemize}
Coding:
\begin{itemize}
    \item Computational physics - Mark Newman
    \item Python Numerical Methods - Berkeley 
    \item Applied Numerical Methods - Crice Carnahan, H.A Luther, James O.Wilkes
\end{itemize}
Other:
\begin{itemize}
    \item Management in engineering - MITOpenCourseware
    \item Principles of microeconomics - MITOpenCourseware
    \item Dynamic leadership: using improvisation in business  - OpenCourseware
    \item Logic 1 - OpenCourseware
    \item Policy for science, technology, and innovation -  MITX
    \item Reducing The Danger Of Nuclear Weapons And Proliferation- MITOpenCourseware

\end{itemize}
\\

\par Now that I have shown what I have already done, now to look for the future. The problem is, as I have continued with my education, finding ways to increase it becomes increasingly hard. Most is found in research papers hard to understand and harder to find, the books that exist have similar problems, and few if any programs outside of this truly exist. 
\par Though, this shouldn't all be negative, some things do exist, and as seen clearly, while my mathematics and physics programs are very high, other topics still need greater research.
\par In the fields of symplectic geometry, gauge theory, and other high-level mathematical physics, there still exist many things for me to learn. As of writing this paragraph I am working through "On Sympletic Reduction In Classical Mechanics" and "Infinite-Dimensional Lie Groups and Algebra In Mathematical Physics"
\par Beyond that, I am very interested in plasma physics, so continual research in different modeling theories like PIC, MHD, Kinetic theory, and the several others. I am currently working on "Advanced MHD"
\par Beyond that I would like to go further in other fields that I don't have the same high level understanding; I want to learn more about PNNN(Physics Neural Networks), advanced statistical analysis, and advanced coding with regards to making physics models. 

\subsection{Research}
\par Once again, I must start with what has already happened. 
\par I have done several small research projects on my own time Finite volume simulation of an egg drop, N-body simulation, and simulated annealing project for plasma thruster dimensions.
\par A bigger project of mine has been the continual creation and updating a MHD simulation tool in FORTRAN, currently it contains ideal MHD plus heat transport. I also wrote a paper "Numerical Methods for 3D compressed Plasma using Lattice Boltzmann." Later I created a research project "A Hamiltonian Framework on ICF Implosions Rocket Equation Based on Rayleigh-Taylor Instabilities."
\par Another project, a failed one, has quite the funny story. I was reading a paper on the effects of a primordial black hole hitting the Earth, when reading I had thought of how the black hole would create an accretion disk, and thus an EMP. I figured this would be an easy thing, given the previous research on the topic by the US government. Sadly, much of the research is classified, and I spent over two months searching for something. Eventually, due to my compulsory need to complete actions, I ended up writing to the FOIA(Freedom of Information Act) asking for the classified documents. I am highly annoyed by the fact it has been over a year and I still haven't gotten the code. 
\par Now, this is all stuff I have done alone; I am also working with the University of Tennessee on a research project. We are working on gravitational wave simulation of Oak Ridges recent simulation of core-collapse super-nova. I mainly work on mathematical analysis. 
\par Now what do I want to do for the future.
\par I wa, nt to continue to add and improve upon my MHD simulation. Planning on making it so I can make a semi-realistic Tokamak simulation.
\par I also plan on making a Hamiltonian MHD simulation.
\par I also plan on working on making PNNNs as mentioned earlier.
\par Beyond these coding projects, I am not sure. As mentioned earlier I plan to continue my education in Hamiltonian mechanics, high-level mathematics, plasma physics, and such. Hopefully, I can find some intersections to continue my education.
\\
\\
\par Update: I have advanced the MHD simulations time steps, magnetic field calculations, current, pressure calculations, and density calculations. Beyond that I have created animations based on magnetic field and made graphical outputs for heat, velocity, temperature, and magnetic field.
\par I also created another one specifically made for Tokamak geometry.
\par I created a basic hamiltonain MHD simulation, with specific simulations in plasma propulsion.
\par I also created a death star simulation, as well as n-body fluid gravitational simulations.
\subsection{SVA}
\par SVA, which stands for Scientific Visionaries of America. It is club I started at my school. We are working on putting on a local science fair and we  also compete in science competitions: science Olympiad, science bowl, MTSEF, and JSHS.
\par Beyond simply preparing for these activities we also have our own fun activities. We have had philosophical discussions on the nature of science, had a round table on space exploration, and we are working on an egg drop experiment. We also do classes on advanced topics like physics, statistical analysis, or astronomy.
\par Now, my plans for next year are likely a continuation of what I did last year, simply better than before.
\subsection{Boy Scouts}
\par I am an active member within a boy scout troop. Previously I was Senior Patrol leader, this is where I truly found how much I enjoy leadership. The constant drive forwards for a goal, to  make decisions, to see nothing else other than the goal and to put all of your mental focus into the achievement of it.
\par Now, I am an Eagle scout. I am still relatively active in the troop, going to meetings and campouts, being a kind older scout for those around me. To be a guiding hand, especially for the youngest of scouts that the current leadership has a hard time controlling.
\subsection{Order of the Arrow}
\par I am a member of the Order of the Arrow. I am currently Chapter Chief and love doing that. I am also active in volunteering for helping with campouts. 
\par While I definitely do have future ambitions in this area, I am not exactly sure what direction to go in at the moment.
\subsection{Slack group}
\par Now what am I even talking about? Several months ago I created a slack community for high achieving high school students interested in physics research. I want to make this community interacting in a way to facilitate the skills, community, intellectual interactions, and more that would be found in actual research communities.

\par Here is a history of the past of this program.
\par The first thing I did was email different organizations that I thought would be able to access the types of people I want in the program. 
\begin{itemize}
    \item STEM learning institute 
    \item NSHS
    \item MIT
    \item Stanford
    \item Princeton 
    \item Caltech
    \item Colorado boulder 
    \item Sigma Xi, research society
    \item Young scientists journal
    \item Journal of emerging investigators
    \item RSI
    \item JSHS
    \item APS
    \item American Association of physics teachers
    \item ISEF
    \item National science teaching association 
    \item Dod STEM
    \item Pathways to science
    \item Institute of Physics
    \item American Institute of Physics (AIP)
    \item International Association of Physics Students (IAPS)
    \item National Science Teachers Association (NSTA)
    \item STEM Education Coalition
    \item Youth Science Canada
    \item European Physical Society (EPS)
\end{itemize}
\par I also put some messages on some reddit and Discord groups. I had limited success but I eventually ended up with 30 members.
\par As people joined I tried to get people interested. I started trying to get people to get to know each other through only a couple of people responded. I also added a journal club were I sent weekly papers to read, nobody interacts. I sent weekly physics puzzles, questions, and open-ended questions: nobody responded.
\par I created several different sub-sections of physics for specific chats; I also added an advice section, other, social, or random. Finally, I tried to create a mock symposium/conference. I had a couple of people interact but never more than one at a time, so the meetings didn't ever end up happening. 
\par Over time I was dropping off in intensity until I eventually stopped altogether.
\\
\par As of the present I have renewed energy and really want this program to work.
\par I sent out a message for a planning meeting about setting this community up for the future. So far nobody has respond but I plan to sent out the link for people to join as an optimistic case.
\par The things I want to talk about are.
\begin{itemize}
    \item Brainstorming the vision in more detail
    \item Going over activities, what to end and what to continue. So far there are the existing mock conference, Journal club, Weekly questions. The I have some more ideas; Collaborative coding projects/research project, Model UN(but for science)
    \item I also plan to add roles for people to have. This could give people "ownership," get people passionate, get those motivated by ambitions goals involved, and split out my own work. So far my ideas are; Code captain, Event host(for separate things), \underline{\hspace{1cm}} captain(for each subsection), European leader(almost half of the community is from Europe)
    \item I also want to brainstorm name ideas for the community.


\end{itemize}
\\
\\

\par Nobody came to the meeting. I sent some messages about coming up with a new date, nobody responded. After a week I ended up just sending messages of what would have been discussed at the meeting and inviting discussion through slack. Nobody has responded as of yet. 
\par Currently my idea is to individually message those in the program I personally know or have been traditionally active to try to convince them to take additional effort to help me make this dream a reality.
\subsection{JROTC}
\par JROTC is another activity I have been getting more and more active in. Recently I have been promoted to S1, administration work. 
\par My current ambitions for this work are to have a master sheet keeping track of cadets; Rank. PT scores, service hours, attendance, uniform grades, awards, and possible action scores. I also plan on having extra sheets that auto update each other that company commandeers can use. I am also planning on working with the S5 and S3 to create a website for our program. I also plan on adding reviews so we may better our program. Lastly, I am working on create a query system for the master sheet to get information more easily.
\subsection{Lógos}
\par Now it may be strange to put a section of your book to be about your ambitions of writing said book, but this is not a regular book. It is a continually updated project.
\par Now, first discussion of the mission of writing this(as in the book in totality, not the section) is required. 
\par Originally, the purpose of this book was to refine my philosophical ideas. As mentioned earlier I had come across the idea of quantum consciousness and was playing around with it. I was originally extremely against it(still am) but I wanted to examine it nonetheless and I found that even my working memory and ability to think of multiple abstract concepts together at the same time and have them influence each other was limited. So I decided to write it down, and had quite some fun, even going beyond my stated goal. So I expanded to other branches of philosophy. Then I realized this strategy could help me better understand physics(my true over-arching goal). I could write notes, explain things to myself, work on articulating ideas, find gaps in my knowledge, propose projects and have the physical reminder both to complete them and that completing them matters.
\par As I continued to do this, I had fun, so I continued to expanded. Writing about my ambitions, culture, random thoughts, and more. It was supposed to be like the conversations I had in my head only more refined.(Though, despite my conversations in my head being true dialogues, my writing has turned more into the style of lecturer with questions, debates, and answers.)
\par Another thing I have had fun doing is inputting my work into AI, like grok or ChatGPT. It can help me find connections to other people's ideas, resources, and more. Though, one fun thing I have found it asking the AI to psychoanalyze the writer. It usually starts of very bland and basic, with most of the stuff being common sense. Though as I help it by asking pointed questions, it speculates more. Gets more right, but also more wrong. It is quite fun, seeing it piece together what it can(it is sometimes surprising seeing what it can piece together) and what it gets wrong and why.(Though after enough questions due to AI's "spiritual bliss attractor state" it eventually drifts off.)
\par One common theme that it gets wrong is the idea of isolation. This makes sense given the fact that I am writing this in isolation, about myself, for myself. Though, in all reality I am not an isolated figure. While I am not overly zealous with making friends, can be a bit disgust sensitive and misanthropic, and other similar things. I do have several friends. Also, my less social traits are normally hidden in social settings so very few if anyone would see me in that light. I am an assertive, extroverted, funny, and talkative kid who gets along with others. Though, don't take this too far. I still do have asocial behaviors. Though not to the extreme one would expect of a 16 year old writing about ethics. Another theme is the idea of "moderate neuroticism" because such things are common in gifted adolescents and the intensity of my words suggest it. I lack neuroticism in an extreme that it is almost a deficit. The 'intensity' of my words is simply a lack of better words to describe it, or a consequence of the spewing of words either without though or with current obsession of thought, not actual feelings on the matter. It mistakes the fact that these are the few times to this is a map of how I am constantly thinking of them.
\par Now back to the original purpose. What to do with Lógos? Well, I will continue to use it to refine and understand my own ideas and how they influence each other without coming to the limits of my working memory. I will use it for notes from readings, projects, manuscripts, whatever it may be. 
\par I am kind of wish to create more chronological study of starting with basic assumptions of epistemology and logic, to develop a full theory, apply that to metaphysics and mathematics, which can be applied to create a formal theory of physics. Rather than a random assortment of essays that sometimes refer to each other. Though, that would be a lot of work.
\subsection{*Substack}

\subsection{Summer Research Programs}
\par Obviously a summer research program, would be a huge boost.
\par It would look great on college applications, be fun, I would learn stuff, I would be challenged, and I would get to meet with other kids like me.
\par Though I have run into some trouble. Most of these are obscenely expensive. The only too exceptions being RSI, Clark Scholarship, Simons Summer Research Program. 
\par Both of these are extremely competitive and I have no idea if I even have a chance of getting in. Though, I will still try.
\\
\par Another chance is Tennessee's computational physics governors school, though I doubt I would learn much from that.
\subsection{College}
\par For a long time now I have said that my college of choice would be the Air Force Academy. Due to its resources, people, academics, boosts, and that I can ski there.
\par Though there are some problems. The primary one being that I am a picky eater. While this may seem like a small thing, it has been a surprisingly powerful control over my life.
\par Beyond that, I wouldn't have as many physics based opportunities, and while the people there are probably more similar to those I would find in most places, I am not sure it is the exact community I am looking for.
\par So where? Well, USAFA, is still very high on the list I am trying to branch out.
\par Somewhere like Princeton, MIT, or Stanford would be ideal, but the likelihood of me getting into one of those is extremely slim.
\par Some other places like University of Colorado, University of Texas Austin, or University of Tennessee Knoxville would be my next choices.
\\
\par Ideally, I would find a place that has opportunities, both academically(to have someone as advanced as me still be intellectually challenged through graduate level work), professionally(internships, work, research projects, etc.), and even fun(skiing, mountain bike riding, climbing, hiking, and other similar activities)
\par Also, a place with people like me. Hard working, self-authored, intelligent, deep thinkers who wish to be challenged. To go above and beyond; not just academically but intellectually, morally, physically, and in all matters of life.
\\
\\
\par Time has gone sense I wrote this. I am now going to Tennessee Knoxville. This is due to the fact that I am graduating early so I am staying near home, already have a relationship with Prof. Mezzacappa, and am likely to get a full academic scholarship.

\section{Undergraduate}
\subsection{Goals}
\subsection{Classes}
\par To achieve my goal of graduating as quickly as possible, I need to come up with a plan.
\par For communication through writing, I have already done two course, so for the final one I will do: I can possibly test out of ethical theory, do mathematical education, or Honors: Special Topics in the Natural Sciences 
\par For oral communication I will take PHYS 451 - A Survey of Contemporary Physics.
\par I will naturally complete quantitative reasoning
\par For arts and humanities, I will take two Philosophy courses, I will look into testing out of PHIL 101 - Introduction to Philosophy.
\par For civilization, I will take honors development of western civilization(I could probably test out of this, though I am not confident enough in this one) then I will probably take honors special topics
\par For science, I will naturally complete this
\par For social studies, I have one credit from AP HG, I can test out of intro to Economics easily.
\par (This adds up to 14 courses)
\par With "A Survey of Contemporary Physics," development of western civilization, and honors special topics in civilization being classes I should take.
\\ 
\par Next degree specific courses:
\par I have already done ap physics 1 and c, so that knocks two off. For the fundamentals. 
\par I also need: 
\begin{itemize}
\item COSC 102 - Introduction to Computer Science 
\item MATH 141 - Calculus I 
\item MATH 142 - Calculus II 

\end{itemize}
\par I can definitely test out of calc 1, can study for calc 2. I also can study to test out of computer science.
\par I also need all of these
\begin{itemize}
\item PHYS 250 - Fundamentals of Physics: Modern 
\item Mechanics PHYS 321 
\item Thermal Physics PHYS 361 
\item Electronics Laboratory PHYS 411 
\item Introduction to Quantum Mechanics PHYS 421 
\item Modern Optics PHYS 431 
\item Electricity and Magnetism PHYS 461 
\item Modern Physics Laboratory
\end{itemize}
\par I can probably test out of mechanics and possibly electricity and magnetism and optics.
(adds up to 13 courses)
\par Making modern physics, electronic lab, quantum mechanics, modern lab, and thermal being classes to take.
\\ 
\par Finally, I take these classes:
\begin{itemize}
    \item PHYS 312 - Mechanics II 
    \item PHYS 361 - Electronics Laboratory 
    \item PHYS 411 - Introduction to Quantum  
    \item PHYS 431 - Electricity and Magnetism I \item PHYS 432 - Electricity and Magnetism II \item PHYS 293 - Introduction to Research  \item Independent Study
    \item PHYS 498 - Honors Thesis in Physics
\end{itemize}
\\
\par That means I need roughly 10 elective classes, here is what I would prefer:
\begin{itemize}
    \item PHYS 441 - Introduction to Computational Physics
    \item PHYS 494 - Special Topics in Physics 
    \item 2nd PHYS 493 - Research and Independent Study \item PHYS 451 - A Survey of Contemporary Physics
    \item PHYS 531 - Mechanics(Maybe, graduate)
    \item PHYS 541 - Electromagnetic Theory(Maybe)
    \item PHYS 571 - Mathematical Methods in Physics I(maybe)
    \item PHYS 573 Numerical methods in physics(maybe)
    \item PHYS 615 Astrophysics(maybe)
    \item PHYS 643 Computational Physics(maybe)
\end{itemize}
\par If I can't take the graduate courses, I will likely take:
\begin{itemize}
    \item Honors Differential geometry 
    \item PHYS 490 - Senior Seminar
    \item MATH 431 - Differential Equations II 
    \item MATH 471 - Numerical Analysis
\end{itemize}
\par So this leads to me with:
\begin{itemize}
    \item PHYS 451 - A Survey of Contemporary Physics
    \item honors development of western civilization
    \item Honors special topics in Civilization
    \item PHYS 250 - Fundamentals of Physics: Modern 
    \item Thermal Physics PHYS 361 
    \item Electronics Laboratory PHYS 411 
    \item Introduction to Quantum Mechanics PHYS 421 
    \item Modern Optics PHYS 431 
    \item Electricity and Magnetism PHYS 461 
    \item Modern Physics Laboratory
    \item PHYS 312 - Mechanics II 
    \item PHYS 361 - Electronics Laboratory 
    \item PHYS 411 - Introduction to Quantum  
    \item PHYS 431 - Electricity and Magnetism I (may test out)
    \item PHYS 432 - Electricity and Magnetism II \item PHYS 293 - Introduction to Research  \item Independent Study
    \item PHYS 498 - Honors Thesis in Physics
    \item PHYS 441 - Introduction to Computational Physics
    \item PHYS 494 - Special Topics in Physics 
    \item 2nd PHYS 493 - Research and Independent Study 
    \item PHYS 451 - A Survey of Contemporary Physics
    \item PHYS 531 - Mechanics(Maybe, graduate)
    \item PHYS 541 - Electromagnetic Theory(Maybe)
    \item PHYS 571 - Mathematical Methods in Physics I(maybe)
    \item PHYS 573 Numerical methods in physics(maybe)
    \item PHYS 615 Astrophysics(maybe)
    \item PHYS 643 Computational Physics(maybe)
\end{itemize}
making it 28/27 classes I have to take
\\
making the classes I have to test out of:
\begin{itemize}
    \item Honors ethical theory
    \item Introduction to Philosophy
    \item Intro to economics 
    \item COSC 102 - Introduction to Computer Science 
    \item MATH 141 - Calculus I 
    \item MATH 142 - Calculus II 
    \item Mechanics
    \item Electricity and Magnetism(Maybe)
\end{itemize}
Here are some other classes I think I could test out of for electives
\begin{itemize}
    \item MATH 371 - Numerical Algorithms
    \item MATH 411 - Mathematical Modeling
    \item MATH 425 - Statistics
    \item MATH 424 - Stochastic Processes
    \item MATH 435 - Partial Differential Equations
    \item PHIL 435 - Intermediate Formal Logic
    \item PHIL 373 - Philosophy of Mind
    \item PHIL 371 - Epistemology
    \item PHIL 360 - Philosophy of Science
\end{itemize}
I can do this through CLEP 
\begin{itemize}
    \item Calculus
    \item Principles of Macroeconomics
\end{itemize}
and ...
\\
\\
This summer I would like to take:
\begin{itemize}
    \item Mechanics 1(def)
    \item Calculus 2(if I can't test out)(definitely)
    \item Differential Equations(def)
    \item Calc 3(wish I could, probably not)
    \item Honors Studies in History - 86903 - HIST 397 - 501
\end{itemize}
\subsection{Research}
\section{*Post-Graduate}

\section{Early career}
\par I want to start my research(though more as a technical management role) in a bridge between abstract mathematical physics and usable applicable tools.
\par At the moment I am thinking about using things like Hamiltonian Mechanics, Lie theory, etc. to create computational models. This is something that affects my goals at the moment.
\\
\\
\par To be exact, I want to focus on Hamiltonian Chaos theory and structure preserving models for plasma physics.
\section{Late career}
\par Now what do I truly want out of my career? Well, some things clearly arise: intellectual challenge, movement forwards, challenge(general), able to influence for the better, have authority over my environment, and other similar ideas.
\par Through this, I believe that the best career for me would be a technical executive level role within a national laboratory or other similar organization.
\par Currently, my aspiration is to become the Director of Princeton Plasma Physics Laboratory.
\section{*Post-}
%%%%%%%%%%%%%%%%%%%
\chapter{Intellectual Disdain center}
"The mediocre are always at their best when they are praising one another" - Cyril Connolly
\section{Introduction}
\par Many of the 'intellectuals' that we hold to such extreme heights, were really morons. That got to those heights though grift, pull, or status. Many were at best insane and by thus interactions 'creative' in the sense the paranoid schizophrenic's capacity to create the conspiracy theories is creative. Here I will examine these people.
\section{George Orwell}
\par Now many see George Orwell as some genius beyond genius, I disagree. First I will go over flaws in his books and ideas, the psychology behind his pathetic life, and finally why people admire him.
\par Let us first start with Animal Farm, while many claim it to be a striking critique of communism. It really missed the mark on all levels. First, it showed the basis of the ideas 'correct.' Ideas like the Russian Monarchy and American Capitalism to be equal evils, the priest to be evil, and claims that the problem was the people in charge.
\par He spreads the myth that it was a problem with revolutionary leadership, not communism. Forgetting about entrepreneurship, incentives, bureaucratic handelings, and price signals. These were what lead to the collapse of the USSR, not that Stalin was evil for the sake of it. Orwell, leads the reader to believe that if Snowball(Trosky) were in-charge things would be different. They wouldn't. Just the same failed radiology based upon faulty philosophical, economic, and sociological theories.
\par All in all, the book brings nothing of value. "Be vigilant of authoritarian leaders" is about the most, but this in the abstract is common sense and the book itself lack the illustration to bring such abstraction to the practice. Leading to nothing, just a children's book on talking animals.
\par Now 1984, the (insert political enemy) here fan-fiction of the century.
\par It is obscene that people even entertain this idea. Of a society formed suddenly without any cracks or mistakes, created by elites, that scarified their own power, ability, wealth, comfort, glory, honor, morality, religion, and more just so their own children would be forced to sacrifice it. Their society exist without any reason. They had no incentive to create this society. They gained nothing, they simply hurt the people in the future by hurting themselves. They choose not to win a war, to keep their children from power, to devolve moral systems, to hurt their citizens, reduce their own comfort, and so much more, and for what? For it to happen, nothing more.
\par From this perspective we can see how the entire world building falls apart(and so do the "themes")
\par This idea of complete surveillance without reason. Not out of safety, personal economic gain, or anything of the such. It just exist because it used to exist, and why it was created in the first place is lost. It is efficient in a world were nothing is efficient, and the people inside have no incentive; ideological, moral, cultural, personal, or otherwise to keep it that way. It is a lazy cop-out to talk about surveillance, and even on that level it fails. Missing the target, the whole reason to observe surveillance in literature is to see how our own natural impulses lead us to totalitarianism not simply to point and laugh at the other side(many use this book as a way to look down on the 'other side' and I will later explain why.
\par Next, language. This has many of the same problems. Though here it fails even more so. It is based upon linguistic determinism, an idea that has fallen out of established fields.
\par Reality is mutable, lies come out. Even the most extreme totalitarianism is incapable of mutating reality as seen in the book. This 'mythological horror' really brings nothing to the table. It is just a fantasy concocted by troubled mind.
\par Individualism is destroyed, individualism is a part of humanity. You can't take it away. In the real world, people try to either convince the populous that their individuality is wrong so the populous temporary suppresses it, but the idea of a government doing it by whatever big brother does is nonsense.
\par Finally, the perpetual war. What really sets 1984 above the rest is the concept of perpetual war. A nonsensical idea.
\par On a tactical basis, it is humanly impossible to have a controlled war that stays rigid forever. Wars are unpredictable and can't be held in any way. Next, the war "absorbs economic surplus." This fails on the fact scarcity is a horrid way of keeping people in place. Beyond that, emotions degrade, people become angry at governments for a couple of year long wars, specifically ones that don't effect their geographic area. A war far away that exist perpetually couldn't last. Next, the fact that the entire thing is created as a suppression tactic couldn't last. People choose not to win the war; not invest in science, not commit too many resources, or do anything of that sort is insane. People want to win wars; whether for public opinion, weariness, political power, geopolitical power, glory, honor, moral code, etc. There is no way that the purposefully deprived political leaders would be also taking masterful actions to once again, hurt themselves.
\par Finally(there is more but I won't go into it), the enemy isn't defined. They are capitalist, or communist, or even jews throughout the book. No defined ideology, just the perfect (inset enemy here). Nothing is defined, expect as I have mentioned, the fact that they shot themselves in the foot so that the bullet will still hit you.
\par The list of stuff goes on and on. As you can see, 1984 is superficial at best and hurtful at worst.
\par Now, after all of this, a new Orwell comes to mind. A pathetic man converted into extreme resentment and fear of others because he can't handle his own failures, extreme paranoia, and simplistic world view.
\par We can see this best in Homage to Catalonia. 
\par It starts with him joining the first revolutionary communist group he find. A teenage group that has few supplies. He later complains the entire time and blames the consequences of his own action on conspiratorial activities.
\par He calls people of more nuanced ideas evil and reduces the complex multifaceted civil war to a moral fable. He also as a messianic tone and mimics the authoritarian mentalities.
\par Beyond that, he calls the looting of churches, stolen wealth, and killing citizens a "state of affairs worth fighting for."
\par He thinks that the POUM, is being systematically erased just as his pathetic Big Brother does. His paranoia then distorts the rest of the book.
\par People only like him because they can use his books to inflate their own intellectual self-worth, or they simply fall it because others consider him intellectual. A bit better is those expecting intellectual ideas look so abstractly that they see beyond Orwell's actual ideas and end up coming up with their own individual ideas seperate from Orwell's.
%%%%%%%%%%%%%%%%%%%%%%%
\chapter{Literature}
"Words are, of course, the most powerful drug used by mankind" | Rudyard Kipling
\section{Introduction}
\par As I read new books I will write notes, analysis, critiques, etc o them. This is not  saying these are my favorite books or that I agree with them. Just I am currently reading them. Many of my favorite books I have already read and most likely won't reread.
\section{*Archetypes and Realism}
\par In modern fiction, the idea of archetypal figures in fiction has all but disappeared. This has primary been in the name of realism and connection with the audience.
\section{*The Hero's Journey and its Future}
\section{*Purpose of Reading}
\section{*Is Good Literature Timeless or Timely}
\section{*The Fall of the Modern Author}
\section{Book Notes}
\subsection{The Karamazov Brothers}
\par While generally I dislike books that appeal, and attempt to have their characters seem more human. I am interested in this one. 
\par One, is it is a great eye into other. How even the 'virtuous' are still slaves to there temperament and their past. That they are examinable, not by the logic of their axioms but their history.
\par Another is that it reinforces my own idea. My own idea of self-ownership and authorship. To control not only my life, but myself.
\par Here will be a series of my notes while I read the book.
\\
\\
\par I am early on in the book, and Fyodor keeps talking in this strange lofty tone and usage. I wonder if he is written as to be so depraved as to make this a conscious effort, or is it simply apart of his strange impulses. That he believes his statements, and yet acts in his manner.
\par I myself, while usually exceptional at understanding others in the most fundamental sense. Understanding their innate insecurities, desires, and hopes. Occasionally fall into the trap of mistaking people 'quirks' for sadism.
\\ 
\par Earlier I had wondered if Fyodor represented an overt narcissist or an vulnerable narcissist. It now seems clear he is a vulnerable one. Searching for others to feel bad for him.
\\
\par "Do not be ashamed of yourself, for that is the root of it all"
\par This isn't some acclaim of self-pleasure and conceit, but rather stating how our own insecurities and shame(lack of pride) that lead to our vices.
\par I am not entirely sure what their insecurities are yet, given I am only a couple pages in, but they are already starting to materialize.

\\

\par The man who lies to himself, well in all reality that is most men to some extent. 
\par When men cannot find the truth, they lose respect. That respect, is their capability to love, because you cannot love without it. The lack of respect for themselves sinks them into their vices.
\par Such men, take offense. Because, taking offense is such a lovely thing to them, these is truly disgusting, but yet so common place. People do it all of the time, and this has many vices within it. 
\par First, it is a sign of the above. Actions are not taken in isolation and thus cannot be taken without the context of their action. Second, such thoughts are disgusting upon the face of it, truly disdainful upon such disgraceful means. Finally, this purposeful offense becomes resentment, resentment that effects them so deeply.
\\
\par Afterwards, it comes to a series of stories of the elder interacting with new people
 They all offer the ideas of Christianity as a service to protect humans from the constant degradation of themselves due to their environment. Showing how higher ideals protect us, and give us reason beyond material comforts.

\subsection{The Sailor Who Fell from Grace with the Sea}
\par The author of this book is such a strange man. A man born to samurai tradition, raised in a war, later raised as a girl, wished for war, disgusted by the peace around him, wrote audofictional stories, romanticized imperialism and fascism, was bisexual, and finally died by his own hand by disemboweling himself after faking a coup. To understand such a man will be an interesting project.
\\




%%%%%%%%%%%%%%%
\chapter{Art and Aesthetics}
\chapter{*Psychology}
\chapter{Culture \& Civilization}
"A nation's culture is the sum of the intellectual achievements of individual men, which their fellow-citizens have accepted in whole or in part, and which have influenced the nation's way of life. Since a culture is a complex battleground of different ideas and influences, to speak of a 'culture' is to speak only of the dominant ideas, always allowing for the existence of dissenters and exceptions." | Ayn Rand
\section{*Introduction}
\section{*Purpose of Civilization and Culture}
\section{*On the Abstract and the Average}
\section{*The Narration of Society}

\section{*AI and the Future}
\section{The Social Games we Play}
\par Given my intense individualism it can be forgiven to believe that I am completely against many 'social games.' This is partially because it is true to an extent. I will explain in a second.
\par Now like always, we must start with definitions. What do I mean by 'social games'? Well I mean unspoken rules, rituals, and maneuvers that govern human interaction. That dictate how we navigate relationships and such.
\par While it would seem I would have disdain for these games because of my individualism, logic driven mindset, and such. This is because I do, I do strongly dislike them, many of them are useless(in a logical sense), contradictory, and pathetic in my eyes. Though, they are extremely important. Their hallow-ness is exactly their importance. The fact that someone makes these concessions for the sake of others is an indicator empathy and respect. It shows them your intentions in a way, while they can be easily construed. Also, beyond that, as I have mentioned earlier of people's pathetic control over their own emotions, many people become angry or hurt by lack of following these rules. 
\par In essence, these games, while lacking true logic, are necessary for human communication. Not only do they simplify human interaction, but they also show things. They separate those that are adjusted to society and those that are not. Whether they by anti-social misfits, sadist, narcissist, idiots, arrogant kids who think they figured out morality, rebels, or any of the such.
\par That is the primary purpose of these social games, to show who people are.
\section{Moral Hedonism and Social Morality*}
\par This concept of moral hedonism(a term I have made up[I am sure an actual term exists but i haven found it yet so I will use this]) is something I have thought about a lot, because I see it all around me, here I will define it, explain it, examine its worth, merit and faults, and more.
\par First, I will define it. This concept of 'moral' hedonism is less of a philosophical and ethical framework but rather a defining trait of many people around you and I. 
\par Based upon psychological hedonism, that people generally act to maximize their own pleasure and minimize their own pain. This idea then expands it to moral actions. That through social training, people find pleasure(usually pride[different pride than I talk about] or some other happy feeling) when doing morally good things and pain(guilt and such) when doing immoral things. 
\par continue...
\section{Cleanliness, Evolution, and Disgust}
\par You know, something I have always wondered about history is how it took us so long to get basic hygiene figured out.
\par I mean I get being confused about brushing teeth, but not eating where you go to the bathroom. I always thought that the extreme visceral reaction to such things could only be explained by genetic and evolutionary marker.
\par Though something I have noticed, is that when camping or such, such manners of disgust go away. At a point cleanliness only becomes either a logical point of health that has been taught to us, or a rule enforced from above. 
\par It is striking how something that seems so natural to the human cause, is not only 'just a cultural thing' but is something that vanishes so quickly and easily.
\section{*Death and Rebirth of Culture in Postmodernity and Meta-modernity}
\section{*Individualism and Communitarianism}
\section{*Culture and System Theory}
\section{*An Examination of American Culture and Globalization}
\section{*Pop culture and Cultural Wars}
\section{*Modern Mythology}
\section{*Honor, Shame, and Guilt: Moral Frameworks Across Cultures}
\chapter{AI}
\section{*How AI Works}
\section{*Machine and Thought}
\section{*AI and Consiousness}
\section{*Ethics and AI}
\section{*AI and Expertise}
\chapter{Counterfactual and Debates}
"The greatest enemy to knowledge is not ignorance, it is the illusion of knowledge" | Stephan Hawking 
\section{Introduction}
\par Random arguments and such
\section{Emotional-Regulation and Morality}
\par A common argument in this day in age is on emotional regulation. Many in the modern day see that this regulation is simply hidden repression and should be ignored. I have many logical and utilitarian arguments against this, but I will take a bit more of an abstract approach. Simply put, my thesis is that it is impossible to live a truly moral life without it.
\par Without emotional-regulation, we are slaves to our impulses. Though some claim, in the idea of 'moral hedonism', that people can find pleasure in the good and pain in the bad. Though, this misses a crucial point, then people are simply slaves to tradition at best.
\par Basically, that this pleasure/pain is mostly just social conditioning(given that it is without emotional regulation). Just vapid following along with what they learned as children. Only through emotional-regulation can you control yourself to become a moral person. To give yourself pleasure/pain on based upon moral thought and pure intentions not vapid action without any real intention at all.
\par This is where guilt comes in, when you simply act in pleasure/pain but when your social teachings either go against other social teachings or your own bodily pleasure and personal temperaments. It also leads to a moral system created by petty emotions like resentment(which is easy to see in the world around you)
\par It is also easy for such a person to easily get confused between morality and pleasure in many case.
\\
\par Though, as mentioned earlier, this 'moral hedonism' isn't antithetical to moral life. So, this abstract argument isn't enough. A more utilitarian mode should be explored.


\chapter{Substack}
\section{Introduction}
\par I have recently started writing on a substack account. I plan to be writing about mathematical physics, with a bias towards Hamiltonian Mechanics 
\subsection{Axiomatic Derivation of Hamilton Mechanics }
\subsubsection{Introduction}
\par For any mathematical physical axiomatic derivation, a mathematical derivation is first required. I won't go super in depth but I will still go over it for rigors sake. I may go back and do a deeper dive but for now I will simply put these things.
\subsubsection{Axiom of Real numbers}
Why needed: Physical quantities like position, velocity, time, energy, and action are represented by real numbers. The calculus used in mechanics relies on the properties of real numbers.
\\
Key axioms:
\begin{itemize}
    \item Commutative, associative, and distributive properties for addition and multiplication.\
    \item Existence of additive and multiplicative identities (0 and 1).
    \item Existence of additive and multiplicative inverses (for non-zero numbers).
    \item Completeness axiom: Every non-empty set of real numbers with an upper bound has a least upper bound (ensures continuity, critical for calculus).
\end{itemize}



Role: These axioms enable arithmetic operations, inequalities, and the construction of functions like the Lagrangian $ L(q, \dot{q}, t) $.

\subsubsection{Axiom of Euclidean Geometry}
Why needed: If the system involves spatial coordinates (e.g., particles moving in 3D space), Euclidean geometry provides the framework for defining positions and distances.
\\
Key axioms(informal):
\begin{itemize}
    \item Points, lines, and planes exist.
    \item A straight line can be drawn between any two points.
    \item Distance between points is defined (e.g., via the Pythagorean theorem).

\end{itemize}

Role: Defines generalized coordinates $ q $ (e.g., Cartesian or polar coordinates) and kinetic energy terms like $ \frac{1}{2} m \dot{x}^2 $.
Note: For abstract systems (e.g., in generalized coordinates), geometry may be less critical, but it’s foundational for physical intuition.

\subsubsection{Axiom of Set Theory (Basic)}
Why needed: Sets are used to define the domain of variables (e.g., time $ t \in \mathbb{R} $, coordinates $ q \in \mathbb{R}^n $) and functions.
\\
Key axioms(informal):
\begin{itemize}
    \item Existence of sets: Sets can be formed to represent collections of objects (e.g., possible paths of a system).
    \item Union, intersection, and Cartesian product: Allow combining and manipulating sets.
    \item Axiom of choice (implicitly): Ensures a choice of path exists in variational problems.

\end{itemize}

Role: Provides the language for defining functions, spaces, and the configuration space of a system.

\subsubsection{Axiom of Calculus (Differential and Integral Calculus)}
Why needed: The principle of stationary action involves integrals (action $ S = \int L \, dt $) and derivatives (in Euler-Lagrange equations).
\\
Key concepts (built on axioms of real numbers):
\begin{itemize}
    \item Limits: Define continuity and differentiability of functions like $ L(q, \dot{q}, t) $.
    \item Derivatives: Partial derivatives (e.g., $ \frac{\partial L}{\partial q} $) are used to describe rates of change.
    \item Integrals: The Riemann integral defines the action $ S $.
    \item Fundamental theorem of calculus: Links derivatives and integrals, essential for variational calculus.

\end{itemize}

Role: Enables the formulation of the action and the variation $ \delta S = 0 $.

\subsubsection{Axiom of Set Theory (Basic)}
Why needed: Sets are used to define the domain of variables (e.g., time $ t \in \mathbb{R} $, coordinates $ q \in \mathbb{R}^n $) and functions.
\\
Key axioms(informal):
\begin{itemize}
    \item Existence of sets: Sets can be formed to represent collections of objects (e.g., possible paths of a system).
    \item Union, intersection, and Cartesian product: Allow combining and manipulating sets.
    \item Axiom of choice (implicitly): Ensures a choice of path exists in variational problems.

\end{itemize}

Role: Provides the language for defining functions, spaces, and the configuration space of a system.

\subsubsection{Axioms of Variational Calculus}
Why needed: The principle of stationary action requires finding the path that makes the action stationary, which is a problem in variational calculus.
\\
Key axioms(informal):
\begin{itemize}
    \item Functional: The action $ S $ is a functional (a function of functions, e.g., paths $ q(t) $).
    \item Variation: Small changes in the path $ \delta q(t) $ are used to compute $ \delta S $.
    \item Stationary condition: The path satisfies $ \delta S = 0 $, leading to the Euler-Lagrange equations.

\end{itemize}

Role: Provides the mathematical machinery to derive the equations of motion from the action.

\subsubsection{Axioms of Linear Algebra (Basic)}
Why needed: For systems with multiple coordinates or degrees of freedom, vectors and matrices describe generalized coordinates $ q_i $ and momenta $ p_i $.
\\
Key axioms:
\begin{itemize}
    \item Vector space axioms: Addition and scalar multiplication of vectors (e.g., coordinates in $ \mathbb{R}^n $).
    \item Linearity: Operations like partial derivatives in Hamilton’s equations are linear.


\end{itemize}
Role: Supports the formulation of the Hamiltonian $ H(q, p, t) $ and phase space (the space of $ (q, p) $).

\subsubsection{Axioms of Variational Calculus}
Why needed: The principle of stationary action requires finding the path that makes the action stationary, which is a problem in variational calculus.
\\
Key axioms(informal):
\begin{itemize}
    \item Functional: The action $ S $ is a functional (a function of functions, e.g., paths $ q(t) $).
    \item Variation: Small changes in the path $ \delta q(t) $ are used to compute $ \delta S $.
    \item Stationary condition: The path satisfies $ \delta S = 0 $, leading to the Euler-Lagrange equations.

\end{itemize}

Role: Provides the mathematical machinery to derive the equations of motion from the action.

\subsubsection{Physical Assumptions (Not Strictly Mathematical Axioms)}
While not mathematical axioms, certain physical assumptions are mathematically formalized:
\begin{itemize}
    \item Time is continuous and one-dimensional ($ t \in \mathbb{R} $).(Though Hamiltonains can work in other forms)
    \item Energy is well-defined: Kinetic energy $ T $ and potential energy $ V $ are functions of coordinates and velocities.
    \item Differentiability: The Lagrangian $ L $ is sufficiently smooth (at least twice differentiable) to allow partial derivatives
    \item 

\end{itemize}
Role: These ensure the mathematical framework applies to physical systems.
\subsection{Axiomatic Derivation of Hamiltonian Mechanics}
% Introducing Hamilton's contribution
Two hundred years ago, William Rowan Hamilton reformulated classical mechanics using the \textbf{principle of stationary action}, a single axiom that unifies the dynamics of physical systems. This principle states that the path taken by a system between two times makes the action \( S \) stationary.

% Defining the action and Lagrangian
The \textbf{action} is defined as:
\[
S = \int_{t_1}^{t_2} L(q, \dot{q}, t) \, dt,
\]
where \( L = T - V \) is the \textbf{Lagrangian}, with \( T \) as kinetic energy, \( V \) as potential energy, \( q \) as generalized coordinates, and \( \dot{q} \) as their time derivatives.

% Deriving the Euler-Lagrange equations
The system follows the path where \( \delta S = 0 \). Varying the action:
\[
\delta S = \int_{t_1}^{t_2} \left( \frac{\partial L}{\partial q} \delta q + \frac{\partial L}{\partial \dot{q}} \delta \dot{q} \right) dt = 0.
\]
Integrating by parts on the second term:
\[
\int_{t_1}^{t_2} \frac{\partial L}{\partial \dot{q}} \delta \dot{q} \, dt = \left. \frac{\partial L}{\partial \dot{q}} \delta q \right|_{t_1}^{t_2} - \int_{t_1}^{t_2} \frac{d}{dt} \left( \frac{\partial L}{\partial \dot{q}} \right) \delta q \, dt.
\]
Since \( \delta q = 0 \) at \( t_1, t_2 \), we get the \textbf{Euler-Lagrange equations}:
\[
\frac{d}{dt} \left( \frac{\partial L}{\partial \dot{q}} \right) - \frac{\partial L}{\partial q} = 0.
\]

% Introducing the Hamiltonian
Hamilton defined the \textbf{Hamiltonian} as:
\[
H(q, p, t) = \sum_i \dot{q}_i p_i - L(q, \dot{q}, t),
\]
where \( p_i = \frac{\partial L}{\partial \dot{q}_i} \) are generalized momenta. Typically, \( H = T + V \).

% Deriving Hamilton's equations
Using the Legendre transform, the dynamics are governed by \textbf{Hamilton's equations}:
\[
\dot{q}_i = \frac{\partial H}{\partial p_i}, \quad \dot{p}_i = -\frac{\partial H}{\partial q_i}.
\]

% Example application
\textbf{Example}: For a particle in gravity, \( L = \frac{1}{2} m \dot{h}^2 - mgh \). The momentum is \( p = \frac{\partial L}{\partial \dot{h}} = m \dot{h} \). The Hamiltonian is:
\[
H = \frac{p^2}{2m} + mgh.
\]
Hamilton's equations yield:
\[
\dot{h} = \frac{p}{m}, \quad \dot{p} = -mg,
\]
reproducing the equation of motion \( m \ddot{h} = -mg \).

% Significance
The principle of stationary action simplifies dynamics, applies to diverse systems, and underpins modern physics, including quantum mechanics and relativity. Hamilton's equations remain a corner stone of of modern physics.
\section{Causality}
Here I will dive into the nature and justification of our concept of causality. First through a linguistic stage, then physical, and finally mathematical. This will be moderately in-depth, but obviously further depth can be found and take all of this with a grain of salt. There are many interpretations of math and quantum mechanics and the one I present could very well be wrong.

So, let us dive in. What is causality, it is very simple. It is the concept that things happen because other things had happened first. That all things have a 'cause' and that that thus causes the effect (then they later become the cause of a later thing)

I can hear you, who in their right mind would question something as basic as cause and effect? Why even look into it. First, that is extremely intellectually lazy, all things have debates against it, you should always look into it. Though, more importantly, there actually are debates against it.

The easiest to dismiss is Hume's skepticism, that causality is simply a human made concept. That things don't really happen because of another action but simply that both actions are just sequences of events. Though, I would argue that this pedantic argument of definitions doesn't truly work. On an epistemological basis, causality is easier for humans to understand, and though breaking things into chunks we may better understand the world. That either concept is equally true, but one concept as a greater pragmatic 'truth' and you can argue that that correlates to a better empirical truth. That there is even a speed of causality written in the nature of reality. (The speed of light) A bit more deeply, once you quantize to the plank time, things truly do divide. Something happens, then a next thing happens. The complexity and abnormal reality of quantum mechanics makes this argument a little fuzzy and thus leads me to my next paragraph.

Now those unfamiliar with quantum mechanics may be confused. I will attempt to explain it to the best of my ability, but the reality of quantum mechanics is extremely complex and without a clear understanding of differential equations, discrete mathematics, linear algebra, and much more it is impossible to understand. I know many science communicators claim to teach "quantum physics and mechanics" to they layman, but this is a lie. These half backed analogies are not quantum mechanics. It does not come close, but luckily for me I am the only one reading this book I will be able to understand my attempt of analogizing(?) the philosophical question.

In essence, in quantum mechanics. Things are not local, for many reasons. The most basic is the idea that particles do not exist in the way we can imagine. They exist in multiple places at the same time[not really but useful analogy], simply with different probabilities (don't pretend to understand, nobody does, they just prove the math and accept as is.) You know what, that analogy was not enough. I will explain in more detail. Particles do not exist (not really). Simply quantum fields exist. The electron field for electrons, the photon for photons, and so on. The particles are not particles as we imagine, but excitations within the field. It is a bit more complex than that, they can act as particles when observed but that is a whole other can of worms. Now back to the matter at hand, now waves act in certain ways that is hard to explain. Due to their quantum nature, they must be decomposed first through Fourier modes to under stand more clearly. This is an example of a very simple form:

 
 
Now I won't explain the equation in its entirety, but I will say this. That you cannot solve it for an absolute certainty of position (not that you could get a "true" answer even if you could given that is is a Taylor Mode and not an analytical equation). Now you may think this is just a quirk of the math, but it is not. It is real for reason that could take up the entirety of this book if I truly understood them.

Next, quantum mechanics goes against causality through entanglement. Now what do I mean, when people refer to the speed of light, they are actually referring to the speed of causality. To explain this is relatively simple with a good understanding of Special Relativity. It is obvious that the universe is invariant to position and velocity(this has been previously proven thoroughly through experiments in high speeds, specifically through electromagnetism but others can apply), but for this to be true there must be a finite constant cosmic speed(because it requires a Lorenz transformation[there is more explained why this si the only answer but I will leave this hear because I have much much more physics to talk about in a philosophy book.]) This is the speed of causality, the speed of information. If something were to go above this speed it would in essence go back in time. This can be proven rather simply through simple algebra:

 
 
 
 
 
 
From here you can look at the square root and easily find that if v>c then gamma (the value of the first part with the

 
 
 
 
would be imaginary. Thus the t value(of time would then become negative, this obviously breaks causality.

Finally, back to quantum mechanics, as I am sure you know entanglement transfers information at faster than the speed. This seems to break causality.

Now I could keep on talking about arguments between quantum physics and causality, but I will only talk about one more. Virtual particles. Now first of all, what are virtual particles. Virtual particles are non-existent but existent particles that exist within the math of quantum field theory(and thus above my pay grade) but I will attempt to explain it. The idea in essence is that due to the uncertainty principle there exist small amounts of excitements within their fields. These excitements are not enough for an actual particle to exist in the way we imagine, but occasionally a particle can appear, primary through the use of "borrowing" energy from another particle or through creating a negative particle(not to be confused with anti-particles those are negatively charged, these have negative energy/mass) These virtual particles are especially important to particle interactions, what I mean is the fact that when two particles are "close" enough together they excite the fields around them to create bosons(force carriers) to cause the actual interaction. As an analogy, imagine two electrons, when close enough together they excite the fields around them to create a virtual photon, the photon then interacts with one electron(and also effects the other electron negatively[it is almost impossible to explain]) and cause the electromagnetic repulsion between the two. Now, back to philosophy, what in the world does this ramble have to do with causality. That is a great question. It is that all of our interactions are based upon something that isn't real. Electrons don't actually create virtual photons, these photons don't create negative photons. Despite having a tangible effect, they don't actually exist. How can something cause another thing if when you break it down to its component parts, particles are behaving 'irrationally' being effected by mathematical objects.

Now after I have listed the critiques, you are probably very aware about how you assumption of the obvious state of causality was intellectually lazy, but you probably also realize I wouldn't have written all of this unless I have an answer to these critiques, or at the very least a possible answer. In all reality I could be wrong about this and causality could be false. Though I am never wrong, so that must not be true.

The uncertainty of particles doesn't necessarily falsify causality, it simply evolves it. We previously see causality as something that is caused by a specific thing, but it can be caused by a probability of things(if that make sense).

Next I will talk about virtual particles, and yes I hear you saying that I should go in the order of which I introduced the ideas, but I do not care. Now, virtual particles. How does causality work with things that are not 'real'? Now the most astute among you might claim, why would causality not apply, and you are exactly right! You all may feel a sense of betrayal, that it was simply a trick of the light. But now, listen carefully, this is how philosophy works. Though asking questions that are hard to understand, questions that play tricks on our language, on our mind. Though these questions we refine our thoughts. This is the problem with modern philosophy, they confuse critiques with criticism. While criticism has a place within philosophy, it should not be as big as it is now. Now before I continue I will define why I consider a critique and a criticism as different things. A criticism takes a part an idea, concept, action, or really anything for the sake of taking it down. This can be justified if such ideas have no worth and the criticism is an attempt to prove it. A critique is a question used to refine a thought not destroy it.(though if the critique can not be resolved then it may.) This critique of virtual particles is not supposed destroy causality by any means, but like the earlier critiques, it is meant to refine our thoughts. It asks us to question what a cause is, what can be a cause. Can virtual particles be causes, modern science suggests yes, but this requires a refinement of our concept of causality. Of what is 'real.' In a later section I will delve into this in a future section, but now I will leave you to ponder.

Now finally, to entanglement. Does this refine our definition or break it? It breaks it, but due to the evidence surrounding causality, including the fact that the special and general relativity are simply logical derivatives of the idea of a finite special to causality suggests that our idea of entanglement is flawed not our idea of causality. Now what is the answer? In short, nobody knows. There are many theories trying to come up with a solution. These include Einstein-rosen bridge, super-determinism, and local hidden variables theory. Though since these ideas are both extremely complex and offer little beyond the problem at hand I have elected to move on.

Now, you may think we are done, but we aren’t. As I mention in other articles, I privilege Hamiltonian Mechanics due to its more fundamental grasp of reality, but many use Hamiltonian Mechanics to justify teleology.

I find that many of the arguments lack enough justification in my eyes. I mean, Hamilton himself had a clear belief in causality (while this obviously isn't enough, I will explain why he and I see it that way.)

First, the concepts of causal forces (forces are by definition causation principles) can be derived through the inverse gradient of potential energy. Showing that even in Hamiltonian mechanics, traditional forces and causality are still within its structure, just hidden.

Second, the connection of the 'principle of least' action and teleology is flimsy at best. This is because their main argument is based upon intuition of mathematic alone not physical realism. It hinges around the idea that this global optimization signifies some kind of “awareness” in optimization, but in all reality this optimization is simply a consequence of the deterministic properties of our world.

Third, some state that the time-symmetric nature of Hamiltonian mechanics can suggest that the future state constrains the present just as much as the past does, aligning with a teleological interpretation where the system’s evolution is influenced by its final state. Though, this is another example of looking too far. Time symmetry is simply a consequence of a lack of energy dissipation or sufficiently complex entropy. The introduction of such arises to asymmetry and some even say it breaks symmetry all together.

In Hamiltonian mechanics, a system’s dynamics are described in phase space, a high-dimensional space of coordinates q and momenta p. The system’s state evolves along a unique trajectory in phase space, determined by the Hamiltonian. This trajectory is a holistic entity, defined by both initial and final conditions in the variational formulation. Unlike Newtonian mechanics, which computes motion incrementally via local forces (e.g., F=ma), the Hamiltonian approach considers the entire path at once. This global perspective can be interpreted as teleological because it implies the system’s behavior is governed by a principle that accounts for its entire history, including future states, rather than a step-by-step causal process. Though, once again the phase space is a mathematical abstraction rather than metaphysical truth. You could do the same to Newtonian mechanics but that wouldn’t suggest Newtonian mechanics implies teleology. The phase space is simply equivalent to the local space in Newtonian mechanics.
\section{*Unified Thermodynamics Algorithm, Explanation}

\section{*Privileging Hamiltonian Mechanics}
\section{*Coding MHD Simulation}
\section{Applying Unified Thermodynamics Algorithm to ELM-MHD(Edge Localized}

%%%%%%%%%%%%%%%%%%%%%
\chapter{Random}
" $\lambda x.think(x) \implies x \in Domain \implies x  \ exists$... wait a second" - AJ
\section{Randomness in the highest extent}
\subsection{The trilogy of the insane, pt 1}
\par You know, I have bright ideas about this chapter, having it be used to formulate thoughts on things not quite suitable for a chapter(or at least not yet). Though, this first section will be a bit more esoteric (I know that isn't how you truly use it, but it runs nicely together and since I have no intentions of making this book in a manner that others can understand I will keep it, for my own amusement if nothing else) Now where was I, yes... This first section will be truly random, I will write and hopefully lead somewhere. Likely I will just continue on this loop of talking about my own idea of writing about this loop of for which I am writing that I am writing about... Well alase I will continue, trying to keep apace my mind in the trivialities of its own containment(what does that even mean, am I going crazy)
\par That is a great Segway, going from am I crazy to the fact that I always refer to myself as crazy, or controcept my actions by saying I am losing my mind.(I am not sure where the controcept word came from, so I looked it up... It does not exist, don't know where it came from, but to continue my pace on unabashed insanity I will attempt to move on without the intersection of more of my attention without true cause[further more this will be a hilarious read later on] 
\par Now back to my original premise(not original but second([?])). Why do I refer to myself as crazy or insane? What is it that is it about my thought, actions, and especially my innermost thoughts on the more abstract concepts as insane? Well, the most obvious answer is that it is slightly abnormal, I have a strange way of looking at things(which I should attempt to articulate how so later) and others think of that as abnormal and thus so do I.[another side note, I realize that it could be more related to when I think deeply within my own mind without external contact rather than just abstract notions.] Another obvious idea is that I pride myself in my own individually, both as irrational pride in oneself and as the fact that I value independent thought and thus those of insanity are most independent of all. Building off that, it is very common that teenagers get carried away in fantasies of their own individuality, this could easily be some manifestation of that common thread(which makes it all the funnier((that sounds really wrong when reading it))) Now in an attempt to engage my mind as I try to think of deeper and greater relations for say... I am going to walk around for a bit to thing(it is insane the amount of times I use pacing to think, like really... it is really abnormal)... I think another thing is internal manifestations of external humor. I tend to make others laugh by my antics and by the irony of the fact that I acknowledge the 'strangeness' and don't care. It is possible that such external attempts at humor can change my self-perception(less my actual self-perception but more of the language I use to describe myself. I obviously don't actually see myself as insane, and in all reality don't see myself as especially abnormal.)((gonna walk around a bit more)) While I can't think of additional reasons at the moment, if I were to take upon a gander, I would guess that the reason is all of the above(and probably a couple I haven't thought of)


\subsection{Physics is the only real science}
\par This is incredibly simple, earlier in this book I refereed to scientific realism vs anti-realism. Now if you take scientific realism to be true in the fullest and unabashed sense, fundamental physics is the only 'real' science. All other sciences are models. For instance, take the biology of healing. A cut heals, but that isn't what is actually happening, it is cells dividing, but that isn't what is happening... so on and so forth. It continues until quantum physics, when we find that we have no clue what is really going on and only a little more about how that effects what we observe.
\par SO, PHYSICS IS THE ONLY REAL SCIENCE!! *kinda*
\par This isn't my actual belief, I just wrote what I was thinking about.
