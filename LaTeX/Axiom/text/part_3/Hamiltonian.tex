\chapter{The Structure of Motion: A Philosophical Journey Through Hamiltonian Mechanics}
“The Hamiltonian formalism of mechanics, especially in its canonical form, has been an inexhaustible source of inspiration for modern theoretical physics.” | Albert Einstein
\section{Introduction}
\par This chapter is planned to be an extreme examination of Hamiltonian mechanics, especially from a philosophical, logic, and abstract way. Eventually, it will go from the basic logic and axioms underlying it, to the metaphysics and epistemology that are derived, and finally to advanced and modern usages. The primary purpose of this exercise is for me to find the gaps in my knowledge and gain a more intuitive and deep understanding of these concepts.(It was also written in 10th grade before the rest of the book so may not have the same standards.)
\par The reasoning for this is due to the extremely complex and abstract nature of such mechanics. To see the world through 'flows' instead of forces.
\subsection{Physical Reality and Frameworks}
\par Now, there are many theories, models, descriptions, framework, and so much more. Though, what really differentiates them? What makes one thing one and the other the other. While I won't truly go into extreme linguistic detail for every single one, I will go into moderate detail within the art to get the general idea down.
\par Let's focus on frameworks, because that is what a Hamiltonian is(others define it as schemes which is another great word and arguably more accurate, but I will refer to it as a framework because it has a more intuitive grasp). It is an entirely different framework than Newtonian. It comes from an entirely different perspective, looking into geometric identities rather than the causal relation of forces and such interactions. Also, other theories and identities are derived through it, independent of the actual reality of it. For instance, the flow of electromagnetic interactions, relativistic, or even gravitational are all derived from this concept of the flow of energies through sympathetic geometry. 
\par Even beyond that, Hamiltonian mechanics can represent any function that changes; statistics, viruses, and many others. This is due, to that Hamiltonian's is a pure mathematic relational concept rather than based upon physical axioms. For instance, Newtonians is based upon the axiom of inertia. Though, I will add one interesting involvement is the idea of privileging Hamiltonian over Lagrangian and stating that Lagrangian is a subset of Hamiltonian rather than its own framework.
\par Then this begins to ask our questions of scientific realism vs anti-realism(I found out the more neutral language is instrumentalism). Basically, what makes such theories more fundamentalist, predictive modeling or metaphysical truth. Further, this now engages the complexities of mathematical axiomatic reduction or physical.
\par Let's start with realism vs instrumentalism. I have already made my stance clear, having metaphysical truth in important but not in the way that non-metaphysical schemes should be thrown away. Though, this then brings about the question of what frameworks have greater metaphysical truth, but I will later look into this in later sections. 
\par This now leads to the question of mathematical abstraction vs physical realism. Though, one thing I will add is that the physical axioms of Newton's laws are based upon falsified ideas like absolute space and time(but you can get around such semantics by employing Whewell's axioms of Mechanics[1st: “Every change is produced by a cause.”
2nd says that: “Causes are measured by their effects.” Finally, the third remain unchanged from Newton's formulation.] Now back to the main event, I feel that the structuralism of Hamiltonian is better due to the fact that when you break down physics to the quantum the classical forces, inertia, and such break down and you can see that the structure is what remains. That while forces still interact as abstract 'causes of change" through bosons, their classical and 'physical' realism fails to encapsulate their extreme complexity.

\subsection{Philosophical Context*}
\par The Kantian influence on Sir Hamilton is clear....
\subsection{Notion of Time, States, and Trajectories}
\par Let's start with time. I have already explained my thoughts on time, but here I will deepen my explanation. Time, is the factor on which states evolve. It is then evolution of entropy. This axiom is very simple and intuitive. While objective empirical proving the nature of time is extremely challenging due to its abstract nature. For a more in-depth analysis of the nature of time visit Chapter 1, section 3\&7.
\par One thing I will discuss is time reversibility. Now, Hamiltonian mechanics works best for time reversal symmetries(it can still work in some use cases but rarely.). In fact, time reversal symmetry is a corner stone of much of modern physics, especially quantum mechanics, but this interferes with our intuitive grasp of reality, the second law of thermal dynamics, and even dark energy; in essence the emergent properties of reality seem to interfere with our 'fundamental mathematical' derivations. So I will examen these questions; both for their own state and because it continues our question of the fundamental nature of reality. I will also attempt to discuss imaginary time with regards to quantum mechanics.
\par Actually, upon further thought the nature of time will be further developed through after the derivation of analytical mechanics.
\par Next will the on trajectories; now like many other ideas presented within this book, that may seem simple but the derivation of the ideas will be thorough and show its true complexity. Within Hamiltonian mechanics, the trajectories hold special geometric identities due to the nature of motion, but what can we derive from this simple observations

\subsection{Introduction to Phase Space}
\par Another question that arises from a Hamiltonian view of metaphysics is Phase Space. The fact that, according to our theories, phase space almost seems real, but yet it is so very different than what we observer. Or, we think so.
\par Obviously, there is the idea of the conflict between the teleology that comes as a consequence and our view of causality. I will address this in a second.
\par A simpler idea is that, from a perspective view, change becomes fundamental and any identity becomes simply emergent. This isn't truly new, simply more emergent in phase space, so I will move on.
\par One new thing, is that Phase Space allows for entropy irreversibility. Traditional theories find time reversibility to be a corner stone, this going against it.
\par This is due to the fact that entropy increases
$$S=-k_B\Sigma_i p_i logp_i$$
Where the volume of the phase space stays the same
$$Vol(\phi_t(\Omega_M))=Vol(\Omega_M)$$

\par Lastly, the 6D phase space seems very different to ours. Having generalized position and momentum coordinates is odd. An instrumentalist view makes this simple, but a scientific realism makes this difficult. Sadly, I don't have the knowledge to come up with.
\subsection{Teleology vs Causality}
\par My privileging of causal relations has already become clear, but now I will justify my thoughts through a Hamiltonian view point. While many claim that the nature of Hamiltonian mechanics necessitates a teleology, and my stated view on structuralism seems to contradict my views on causality, but I will address these claims categorically.
\par First, the nature of Hamiltonian mechanics necessitates a teleological view of the natural world. This statement will seemingly logical, lacks the tautological strength required of such an extreme statement. For one, Hamilton himself had a clear belief in causality(while this obviously isn't enough, I will explain why he and I see it that way.) First, the concepts of causal forces(forces are by definition causation principles) can be derived through the inverse gradient of potential energy. Second, the connection of the 'principle of least' action and teleology is flimsy at best. This is because their main argument is based upon intuition of mathematic alone not physical realism(I know this goes against stated structuralism but I will come back to that), the reason I state this against due to the fact that, mathematical intuition in the face of metaphysical truth. For instance, as stated earlier the teleological 'energy' is found as a derivation of causal force. Finally, returning to our arguments placed in the first chapter.
\subsection{Role of Variational Principles in Mechanics}
\section{Axiomatic Mechanics}
\subsection{Axiomatic Derivation of ... }
\subsubsection{Introduction}
\par For any mathematical physical axiomatic derivation, a mathematical derivation is first required. I won't go super in depth but I will still go over it for rigors sake. I may go back and do a deeper dive but for now I will simply put these things.
\subsubsection{Axiom of Real numbers}
Why needed: Physical quantities like position, velocity, time, energy, and action are represented by real numbers. The calculus used in mechanics relies on the properties of real numbers.
\\
Key axioms:
\begin{itemize}
    \item Commutative, associative, and distributive properties for addition and multiplication. 
    \item Existence of additive and multiplicative identities (0 and 1).
    \item Existence of additive and multiplicative inverses (for non-zero numbers).
    \item Completeness axiom: Every non-empty set of real numbers with an upper bound has a least upper bound (ensures continuity, critical for calculus).
\end{itemize}



Role: These axioms enable arithmetic operations, inequalities, and the construction of functions like the Lagrangian $ L(q, \dot{q}, t) $.

\subsubsection{Axiom of Euclidean Geometry}
Why needed: If the system involves spatial coordinates (e.g., particles moving in 3D space), Euclidean geometry provides the framework for defining positions and distances.
\\
Key axioms(informal):
\begin{itemize}
    \item Points, lines, and planes exist.
    \item A straight line can be drawn between any two points.
    \item Distance between points is defined (e.g., via the Pythagorean theorem).
\end{itemize}

Role: Defines generalized coordinates $ q $ (e.g., Cartesian or polar coordinates) and kinetic energy terms like $ \frac{1}{2} m \dot{x}^2 $.
Note: For abstract systems (e.g., in generalized coordinates), geometry may be less critical, but it’s foundational for physical intuition.

\subsubsection{Axiom of Set Theory (Basic)}
Why needed: Sets are used to define the domain of variables (e.g., time $ t \in \mathbb{R} $, coordinates $ q \in \mathbb{R}^n $) and functions.
\\
Key axioms(informal):
\begin{itemize}
    \item Existence of sets: Sets can be formed to represent collections of objects (e.g., possible paths of a system).
    \item Union, intersection, and Cartesian product: Allow combining and manipulating sets.
    \item Axiom of choice (implicitly): Ensures a choice of path exists in variational problems.

\end{itemize}

Role: Provides the language for defining functions, spaces, and the configuration space of a system.

\subsubsection{Axiom of Calculus (Differential and Integral Calculus)}
Why needed: The principle of stationary action involves integrals (action $ S = \int L \, dt $) and derivatives (in Euler-Lagrange equations).
\\
Key concepts (built on axioms of real numbers):
\begin{itemize}
    \item Limits: Define continuity and differentiability of functions like $ L(q, \dot{q}, t) $.
    \item Derivatives: Partial derivatives (e.g., $ \frac{\partial L}{\partial q} $) are used to describe rates of change.
    \item Integrals: The Riemann integral defines the action $ S $.
    \item Fundamental theorem of calculus: Links derivatives and integrals, essential for variational calculus.

\end{itemize}

Role: Enables the formulation of the action and the variation $ \delta S = 0 $.
\subsubsection{Axioms of Variational Calculus}
Why needed: The principle of stationary action requires finding the path that makes the action stationary, which is a problem in variational calculus.
\\
Key axioms(informal):
\begin{itemize}
    \item Functional: The action $ S $ is a functional (a function of functions, e.g., paths $ q(t) $).
    \item Variation: Small changes in the path $ \delta q(t) $ are used to compute $ \delta S $.
    \item Stationary condition: The path satisfies $ \delta S = 0 $, leading to the Euler-Lagrange equations.

\end{itemize}

Role: Provides the mathematical machinery to derive the equations of motion from the action.

\subsubsection{Axioms of Linear Algebra (Basic)}
Why needed: For systems with multiple coordinates or degrees of freedom, vectors and matrices describe generalized coordinates $ q_i $ and momenta $ p_i $.
\\
Key axioms:
\begin{itemize}
    \item Vector space axioms: Addition and scalar multiplication of vectors (e.g., coordinates in $ \mathbb{R}^n $).
    \item Linearity: Operations like partial derivatives in Hamilton’s equations are linear.


\end{itemize}
Role: Supports the formulation of the Hamiltonian $ H(q, p, t) $ and phase space (the space of $ (q, p) $).

\subsubsection{Axioms of Variational Calculus}
Why needed: The principle of stationary action requires finding the path that makes the action stationary, which is a problem in variational calculus.
\\
Key axioms(informal):
\begin{itemize}
    \item Functional: The action $ S $ is a functional (a function of functions, e.g., paths $ q(t) $).
    \item Variation: Small changes in the path $ \delta q(t) $ are used to compute $ \delta S $.
    \item Stationary condition: The path satisfies $ \delta S = 0 $, leading to the Euler-Lagrange equations.

\end{itemize}

Role: Provides the mathematical machinery to derive the equations of motion from the action.

\subsubsection{Physical Assumptions (Not Strictly Mathematical Axioms)}
While not mathematical axioms, certain physical assumptions are mathematically formalized:
\begin{itemize}
    \item Time is continuous and one-dimensional ($ t \in \mathbb{R} $).(Though Hamiltonains can work in other forms)
    \item Energy is well-defined: Kinetic energy $ T $ and potential energy $ V $ are functions of coordinates and velocities.
    \item Differentiability: The Lagrangian $ L $ is sufficiently smooth (at least twice differentiable) to allow partial derivatives
    \item 

\end{itemize}
Role: These ensure the mathematical framework applies to physical systems.
\subsection{Axiomatic Derivation of Hamiltonian Mechanics}
% Introducing Hamilton's contribution
Two hundred years ago, William Rowan Hamilton reformulated classical mechanics using the \textbf{principle of stationary action}, a single axiom that unifies the dynamics of physical systems. This principle states that the path taken by a system between two times makes the action \( S \) stationary.

% Defining the action and Lagrangian
The \textbf{action} is defined as:
\[
S = \int_{t_1}^{t_2} L(q, \dot{q}, t) \, dt,
\]
where \( L = T - V \) is the \textbf{Lagrangian}, with \( T \) as kinetic energy, \( V \) as potential energy, \( q \) as generalized coordinates, and \( \dot{q} \) as their time derivatives.

% Deriving the Euler-Lagrange equations
The system follows the path where \( \delta S = 0 \). Varying the action:
\[
\delta S = \int_{t_1}^{t_2} \left( \frac{\partial L}{\partial q} \delta q + \frac{\partial L}{\partial \dot{q}} \delta \dot{q} \right) dt = 0.
\]
Integrating by parts on the second term:
\[
\int_{t_1}^{t_2} \frac{\partial L}{\partial \dot{q}} \delta \dot{q} \, dt = \left. \frac{\partial L}{\partial \dot{q}} \delta q \right|_{t_1}^{t_2} - \int_{t_1}^{t_2} \frac{d}{dt} \left( \frac{\partial L}{\partial \dot{q}} \right) \delta q \, dt.
\]
Since \( \delta q = 0 \) at \( t_1, t_2 \), we get the \textbf{Euler-Lagrange equations}:
\[
\frac{d}{dt} \left( \frac{\partial L}{\partial \dot{q}} \right) - \frac{\partial L}{\partial q} = 0.
\]

% Introducing the Hamiltonian
Hamilton defined the \textbf{Hamiltonian} as:
\[
H(q, p, t) = \sum_i \dot{q}_i p_i - L(q, \dot{q}, t),
\]
where \( p_i = \frac{\partial L}{\partial \dot{q}_i} \) are generalized momenta. Typically, \( H = T + V \).

% Deriving Hamilton's equations
Using the Legendre transform, the dynamics are governed by \textbf{Hamilton's equations}:
\[
\dot{q}_i = \frac{\partial H}{\partial p_i}, \quad \dot{p}_i = -\frac{\partial H}{\partial q_i}.
\]

% Example application
\textbf{Example}: For a particle in gravity, \( L = \frac{1}{2} m \dot{h}^2 - mgh \). The momentum is \( p = \frac{\partial L}{\partial \dot{h}} = m \dot{h} \). The Hamiltonian is:
\[
H = \frac{p^2}{2m} + mgh.
\]
Hamilton's equations yield:
\[
\dot{h} = \frac{p}{m}, \quad \dot{p} = -mg,
\]
reproducing the equation of motion \( m \ddot{h} = -mg \).

% Significance
The principle of stationary action simplifies dynamics, applies to diverse systems, and underpins modern physics, including quantum mechanics and relativity. Hamilton's work remains a cornerstone of theoretical physics.
\subsection{*Hamiltonian Mechanics ad a Geometric Theory}
\subsection{*Poisson Brackets and the Algebra of Dynamics}
\subsection{*Canonical Transformations: Symmetry, Simplicity, an Structure}

\section{Philosophical Implications*}
\subsection{*Time in Essence}
\subsection{*Hamiltonian Constraints and the Problem with Time}

\section{Modern Hamiltonian}
\subsection{Modern Research}
\subsubsection{*Hamiltonian Formulation in Plasma Physics}
\subsubsection{*Hamiltonian and Metriplectic Mechanics}
\subsubsection{*Symplectic Integrators: Preserving the Structure of Nature}
\subsubsection{*HNNN(Hamiltonian Neural Networks)}
\subsection{Fun Mess Around}
\subsubsection{Non-Canonical Poisson Brackets}
\par A Poisson bracket is a bilinear, antisymmetric operation $[F, G]$ that defines the time evolution of a functional $F$ via $\dot{F} = [F, H]$, where $H$ is the Hamiltonian.  Non-canonical brackets arise when the phase space variables (e.g., density, velocity, magnetic field in MHD) do not follow the standard canonical structure $\{q_i, p_j\} = \delta_{ij}$. 
\\
Properties: The bracket must satisfy:
\\
\begin{itemize}


    \item Antisymmetry: $[F, G] = -[G, F]$
    \item Leibniz Rule: $[F, GH] = [F, G]H + G[F, H]$
    \item Jacobi Identity: $[F, [G, H]] + [G, [H, F]] + [H, [F, G]] = 0$
\end{itemize}
\par For this I will be defining my own Poisson bracket specifically for plasma thrusters.
\par For this I will obviously need to define the variables. Things like:
\begin{itemize}
    \item Mass density: $\rho(\mathbf{r}, t)$
    \item Velocity field: $\mathbf{v}(\mathbf{r}, t)$
    \item Magnetic field: $\mathbf{B}(\mathbf{r}, t)$
    \item Entropy or internal energy: $s(\mathbf{r}, t)$ or $\epsilon(\rho, s)$
\end{itemize}
\par Now define the Hamiltonian, for this I will use the pre-established ideal-MHD, but in all reality for plasma thrusters I should use a more expanded model.
$$H[\rho, \mathbf{v}, \mathbf{B}, s] = \int \left( \frac{1}{2} \rho v^2 + \rho \epsilon(\rho, s) + \frac{B^2}{2\mu_0} \right) d^3x$$
\par Next, I must define the phase spaces and constraints. Incompressibility or magnetic field divergence. 
\par Next I propose a bracket system, with many of them being 
$$[F, G] = \int \sum_{i,j} \frac{\delta F}{\delta \xi_i} J_{ij} \frac{\delta G}{\delta \xi_j} d^3x$$
\par Such that Ideal MHD is:
$$[F, G] = -\int \Bigg\{
\rho \left[ \frac{\delta F}{\delta \rho}, \frac{\delta G}{\delta \mathbf{v}} \right]
+ \left[ \frac{\delta F}{\delta \mathbf{v}}, \frac{\delta G}{\delta \mathbf{v}} \right] \cdot \left( \frac{\mathbf{B}}{\rho} \times \nabla \times \frac{\delta G}{\delta \mathbf{B}} \right)
+ \frac{1}{\rho} \left( \nabla \times \frac{\delta F}{\delta \mathbf{B}} \right) \cdot \left( \frac{\delta G}{\delta \mathbf{v}} \times \mathbf{B} \right)
+ s \left[ \frac{\delta F}{\delta s}, \frac{\delta G}{\delta \mathbf{v}} \right]
\Bigg\} d^3x$$
\par Another approach is using lie-groups. The Lie-Poisson approach is a method to construct non-canonical Poisson brackets by reducing a canonical Hamiltonian system on a large phase space (e.g., particle coordinates) to a smaller phase space of collective variables (e.g., fluid fields in MHD). It is rooted in the symmetry properties of the system’s configuration space, described by a Lie group.
\par ...
\par The other way is through Casmir invariants. Casimir invariants are functionals $C$ that commute with all functionals $F$ under the Poisson bracket: $[C, F] = 0$. They are conserved quantities that arise from the degeneracy of the non-canonical bracket and provide constraints on the system’s dynamics.
\par Degeneracy: Non-canonical Poisson brackets are degenerate, meaning their Poisson tensor $J_{ij}$ has a non-trivial kernel. Casimirs live in this kernel, satisfying:
$$J_{ij} \frac{\delta C}{\delta \xi_j} = 0$$

\par Physical Role: Casimirs represent invariants tied to the system’s topology or symmetries, such as helicity in MHD, which constrain the evolution of plasma in thrusters.
\par Identification: Casimirs are found by solving the functional equation $[C, F] = 0$ for all $F$, often using the Lie algebra’s cohomology.
