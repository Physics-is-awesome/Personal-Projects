% Auto-generated Daily Notes

\section{2025-11-27}

\section{Emacs}
\label{sec:orgcd41949}

This is my first journal write using EMACS, for that has been my project at the moment. Creating a doom Emacs with Org-mode enhancements, AUCTeX/pdf tools. Org-roam(graph+server), Org-bable for multiple languages(FORTRAN, python, Julia, Jupyter, and shell), LSP for coding, Magit(for git), projectile/treemacs(for using projects), a daily journal that auto inserts into my \LaTeX{} book, and create Org nodes and graphs for chapters and journals.

For the most part I have this set up. I need to edit AUCTeX so that it uses latexmc rather than latex, so that it will run multiple times for the book-like structure. I think I have more dependencies I need for the languages. I still have more to learn to get comfortable with this program. Also, I am not sure that the program auto inserts my journals.

\begin{verbatim}

print("2+2 =", str(2+2))

\end{verbatim}


\section{2025-11-29}

\section{Axiom}
\label{sec:org8b68e15}
As is very obvious, I am making a transfer from now on. I will have this section for day-to-day journaling and notes, while the previous sections for more concise and developed thought. I won't change the already made stuff. This new
This takes me to Emacs, I have figured out how to incorporate Org files into my \LaTeX{} book.

I do need to work on attaching it to the table of context. I believe I should have this completed by the end of the day.
\section{Emacs}
\label{sec:orgba87eba}
I am learning much about how to use Emacs, as I have mentioned above I have been working to better understand and use Emacs. I have created two new function's. One to make IDs for Org files written outside of Emacs and also to turn Org files into \LaTeX{} to use.
\section{Media}
\label{sec:orgcdd1279}
\subsection{Zettelkasten}
\label{sec:orge48842b}
I was originally having this section be a Zettelkasten-like notes taking. Though after more consideration, I have reconsidered. While a similar linking will be useful, I am unlikely going to have the same extreme modularity that is accustomed to the system. Thus, I am unlikely to use it fully, but some use is likely.


\section{2025-12-03}

\section{Links to Book Chapters}
\label{sec:org49b9d29}
\begin{itemize}
\item \href{../text/org/chapters/coding.org}{coding} \href{../text/org/chapters/Axiom.org}{Axiom} \url{roam:Omega-X} \url{roam:plans} \url{roam:personal\_analysis}
\end{itemize}
\section{Emacs}
\label{sec:orgf286cbb}
\subsection{Emacs updates}
\label{sec:org07ff349}
For this journal, remember that the command is SPC n r d t. Beyond that I have fully completed the Org \(\rightarrow\) \LaTeX{} file. Though, as useful as these programs can be it does raise the question of if I should use it. I have begun to question its usefulness.
\subsection{Emacs vs Vim}
\label{sec:org70adfab}
I have found that as I have created these things, I have realized I can substitute a lot of the programs by simple scripting and vim. While Vim's speed and ease of use has its use. I do think the scripting of Emacs it also quite fast, ease of moving between files, and even the spell check. Beyond that the integration of things coding languages is potentially very useful once I gain the hang of it.
\subsection{Org vs Tex}
\label{sec:org88716d1}
One question I have been looking into is whether I should write my notes directly in \LaTeX{} or continue to write in an Org file and export to \LaTeX{} after the fact.
\textbf{Org}
\begin{center}
\begin{tabular}{1|1}
Pros & Cons\\
Org babel to run commands & Easy errors\\
Org-capture to quickly make notes & whitespace inconsistency\\
Graph view & Hard to write math with\\
Easy connections & Difficulty with more complex \LaTeX{} commands\\
Easy to use syntax & Degregation of performace over time\\
Easy to use Hierarchy & \\
\end{tabular}
\end{center}
\textbf{Tex}
\begin{center}
\begin{tabular}{1|1}
Pros & Cons\\
Absolute control & more verbose\\
none of the easy errors & Have to script new stuff\\
Can be scripted to simulate Org-capture-like & No babel\\
More freedom of use in \LaTeX{} commands & \\
\end{tabular}
\end{center}
\subsection{Future}
\label{sec:org9454767}
Currently my plan is to have continue to use Org for journaling, though I may revisit this idea later on.

With this I will continue to learn how to use Org/\LaTeX{} integration and use. Also some of syntax of Org. I also need to learn more about running code and using it in Emacs. I currently only use it for notes.
\begin{itemize}
\item note-book enviorment(like jupiter)
\item build-up the graph view
\item Agenda
\item linking
\item task automation
\end{itemize}
\section{Omega-X}
\label{sec:org35ee710}
\subsection{Arguments}
\label{sec:orgf4c335b}
Currently, I have the poor practice of setting variables in a module and ignoring using arguments or such things. I plan to rectify this. This is for several reasons; to protect from errors and to allow for MPI(my next step).
\subsection{Updates}
\label{sec:orgef8735e}
I need to continue on a couple of sets. Create a IO files, finish the evolution driver, and have ways to analyze the data.
\section{Future Ideas}
\label{sec:orga3bbd67}
I have a couple ideas for future projects:
\subsection{Categorical Metriplectic}
\label{sec:orge23f869}
Using category theory to set up a fully and complete formal theory of the Metriplectic system. This is highly ambitious considering my limited knowledge in category theory, but it is a field that is thoroughly explored and thus I believe it is within my reach to accomplish.
\subsection{Cason Brackets Dynamics(name pending)}
\label{sec:org7bc73bd}
I believe I can integrate metriplectic systems to Nambu mechanics. As far as I know this is a completely novel goal and thus would be my own.
Let's see with the names: Cason Formalism, Cason Mechanics, Cason dynamics, Cason framework, Cason Analysis, Casonian Dynamics, Casonian Mechanics, Cason's Method, Cason's Dynamics, Casonic Dynamics, The Cason Principle, Cason Bracket Dynamics, Cason system, Cason Framework for hybrid Dynamics. I know this seems overtly narcissistic, trying hard to come up with a name based upon my own; I don't really care.
\subsection{Metriplectic PINN}
\label{sec:orga999662}
I have read several papers using Metriplectic dynamics to create PINN(Physics-informed-neural-networks).
\section{Media}
\label{sec:orge36e752}
\subsection{Star Trek}
\label{sec:orgb991892}
I forgot how much I like Star Trek next Gen, especially Picard.
\section{Self-analysis}
\label{sec:orgfe507cb}
\subsection{Stress $\backslash$& TV}
\label{sec:orgea2dd21}
For some reason, watching TV where the show lets you in on how stupid an idea is and yet the character does it stresses me out. Especially with lying and illicit activity. When they lie and it spirals out, then they lie more. I don't like it. Messes with me.


\section{2025-12-04}

\section{Links to books}
\label{sec:org1b5b9dc}
\begin{itemize}
\item \href{../text/org/chapters/plans.org}{plans}
\end{itemize}
\section{Projects}
\label{sec:orga8416c8}
I have recently come into more free time and would like to use it to my advantage to increase my abilities and capacity; though, I am having trouble considering what to do, in what order, and how to block my time.
\subsection{Formal projects}
\label{sec:org524eee1}
The actual projects I want to work on are continuing:
\begin{itemize}
\item FORTRAN Omega-X project([\href{../text/org/chapters/Omega-X.org}{Omega-X}]finish it, develop parallel computing, develop TUI, continue documentation, profile it, add higher order methods, more time stepping methods, add modules for working with additional systems, data analysis, visualization, GPU acceleration, etc)
\item Cason Bracket dynamics: synthesis Nambu and Metriplectic and incorporate into Omega-X
\item Categoric Theory of Metriplectic Dynamics: Self explanatory
\item Metriplectic PINN: PINN based upon PINNs
\item Emacs Building: I want to build workflows and other things in Emacs to make stuff easier for me(coding workflow, see data, writing book stuff)
\end{itemize}
\subsection{Skills}
\label{sec:org34a5b22}
There are several skills I want to continue to develop:
\begin{itemize}
\item Julia coding: right now in the basics
\item C++: similar to Julia
\item Emacs: I want to get more comfortable with the key-bindings and creating commands to make things simple
\item Parallel computing: most clearly can gain this through my Omega-X project
\item Org/\LaTeX{}: Learing how to use both org and tex so they work together.
\end{itemize}
\subsection{Plan}
\label{sec:org6eac500}
I am thinking schedule it:
\begin{enumerate}
\item develop a plan for the day in Org, experiment with Org/\TeX{} integration to gain that experience in that regard.
\begin{itemize}
\item Incorporate Julia whenever possible
\end{itemize}
\item Theoretic study - work on one a day
\begin{itemize}
\item Categoric Theory of Metriplectic Dynamics
\item Cason Bracket Dynamics
\end{itemize}
\item Programming Project - work on one maybe two a day
\begin{itemize}
\item Develop Omega-X project
\begin{itemize}
\item finish it
\item develop parallel computing
\item develop TUI
\item continue documentation
\item profile it
\item add higher order methods
\item more time stepping methods
\item add modules for working with additional systems
\item data analysis, visualization
\item GPU acceleration
\end{itemize}
\item Develop Emacs tools
\begin{itemize}
\item Create Julia REPL in Emacs and keybinding to make it simple
\item custom \LaTeX{} keys
\item Set FORTRAN build and run
\end{itemize}
\end{itemize}
\item Skill Development - work on one or two a day
\begin{itemize}
\item Julia
\begin{itemize}
\item symbolic Poisson/Nambu brackets solver
\item ODE solver using the SciML interface
\item GPU-accelerated metriplectic evolution
\item Category theory for dynamics
\end{itemize}
\item C++
\begin{itemize}
\item Simple library
\item SIMD-accelerated
\item learn expression templates
\item GPU version using CUDA or HIP
\item Plug-in architecture
\item simple MHD solver
\end{itemize}
\item FORTRAN
\begin{itemize}
\item Coarrays
\item Polymorphic OOP with type-bound procedures
\item Generics (Fortran 2023)
\item MPI
\item GPU Offloading
\item Memory alignment + SIMD-aware design
\end{itemize}
\end{itemize}
\end{enumerate}


\section{2025-12-05}

\section{Links to Book Chapters}
\label{sec:orgc785761}
\begin{itemize}
\item 
\end{itemize}
\section{Plans for Today:}
\label{sec:orgd8c8ae1}
I have been currently working on my school work; getting caught up with that. I have also conntacted NSCC about the dual enrollment program to make sure I am not missing anything.

For my next steps:
\begin{itemize}
\item Programming Project: Omega-X
\begin{itemize}
\item First: I am going to build a type binding for Omega-X's ininitilize subdriver.
\item Second: formalize declare\textsubscript{vars} module to work for this new update
\item Third: Move C++ binding to a safe place(C++/FORTRAN?[either fits])
\end{itemize}
\item Theoretic study: Catogoric Metriplectic:
\begin{itemize}
\item Create \LaTeX{} notes sections
\item Write list of possible papers to read
\item Read one and write notes
\end{itemize}
\item Programming/skill: EMACS building/Julia
\begin{itemize}
\item Develop Julia REPL in EMACS
\item Play around with it once in
\end{itemize}
\end{itemize}
\subsection{Omega-X Notes}
\label{sec:orgc41761f}
As I mentioned in the above section my first task is to develop a type system for driver\textsubscript{init.F90}. To create the types:
\begin{verbatim}
module types_make
     !individual types
      type :: name_of_type_1
        real :: variable_in_type_1
      end type name_of_type_1

      type :: name_of_type_2
        real :: variable_in_type_2
      end type name_of_type_2

      ! higher level type
      type :: higher_level_type
        type(name_of_type_1)
        type(name_of_type_2)
      end type higher_level_type
end module types_make
\end{verbatim}
The to use it:
\begin{verbatim}
program use_types
        use types_make

        type(higher_level_type) :: H_T_L

        ! initialize fields
        H_T_L_of_type_2%variable_in_type_2 = 1.0
\end{verbatim}


I ended up doing less than I imagined. I created the type system and corrected the arguments for many things. My next steps should be to develop the
\begin{enumerate}
\item declare\textsubscript{vars}
\item fix uses of modules
\item Change python code to work with new type arguments.
\end{enumerate}

This is what I can do tommorow.
\subsection{Emacs Build}
\label{sec:orgdec65de}
While working on Omega-X I figured I should focus on FORTRAN integration before Julia, I also saw a cool calculator thing I will set up and shouldn't take long.

Calculator:
\begin{itemize}
\item Get in M-x calc or C-x * or M-x full-calc
\item It is in reverse polish notation: operators follow their operands
\item t to get to trail
\item i to enter
\item u to undo
\item DD to redo
\item TAD to swithc
\item DEL to delete one
\end{itemize}
There is also a \LaTeX{} embeded system:
\begin{itemize}
\item C-x * e → Enter calc-embedded mode.
\item C-x * r → Replace the marked expression with its result.
\item C-x * q → Quick evaluation of inline expressions.
\end{itemize}

calc-embedded	C-x * e	Enter embedded mode on the current region (or insert delimiters if none).
calc-embedded-word	C-x * w         Evaluate the word at point as a Calc expression.
calc-embedded-activate	C-x * a	Activate embedded formulas in the current buffer.
calc-embedded-new-formula	C-x * n	Insert a new embedded formula at point.
calc-embedded-select	C-x * s	Select the next embedded formula in the buffer.
calc-embedded-edit	C-x * e (when inside formula)	Edit the embedded formula in Calc.
calc-embedded-update	C-x * r	Replace/update the formula with its evaluated result.
calc-embedded-update-all	C-x * u	Update all embedded formulas in the buffer.
calc-embedded-clear	C-x * c	Clear (remove) the result part of an embedded formula.
calc-embedded-clear-all	C-x * C	Clear results for all formulas in the buffer.
calc-embedded-duplicate	C-x * d	Duplicate the current formula.
calc-embedded-next	C-x * n	Jump to the next embedded formula.
calc-embedded-previous	C-x * p	Jump to the previous embedded formula.
calc-embedded-reload	C-x * l	Reload Calc settings for embedded mode.

Now I have created the FORTRAN stuff:
\begin{enumerate}
\item Fortran-lang LSP
\item Fortran-indent improvements and compile.el
\item custom build commands
\item Org babel
\end{enumerate}
\subsection{Categoric Metreplectic}
\label{sec:org359be43}


