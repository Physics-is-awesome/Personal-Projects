\chapter{Logic}
% Insert quote here
% Extra line
\singlespacing
\section{Introduction}
Logic is an important tool for the creation of all formal systems. It is the foundation of mathematics, which is for the most part the foundation of everything else. It is also simply important to anything else by clarify concepts, giving them formal definitions, and insures consistency. Allowing for a bedrock behind all analytical thinking. 

Though, before I can get into it too much, I should define what the study of logic is. It is the branch of philosophy that critically examines the fundamental nature, scope, and principles of logic itself.

It has a couple fo subfields as well. 
\begin{itemize}
    \item Philosophy of Logic
    \begin{itemize}
        \item The study of logic at a high level, analyzing its nature and scope.
    \end{itemize}
    \item Formal Logic
    \begin{itemize}
        \item A creation and usage of symbolic logic symbols to create systems; mainly mathematics.
    \end{itemize}
    \item Metalogic
    \begin{itemize}
        \item The technical, mathematical study of the properties of formal logical systems. The philosophy of logic often uses the results of metalogic (like Gödel's incompleteness theorems) to address its conceptual questions, such as the limits of a formal axiomatic system
    \end{itemize}
\end{itemize}
\section{Defining an Axiom}
This entire text is named Axiom, for it is the foundation. To define systems axiomatically and then build them out into their true complexity is not only one of my favorite things to do, but as an extension of that it is what this book is trying to do.(Obviously, I wouldn't do something like this if I didn't enjoy it.)

So let's define it. As I mention extensively in my last book, Lógos, we must define things extremely rigorously. By testing against edge cases and keep it with consistent.

A good way to start is with the dictionary definition. "a statement accepted as true as the basis for argument or inference." For instance, our understanding of scientific realism is the understanding of the fact that our observations of reality are accurate.

A more defined definition can then be defined to say that an axiom is a statement that either cannot be proved or cannot be proved currently that is used within a greater statement that can be proved upon its edifice. These axioms, while don't require 'proof' generally desire some type of reasoning even if it is not objective or purely analytical in nature. For instance, the above arguments for the objective nature of reality aren't true objective proof, they still exist as semi-arguments. While we cannot prove the objectivity of reality with true and objective reasoning, some reasoning can be applied to get some kind of 'proof' of the statement so that further studies can be applied based upon that said axiom.

An astute observer can find that these mean the same thing, my is more of an explanation so, the dictionary definition does work:
\begin{definition}
    Axiom: a statement accepted as true as the basis for argument or inference.
\end{definition}
\section{Defining our Axioms}
So, let us define the axioms used in our system.
