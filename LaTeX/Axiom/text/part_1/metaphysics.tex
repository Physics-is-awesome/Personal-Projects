\chapter{Fundamental Metaphysics}
\section{Scope}
This chapter will cover fundamental metaphysics, what I mean by this is for my progression of metaphysics I need a theory of epistemology, philosophy
of science, inductive and deductive research on physics, etc; however, for all of these things I need some concepts of metaphysics. Things like the
existence, reality, consciousness, questions of God, and so much more. Thus, this will serve that purpose.
\section{Reality}
\subsection{Existence Exists}

To truly come to a thorough understanding of a system, one must create it from first principle slowly and carefully. To question every implicit assumption,
this for the most part is the quest of philosophy; especially metaphysics.

Thus let us begin our creation of reality.

\emp{Existence Exists}, this is by definition an irrefutable point. To negate an argument an antithesis must be supplanted, non-existing, but non-existence by definition implies a existence.
This is because you cannot have a non-existence without an existence. $\neg E(x) \implies E(x) \iff ((\forall P(Prop(P) \implies E(p)) ) \land (\exists Prop(E(\neg E(x))))) $.
By the simple definition of existence to be everything. Another more concrete argument is that to deny the existence is a claim, a claim is an act, an act requires and actor, and thus an actor must exist,
an actor exists within the subsect of existence, so if it exists so must existence.

Though, this doesn't say anything about the existence, in fact there is a special case in which existence is the absent of all things. Using the second argument, we create
a conscious entity. \emp{Cognito, ergo sum}. I think therefore I am. Now upon inspection of this entity it is quite simple, the only capacity it necessarily has is
the ability to make a claim, such a claim needs no basis in reality, sense, logic, or anything of such. It could be the dream of a butterfly, random
interactions of particles in the dead of space, or anything of the such.

One constraint does exist, that entity is me. While it could still(at least for now)
be some random assortment of particles or what ever machinations you may dream of, there is some knowledge we do know. It has the ability to observe,
reason, and has innate concepts like space, time, similarities, and differences. One thing to be clear, at this point their is no justification that
any of these things have any connection to reality, but they are apart of the entities observations.

\subsection{The Entity and Observable reality}
Now, it is technically true that the entity has no way to truly and faithfully validate his observations and reasoning beyond self-referential, due
the fact that these are the entities' only capacity to make claims.

Thus, taking these as true is a fair axiom, unprovable but logical in the necessity of the argument. I do not mean taking all observations as true(in fact always questioning them is of great importance as I will
address later). I mean regarding that our observations are a consequence of reality. Taking the reality that existed around us as the "truth." While
observations can be complete, wrong, they are not false by the nature of being observations by the entity.

Now this creates a dichotomy between the observed reality and "true reality"(whatever that might be). There is a spectrum of how much these two overlap.

Ranging from:

A: Scientific Materialism; that all of observed reality and true reality overlap because these observations are directly true, and nothing exists beyond
what is within the capacity of theoretical observation.

B: Idealism; there is no overlap at all. Such as the Boltzmann brain.

Now, this observable reality, is still apart of existence, and is thus no metaphysically distinct if some greater "true reality" and is thus given the
same importance and study. While other sub-existences beyond our observable are lesser for reasons, these are more ethical, aesthetic, and theological
rather than truly metaphysical.

\subsection{Beyond Observable}
Given this spectrum, it is entirely possible for there to exist unobservable reality that is true. This reality has unknown capacity and can theoretically
interact and cause observabes.

\subsection{Independence of Reality}
Given that this entity is me, I observe the continuation of reality beyond my direct observation. Once again this is not direct proof that it is, but
makes it indistinguishable from if it was.
\section{The Entity and Observations}
\subsection{Introduction}
Now this entity observes a great many things; matter, other entities, change, relations, space, time, identity, theory, mathematics, difference,
categories, individuation, emotions, structure, aesthetics, fundamentalism, layers, evidence, argumentation, possibility, necessity, abstracta,
causation, and much more. Categorizing these and understanding them is the foundation of metaphysics.

\subsection{Properties}
All observables are made up of properties, and many of the observables themselves are properties.
\subsection{Abstraction: Math, sets, and Relations}
Given this nature, these properties can act as sets of 1 and can make up object which are thus observables. They thus follow the same mathematical rules
of sets.
\begin{itemize}
    \item Relations to other sets(and themselves)
    \item Shared Properties with other sets
    \item can be apart of larger sets(categories)
    \item and so much more
\end{itemize}
Now I won't examine all of this because it is mentioned in the earlier chapter so we will move on.

These sets can be whatever, such as the other properties or something new all together.

\subsection{Matter}
One of the most important sets is matter, within this set is a near infinite amount of subsets. These subsets can "theoretically" have about any property.
The main one that define it is that it must have the capacity to affect the properties of other matter(and thus be effected too). It must also have the
properties of existing within space and time. Direct space or time is not needed, but having the property is required. Finally, it must be independent
of a thinker.

\subsection{SpaceTime}
Once again, given the simplicity of this model this examination will be quite simple. Essentially all that can be taken at this moment is that
space and time are properties of many things. Extremely important ones,

\subsection{Thinker}
A thinker is an entity like object of matter. These entities are capable of creating, holding, using, being effected by, etc by non-material objects.
These thinkers do not need to be human, or even alive for that matter.

\subsection{Non-material}
Already defined to be objects that exist(observable have effects) dependent of a thinker.
