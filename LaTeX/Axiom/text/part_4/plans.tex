\chapter{Plans}
\section{Now}
\subsection{Self-Education}
elf-education has always been a great value of mine. As mentioned earlier, I became obsessed with learning Einstein's field equation in 3-4th grade. Beyond that I have spent much of my time reading books, watching YouTube videos, and more about topics that interest me. Though, late in 8th grade, this desire to teach myself grew a great deal. Before I can go about my future goals for teaching myself I must first go over what I have already down.(This list is not full, mainly lectures finished and books from lectures, this does not include books not finished, non-lecture based video education, research papers read, projects where I learned things for the project alone, and other similar programs)
\\
Mathematics:
\begin{itemize}
    \item AP Calculus AB and BC - Khan Academy
    \item Multivariable Calculus 
    \item Khan Academy
    \item 18.03 Differential Equations - MIT OpenCourseWare
    \item Gilbert Strang on linear algebra - MIT OpenCourseWare
    \item Vector Calculus - Trevor Bazett
    \item Tensor Analysis - eigenchris
    \item Tensor Calculus - eigenchris
    \item Crash Course in Complex Analysis - Steve Burton 
    \item Introduction to Applied Numerical Analysis - Richard W. Hamming
    \item Symplectic geometry \& classical mechanics - Tobias Osborne 
    \item Lie groups, algebras, brackets - Mathemaniac (more conceptual than mathematical)
    \item Differential geometry - Robert Davie
    \item Calculus of variations - Faculty of Khan
    \item Working on
    \begin{itemize}
        \item Advanced Analytic Methods in Continuum Mathematics: Fundamentals for Science and Engineering - Hung Cheng 
        \item Discrete Differential Geometry - CMU
        \item Differential Geometry for students of Numerical Electrodynamics - Alain Bossavit
    \end{itemize}

\end{itemize}
Physics:
\begin{itemize}
    \item 8.02 Physics II - MIT OpenCourseWare
    \item 8.03 Physics III - MIT OpenCourseWare
    \item 8.033 Relativity - MIT OpenCourseWare
    \item General Relativity - Stanford Online
    \item Introduction to plasma physics - USYD senior plasma physics lectures
    \item Introduction to electromagnetism - Griffiths
    \item Computational physics - Mark Newman
    \item 8.224 Exploring Black Holes: General Relativity and astrophysics - MIT OpenCourseWare
    \item Classical Dynamics of Particles and Systems - Thornton Marion
    \item Introduction to cosmology - Stanford Online
    \item Introduction to fusion energy and plasma physics course - PPPL
    \item Seminar: Fusion and plasma physics - MIT OCW
    \item Statistical Mechanics - Stanford Online
    \item Hamiltonian description for magnetic field lines in fusion plasmas: A tutorial - AIP
    \item Fusion economics: power density, materials and maintenance - PPPL Frontiers Colloquia
    \item Flash-X code tutorial, a users perspective - University of Chicago(skimmed) 
    \item Flash-X user guide - Flash-x
    \item Flash4 User support(skimmed)
    \item Radiative Procuresses in Astrophysical Phenomena- George B. Rybicki, Alan P. Lightman
    \item Goldstien Classical Mechanics Lectures - Prof. Jacob Linder
    \item PiTP 2016 - Institute of Advanced study
    \item 2024 PPPL Graduate Summer School
    \item Hamiltonian description of the ideal Fuid - P.J.Morrison
    \item Quantum Mechanics with Applications - David B. Beard \& George B. Beard
    \item "What IS Structure, How Do You Create or Recognize It, and How Can You Use It? or Metriplectic Dynamics: Using the 4-Bracket for Constructing Thermodynamically Consistent Models." - workshop Geometric Mechanics Formulations for Continuum Mechanics
    \item "On an Inclusive Curvature-Like Framework for Describing Dissipation: Metriplectic 4-Bracket Dynamics." - workshop Infinite Dimensional Geometry and Fluids
    \item Haven’t finished
    \begin{itemize}
        \item Theoretical Physics - Georg Joos
        \item Advanced MHD with Applications to Laboratory and Astrophysical Plasmas - Cambridge
        \item Classical Mechanics: A Modern Perspective - Sudarshan and N. Mukunda
        \item The Interpretation of Structure From Motion - Ullman, S.
        \item Radiative Processes in Astrophysics - George B. Rybicki \& Alan P. Kightman
        \item Symplectic geometry \& classical mechanics - Tobias Osborne
    \end{itemize}

\end{itemize}
Coding:
\begin{itemize}
    \item Computational physics - Mark Newman
    \item Python Numerical Methods - Berkeley 
    \item Applied Numerical Methods - Crice Carnahan, H.A Luther, James O.Wilkes
\end{itemize}
Other:
\begin{itemize}
    \item Management in engineering - MITOpenCourseware
    \item Principles of microeconomics - MITOpenCourseware
    \item Dynamic leadership: using improvisation in business  - OpenCourseware
    \item Logic 1 - OpenCourseware
    \item Policy for science, technology, and innovation -  MITX
    \item Reducing The Danger Of Nuclear Weapons And Proliferation- MITOpenCourseware

\end{itemize}
\subsection{Research}
\begin{itemize}
    \item Egg drop simulation
    \item Numerical Methods for 3D compressed Plasma using Lattice Boltzmann(paper written)
    \item Simulated Annealing for plasma thrusters
    \item High Altitude EMP for primordial black holes(never finished) 
    \item Flash-x simulations:
    \begin{itemize}
        \item Basic Sedov
        \item Running simple MHD simulation of my own design
    \end{itemize}
    \item A Hamiltonian Framework on ICF Implosions Rocket Equation Based on Rayleigh-Taylor Instabilities
    \item Contributed to PlasmaPY
    \item Relativistic Rocket Equation(Newtonian)
    \item Relativistic rocket equation(Hamiltonian)
    \item Finite difference MHD Model in Fortran
    \item Computes velocity based upon Navier Stokes equation adapted to magnetic and electric fields
    \item Outputs graphical color field of velocity 
    \item Asteroid game
    \begin{itemize}
        \item Classical
        \item Time
        \item Survival
        \item Newtonian(player and asteroids have gravity)
        \item Dark Matter(Invisiable gravity fields)
        \item Relativistic(time dilation, spacial contraction, black holes, etc)
        \item Proximal Policy Optimization “AI” played game
    \end{itemize}
    \item Hamiltonian based plasma thruster Fortran Numerical Code
    \item N body simulation 
    \item Fluid based N-body simulation
    \item Death Star
    \begin{itemize}
        \item Basic gravitational ‘planet’
        \item MHD Plasma beam hitting
        \item Basic evaporation and kinetic energy transfer
    \end{itemize}
    \item OpenMHD usage
    \item Usage Gkyl
    \item Omega-X project
    \begin{itemize}
        \item Inspired by FLASH-X, attempting to build a multi-physics plasma physics codebase using Metriplectic forumism rather than Lagrandian and Newtonian
        \item I have successfully implemented the 1D thermofluid metriplectic discretization onto a discontinuous Galerkin subspace using an $L^2$ projection. Integrated using implicit midpoint rule.
        \item Working on module registry \& driver pattern
    \end{itemize}
\end{itemize}
\subsection{Other activities}
I also had some other activities that I have done in high school, I won't go too much in them. 
\begin{itemize}
    \item SVA
    \item Boy Scouts
    \item Church activities
    \item JROTC
    \item FCA
    \item OA
    \item Etc
\end{itemize}
The OA, Order of the Arrow, I will continue into college.
\section{Pre-University}
\subsection{Independent Learning}
One thing I need to do is nail calculus completely. I am going to be taking the AP calc BC test without taking the course. 

Beyond that, I am completing differential geometry. Once done I will finish Symplectic geometry. Then lie (brackets, algebra, group, and theory)

I also want to get into functional analysis.

I should also learn C++ beyond what I have done in the past.
\section{*Undergrad}

\section{*Grad}

\section{*Early career}

\section{*Late Career}
