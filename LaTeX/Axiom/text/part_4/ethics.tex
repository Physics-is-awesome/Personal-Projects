\chapter{Ethics}
%%%%%%%%%%%%%%%%%%%%%%%%%
"Finally, brothers and sisters, whatever is true, whatever is noble, whatever is right, whatever is pure, whatever is lovely, whatever is admirable—if anything is excellent or praiseworthy—think about such things." | Philippians 4:8

\section{Introduction to ethics}
\par Many that know me, know the question I love to ask, "what is your moral philosophy?" I love this question for many reasons. First, it intrigues people, they love to talk about their own morals and beliefs, but engages their curiosity with these newish words. Second, Morality is something that is both simple and fundamental enough that everyone has thoughts of it but still complex enough that it can have engaging discourse. Three, it get to explain stuff, most people haven't heard of virtue ethics, deontology, egoism, etc.

\par Despite this fundamentalness to ethics, many questions are still unanswered, which fundamental theory? Many questions are raised for each theory that must be addressed? Even people question if Morality is even objective or not?
\section{Virtue Ethics}
\par My fundamental theory is virtue ethics
\par Virtue ethics is based upon the idea that Morality is derived from living a virtuous life rather than the adherence to rules or a logical stance. That you must pick a set of virtues and live by those principles.
\par Now why do I choice this over the other types of theories of morality and ethics. To understand this I will first criticize(yes criticize not critique) other theories then I will defend virtue ethics.
\par First, deontology. Deontology comes from the Greek word deon meaning duty. That you have a duty to follow certain 'rules.' That morality is based upon a set of rules and concepts rules and concepts to follow. Say, don't kill, don't steal, give to others.
\par My problem with this is not that it is too ridged, but that it is impossible. It would be impossible. You can't make rules for all possible interactions so must theories of deontology tend to just be contorted virtue ethics. Also, that both the action and consequences should be included when making an ethical decision but generally rules can lose this complexity.
\par Next, egoism. Yeah no
\par OK, fine, I'll do a little more. Most theories of egoism fall into either three camps; nonsense hedonism(which needs no criticism), logical consequentiality based concepts, or altruistic egoism. First, the consequentialism type eventually leads to normal consequentialism but with more steps to find the conclusion so I will focus on that when I get to it.
\par Next, consequencialism. Frist, for consequencialism to work a set of values must first be prescribed(so it isn't fundamental) there are others like utilitarianism but they aren't self-justifying nor truly justified by other means. There are some that exist but I will show that to analyses all in-depth is a waste of time. My reasoning is that both actions and consequences should be analyzed. Why, good question... Well, first there is religion obviously. Though a more secular reasoning is that ethics is used as a binding agent, people emotionally don't bind well to 'immoral' actions, and those that do tend not to bode well with other things.
\par Now for justification of virtue ethics. The first and most important is that I am a Christian, from my analysis most Christian ethics is highly tided in virtue ethics. The fruits are virtues. Most sins are vices. Most ethical advice is based upon abstract virtues rather than a simple value(consequentialism) or rule(deontology). While there are many rules, most of them are based upon the virtues later mentioned rather than justified by themselves.
\par Next, on a more secular level, virtue ethics are far more complete and level for people to analyze, create, and apply to your life. They can included anything you want to have in them without creating paradoxes and problems(as long at the two concepts are not are paradoxical in it of themselves)
\par I will show this through the future sections that go through a list of virtues and why I feel they should represent my ideas.
\par One final thing I will talk about, is something that may be a strength or weakness depending on how you look at it. Virtue ethics has a much more 'subjective' connentation. A lot of virtues that I value another may not, also these virtues can be analyzed in your own way.
\par The way I see it, some virtues are fundamental and should be applied to all given they are principles derived by God, others I feel as personal virtues and feel no sense that they should be forcefully applied to others(though I feel things would be better for all that way.)
\par In essence, the idea of virtue ethics is to pursue perfection of the mind, body, moral thought, etc in order to create a purity of intention. This is the goal, not absence of sin, but the perfection of moral intentions. To create yourself in the image of your own ideal and to pursue it. That morality isn't just a responsibility but ontological alignment. To pursue the ideal of Jesus.
\section{Stewardship}
\par Throughout the rest of this chapter, many times I reflect on morality as my values. Values that are observed because of some 'internal promise.' This is because it is the way I have naturally thought, though I am trying to evolve past this.
\par I have always seen the internal promise as more pure than any other form of moral enforcement, though I have found this not to be true in the highest extent.
\par This obsession with dignity and following through with my own values alone is cornered in the vice of pride. It sits upon it, being thus controlled by it, and by proxy then so am I.
\par Now I have found, not responsibility, but existence in stewardship. In stewardship to the will of God, his grace, his mercy, his creation, and most importantly each other.
\par You see, God brought Adam and Eve into the world, as images of himself: builders, molders, and beings capable of reason. A creation to tend the garden, name and care for the animals, and to care for each other. This was the reason for humanities existence.
\par After the fall of mankind, humanities purpose shifted, but not much. We are still arbiters of his will and creation, by authority he has given us. Though, because of the imperfections made by us, our responsibility is to fix them. To perfect the world around us through the words that God has authored for us.
\par For God himself has given us this divine commission; though he hasn't left us to do it alone. He gives us gifts, hope, and healing. He feeds us and nurtures us so we made be sanctified in his love for us.
\par Now in practice, this becomes our purpose of moral perfection. Both for ourselves and for those around us. To create, not only pure intentions for ourselves but for the society and people around us. I will discuss this more throughout this book, especially this chapter, but here I will leave with this.
\par Though our actions, we may be able to co-create with God to bring Heaven on Earth. Bring about social holiness caused by our own actions with others. This is the purpose of Christianity, of ourselves. Only secondary to a relationship with God.
\section{Categorical Good}
\par Based upon my religious beliefs I propose this axiom, "that all of reality is designed in such a way in that "good" can be logically inferred and the actions to find this good is what we call virtue."
$$\forall x( R(x)\rightarrow \exists P(I(P,G(x))) \land \forall A(V(A)  \iff \exists y(L(A, y) \land G(y))))$$
\par In essence, there exists a good for all categories(ex. communication, get information delivered clearly). Thus, there are good ways to do this, virtue(ex. use precise language when needed but don't use overtly complex language when not). Finally, not only can iterations of this build up an idea for how to "be a good person" based upon a complex model of different circumstances throughout life, but that what virtues exist reveal what makes a person "good" by what God has consciously divined, giving the actor "wisdom". This is all centered around the Stewardship given to us by God as mentioned in the previous section.
\par Obviously, this isn't a purely religious view. Aristotle came through with an idea very similar idea, but even his idea had a pantheistic view, and other secular versions generally make it an unquestionable axiom without further backing other than the fact it creates a non-paradoxical system. Also, this system allows further development, like founding it in stewardship or having all things reveal "wisdom" to further develop virtue.
\section{Cardinal Virtues}
\par Now, while my cardinal virtues have basis in the traditional cardinal virtues of the catholic faith they are not the same. The name comes from the fact that cardinal comes from the Latin word 'cardo' which means hinge. In essence all other virtues hinge on these concepts.
\par Here is a short list of the main ones
\begin{itemize}
    \item Temperance: Excess in most matters proves detrimental; self-control is paramount in one’s life.
    \item Prudence: Deliberation is essential before action in all circumstances.
    \item Fortitude: Mastery over one’s emotions is a paramount virtue, as the inability to do so inflicts harm upon oneself and others.
    \item Faith: Maintain strong belief not only in God but also in oneself and one’s values and beliefs; these should remain immutable unless confronted with supreme evidence.
    \item  Duty: One has obligations to oneself, family, others, and one’s values, which should drive one’s life. Duty to oneself includes ambition and adherence to personal values; duty to family involves support and politeness; duty to others is similar but to a lesser extent.
    \item  Individualism: Embrace self-reliance, self-respect, and ambition, and adhering to personal values.
\end{itemize}
\par As you can see all of these virtues are simple and widely acknowledged and don't require much additional justification. The only true exception being individualism....
\section{*Salient Virtues}

\section{The Philotimic Virtue}
\par Now Philotimic is a word I have made up, I will go into more detail about what exactly it means later, for now assume it means pride.
\par Now, I hear your vapid and lost mind screaming, "isn't pride a vice, what are you doing" Though I will examine why I think pride in some forms is a virtue not a vice(at least certain types of pride)
\par First to examine we must first define pride and what it is. One definition is ,"feeling of deep pleasure or satisfaction derived from one's own achievements, the achievements of those with whom one is closely associated, or from qualities or possessions that are widely admired." or "consciousness of one's own dignity." Now what does this mean and how do they fit together.
\par Let us start with the first one, a deep feeling of pleasure in ones own achievements. This can manifest itself in many ways; work, morality, intellectual achievement, and so many others. Beyond that it can effect people's actions in both positive and negative ways. Let's start with negative, it can lead to obsession with ones own capacity and the product of such. On the flip side, a healthy obsession will lead to personal growth on whatever they feel pride in. Also, it can give them confidence in their own capacity. An example would be with morality, if they feel pride in their own sense of morality it can lead to self-assurance in their own senses leading to them being able to apply and do them, but it can lead to close-mindedness and looking at just your own thoughts and nothing more.
\par Now lets look at the second definition, consciousness of one's own dignity. Meaning that they are aware and influenced by concepts of their own dignity. The belief that somethings are within their ability and others are beneath them.
\par Now what I mean by philotimic, as essentially the positive aspects of this. That to have philotimic virtue is to take your own life seriously. To be principled, to be absolute. Not just in behaving moral in a moral and dignified manner, though this is definitely a part of it. In fact, even within this manner this virtue is partly about fanatical pursuing such moral perfection. Though, outside of this there are many things. Most notably is having such fanatic belief in things beyond morals, other values like being early, being prepared, working hard, being mature, and being other similar concepts. That while there are values that God doesn't arbitrate, but you do, and thus you should keep them to almost the same esteem.
\par This concept, as mentioned earlier is about taking your own dignity, self-esteem, self-respect, virtue, and above all life seriously. Because that is what you should do, take it seriously in a dignified and ridged manner. To never compromise upon even the most minimalist value, because to compromise on such things is far worse than any other conceivable interaction.
\par Another concept within it is control. Control over your inner-mind. Your thoughts, emotions, temperament, and even personality being bent to the will of your own 'ego.' Not ego in the regular sense, but as in the inner consciousness of your own self. The self of morality, values, reason, and more(more closely related to the idea of 'superego')
\par In essence, recursively creating your own self based your higher values rather than letting your environment impact your supposed values.
\par Philotimia in its highest sense, is not merely pride, but the love of what is worthy of love. To find what is worthy of such esteem and pursue it with fanaticism and create your soul around such values. To take responsibility for your own existence, morality, self-esteem, and so much more.
\par An even further simplification is it is taking upon yourself the responsibility to be human. The responsibility to think, to fear God, to live accordance with your values, to have values, and to live in the image of your highest self. Not only that but bend the world into the image of your highest self.
\\
\\
\par One thing I will add, is in the conflict of universal morality(derived by God) and personal values. This conflict doesn't really exist. Both of these values exist in my mind. While universal morality is obviously stronger, this doesn't dissipate the importance of personal values. The largest differentiator is others; for universal morality, others disobeying. is an infringement of morality, while for personal values only you 'break them,' no one else.

\section{*The Internal Promise and Dignity}
\section{Good Life and Eternal Struggle}
\par There are two parts to this. Obviously the good life and the eternal struggle as defined in the title. This are both extremely connected, as I will show momentary.
\par First, the good life. To live a virtuous life is to live a good life, but there is challenge. One, you should train yourself to desire this good and enjoy it. as defined in the philotimic virtue, take this training upon yourself rather than simply passively let culture and external figures train for you. While yes, culture can do a good job, training yourself creates a better feedback loop and consequence; beyond that, training yourself is your own moral responsibility as a human being.
\par Find this good to be your highest desire, so that you not only can but will naturally follow it and ignore sin and vice around you. To desire it fanatically. While many consider fanaticism to be a bad thing, in the pursuit of virtue, it isn't just a good thing but a lack of it is evil.
\par Beyond that, this struggle will be enteral. Virtue by its very nature is never perfected. Therefore, you should always struggle. To take this further, you should love the struggle against it.
\section{*Personal Virtues, Values, and God's Morality}

\section{Intentions}
\par In my theory of morality, intentions matter just as much as the action in it of itself. "The LORD does not see as man does. For man sees the outward appearance, but the LORD sees the heart." 1 Samuel 16:7.
\par I could go on and on, but this idea is relatively intuitive and while many don't consciously think this way it isn't far from our minds.
\section{*Morality and Religion}
\section{*The Ethics of Knowledge}

\section{Recursive Connection of Civilization \& Virtue}
\par Our capacity towards civilization and virtue are strictly linked in a depth rarely acknowledged.
\par Let's begin with the start, for civilization to happen, we must be able to do it well. This is the heart of virtue, while virtue can be connected with any doing any good thing well, the ability to be civilized and do civilization "well" is the very heart of it.
\par Now, how do we do civilization well? To make sure people do not become distrustful, honesty should be brought about. To bring about any positive outcome, discipline becomes a need. To expect pro-social activities, empathy is not only an ideal but a requirement. While I won't explore every angle, it is clear where this leads. Our modern ideas of culture and morality is primary based upon how to build a civilization, a society.
\par Though, how to develop virtue on a mass scale? Society. Through community virtue is developed, the most obvious being how we enforce guilt and pride into those that exhibit virtues and vices. To teach them young, how ethical theory works.
\par A bit deeper, through habits. Virtues are all about who we are, our attributes, our character, presence, intellect, and much more. We become who we are through our habits; habits of action, ideas, thoughts, and much more. Society creates these habits, even when we don't realize this. To explore this idea, I will present a case study: a commonly criticized culture rule is to not curse. It may seem illogical, they are words, sometimes not even directed at the person offended. Now, before I begin on this analysis, I will say that this particular analysis will focus on social rules as habit forming$\implies$ virtue forming. It will ignore the dozens of other reasons and arguments.
\par From this idea of it being illogical, you can clearly see the reason by what the person is actually doing. They are \textit{tempering} their own \textit{impulses} for the sake of \textit{social harmony} and \textit{empathy} towards others. Having \textit{interpersonal tact} and learning from \textit{moral authority and teachers} without rebelling because you didn't see the big picture. These habits are both skill and virtue building so we may be civilized adults. This is also why it is so extremely emphasized during childhood.
\\
\par Now, while these parts are extremely important, they aren't simply circular self-defining concepts. Virtues exist beyond civilization and their primary justification being to mimic Christ and fulfill our duty to steward his creation.
\subsection{*Social Rules and Morality}
\subsection{*Culture as Moral Education}
\subsection{Intensity of Duty*}
\par While on a strictly moral view, ethics and their virtues are extremely important. I have gone over again and again, how these ideas should be pursued fanatically, but individually. While they should be taught to others, expecting purely moral actions from other is not something to be expected from them.
\par Virtues of civilization are another story. Some lenience is required, but to much leads to a denegation, a slippery slope. Things like duty, respect, loyalty,
