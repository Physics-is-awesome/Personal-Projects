\chapter{Metaphysics}
The verbal interpretation, on the other hand, i.e. the metaphysics of quantum physics, is on far less solid ground. In fact, in more than forty years physicists have not been able to provide a clear metaphysical model. | Erwin Schrodinger
\section{Scientific Materialism and God}

\par Despite the fact that there exist no
formal term for such \\
philosophies(I should come up with a name for it, Christian Scientific Materialism?)it in a broad stroke is most likely the most common view of meta-physics of all, though I plan to expand upon the concepts more deeply than that which is common. In essence this view can be summarized by the combination of two seemingly contradictory concepts:

\par    First, is scientific materialism, this is the idea that there is nothing further than the material, that reality can be derived empirically through the laws of science and nature.

\par    Second, the idea of the Christian God. That there exist an external omnibenevolent, omnificent, omnipotence, omnipresence, and omniscience as described by the Christian Bible. Later I will explain my faith in more detail, but for now this is sufficient.

\par You may be wondering how these concepts can be combined due to their seemingly contradictory nature. This is a simple enough question, if the material is all that exist how can a non-material God exist? The answer in all reality is very complex, but it can be simplified to the fact that I view the material world to follow the philosophy of scientific materialism but there to also be a non-material "world" that God exists within. The reality of such a non-material world is sadly beyond my grasps due to my life within the material, I make no assertion that I understand such things. Partly because by definition such things are not possible to understand, like a 4th spacial dimension. While we can understand it through analogies and abstraction based upon prior more concrete idea, but sense we have no clear or objective priors we have no capacity to do such. \\
You can simplify such thoughts through formal logic \\

Let: \\
X=Material \\
Y=Non-Material \\
C=Laws of nature and reality \\
C(X)=Matter/Energy/Both interactions(all material action) \\
C(X,Y)= Observable phenomena
A=Scientifically explainable phenomena \\
$$
X \implies C
$$
$$
C(X) \iff C(X) \models A(X) \top
$$
$$
\forall C(X,Y) \exists C
$$
$$
\nexists Y(C) \therefore \nexists A(Y)
$$
\section{Scientific Realism vs Anti-realism}
\par What is scientific realism and anti-realism? It is the debate between whether our theories of science are ontologically real, or simple predictive models.
\par My view is that it depends, when it comes to macroscopic and easily observable and prove able; these theories aren't true. You may be thinking the opposite, that it is those that are most definitively true and the rest that aren't. Though I would argue these are the models; a ball don't fall; a collection of subatomic particles follow their geodistic, which is restricted by other particles and forces, and so much more. This can be applied to all things other than the most fundamental of physics. It can especially true for the softer sciences; chemist, biology, psychology, and so many more.
\par While I think science does study the metaphysical truth through fundamental physics and abstract mathematics, on a broad stroke it does not.
\section{Causality}
\par Now I am sure you are confused, I mentioned I would dive into deeper questions related to free will, determinism, what is real, divine voluntarism, occasionalism, how quantum mechanics plays within this and other 'deeper' concepts, but we must take things one at a time.
\par So, let us dive in. What is causality, it is very simple. It is the concept that things happen because other things had happened first. That all things  have a 'cause' and that that thus causes the effect(then they later become the cause of a later thing)
\par I can hear you screaching that this is non-sense, who in their right mind would question something as basic as cause and effect? Why even look into it. First, that is extremely intellectually lazy, all things have debates against it, you should always look into it. Though, more importantly, there actually are debates against it.
\par The easiest to dismiss is Hume's skepticism, that causality is simply a human made concept. That things doesn't really happen because of another action but simply that both actions are just sequences of events. Though, I would argue that this pedantic argument of definitions doesn't truly work. On an epistemological basis, causality is easier for humans to understand, and though breaking things into chunks we may better understand the world. That either concept is equally true, but one concept as a greater pragmatic 'truth' and you can argue that that correlates to a better empirical truth. That there is even a speed of causality written in the nature of reality.(The speed of light) A bit more deeply, once you quantize to the plank time, things truly do divide. Something happens, then a next thing happens. The complexity and abnormal reality of quantum mechanics makes this argument a little fuzzy, and thus leads me to my next paragraph.
\par Now those unfamiliar with quantum mechanics may be confused, how does quantum mechanics may be a little confused. I will attempt to explain it to the best of my ability, but the reality of quantum mechanics is extremely complex and without a clear understanding of differential equations, discrete mathematics, linear algebra, and much more it is impossible to understand. I know many science communicators claim to teach "quantum physics and mechanics" to they layman, but this is a lie. These half backed analogies are not quantum mechanics. It does not come close, but luckily for me I am the only one reading this book I will be able to understand my attempt of analogizing(?) the philosophical question.
\par In essence, in quantum mechanics. Things are not local, for many reasons. The most basic is the idea that particles do not exist in the way we can imagine. They exist in multiple places at the same time[not really but useful analogy], simply with different probabilities(don't pretend to understand, nobody does, they just prove the math and accept as is.) You know what, that analogy was not enough. I will explain in more detail. Particles do not exist(not really). Simply quantum fields exist. The electron field for electrons, the photon for photons, and so on. The particles are not particles as we imagine, but excitations within the field. It is a bit more complex than that, they can act as particles when observed but that is a whole other can of worms. Now back to the matter at hand, now waves act in certain ways that is hard to explain. Due to their quantum nature they must be decomposed first through Fourier modes to under stand more clearly. This is an example of a very simple form:
$$
\psi(x) = \frac{1}{2\pi} \int \phi(p)^{ipx}dp
$$
\par Now I won't explain the equation in its entirety but I will say this. That you cannot solve it for an absolute certainty of position(not that you could get a "true" answer even if you could given that is is a Taylor Mode and not an analytical equation). Now you may think this is just a quirk of the math, but it is not. It is real for reason that could take up the entirety of this book if I truly understood them.
\par Next, quantum mechanics goes against causality through entanglement. Now what do I mean, when people refer to the speed of light, they are actually referring to the speed of causality. To explain this is relatively simple with a good understanding of Special Relativity. It is obvious that the universe is invariant to position and velocity(this has been previously proven thoroughly through experiments in high speeds, specifically through electromagnetism but others can apply), but for this to be true there must be a finite constant cosmic speed(because it requires a Lorenz transformation[there is more explained why this si the only answer but I will leave this hear because I have much much more physics to talk about in a philosophy book.])  This is the speed of causality, the speed of information. If something were to go above this speed it would in essence go back in time. This can be proven rather simply through simple algebra:
$$t'=\frac{1}{\sqrt{1-\frac{v^2}{c^2}}}(t-\frac{vx}{c^2})
$$
\par From here you can look at the square root and easily find that if $v>c$ then $\gamma $(the value of the first part with the $\frac{1}{\sqrt{1-\frac{v^2}{C^2}}}$ would be imaginary. Thus the t value(of time would then become negative, this obviously breaks causality.
\par Finally, back to quantum mechanics, as you know entanglement transfers information at faster than the speed. This seems to break causality.
\par Now I could keep on talking about arguments between quantum physics and causality, but I will only talk about one more. Virtual particles. Now first of all, what are virtual particles. Virtual particles are non-existent but existent particles that exist within the math of quantum field theory(and thus above my pay grade) but I will attempt to explain it. The idea in essence is that due to the uncertainty principle there exist small amounts of excitements within their fields. These excitements are not enough for an actual particle to exist in the way we imagine, but occasionally a particle can appear, primary through the use of "borrowing" energy from another particle or through creating a negative particle(not to be confused with anti-particles those are negatively charged, these have negative energy/mass) These virtual particles are especially important to particle interactions, what I mean is the fact that when two particles are "close" enough together they excite the fields around them to create bosons(force carriers) to cause the actual interaction. As an analogy, imagine two electrons, when close enough together they excite the fields around them to create a virtual photon, the photon then interacts with one electron(and also effects the other electron negatively[it is almost impossible to explain]) and cause the electromagnetic repulsion between the two. Now, back to philosophy, what in the world does this ramble have to do with causality. That is a great question. It is that all of our interactions are based upon something that isn't real. Electrons don't actually create virtual photons, these photons don't create negative photons. Despite having a tangible effect, they don't actually exist. How can something cause another thing if when you break it down to its component parts, particles are behaving 'irrationally' being effected by mathematical objects.

\par Now after I have listed the critiques, you are probably very aware about how you assumption of the obvious state of causality was intellectually lazy, but you probably also realize I wouldn't have written all of this unless I have an answer to these critiques, or at the very least a possible answer. In all reality I could be wrong about this and causality could be false. Though I am never wrong, so that must not be true.

\par The uncertainty of particles doesn't necessarily falsify causality, it simply evolves it. We previously see causality as something that is caused by a specific thing, but it can be caused by a probability of things(if that make sense).

\par Next I will talk about virtual particles, and yes I hear you saying that I should go in the order of which I introduced the ideas, but I do not care. Now, virtual particles. How does causality work with things that are not 'real'? Now the most astute among you might claim, why would causality not apply, and you are exactly right! You all may feel a sense of betrayal,  that it was simply a trick of the light. But now, listen carefully, this is how philosophy works. Though asking questions that are hard to understand, questions that play tricks on our language, on our mind. Though these questions we refine our thoughts. This is the problem with modern philosophy, they confuse critiques with criticism. While criticism has a place within philosophy, it should not be as big as it is now. Now before I continue I will define why I consider a critique and a criticism as different things. A criticism takes a part an idea, concept, action, or really anything for the sake of taking it down. This can be justified if such ideas have no worth and the criticism is an attempt to prove it. A critique is a question used to refine a thought not destroy it.(though if the critique can not be resolved then it may.) This critique of virtual particles is not supposed destroy causality by any means, but like the earlier critiques, it is meant to refine our thoughts. It asks us to question what a cause is, what can be a cause. Can virtual particles be causes, modern science suggests yes, but this requires a refinement of our concept of causality. Of what is 'real.' In a later section I will delve into this in a future section, but now I will leave you to ponder.
\par Now finally, to entanglement. Does this refine our definition or break it? It breaks it, but due to the evidence surrounding causality, including the fact that the special and general relativity are simply logical derivatives of the idea of a finite special to causality suggests that our idea of entanglement is flawed not our idea of causality. Now what is the answer? In short, nobody knows. There are many theories trying to come up with a solution. These include Einstein-rosen bridge, super-determinism, and local hidden variables theory. Though since these ideas are both extremely complex and offer little beyond the problem at hand I have elected to move on. Now you may say that my Faustian arrogance has lead me astray to ask questions I can't answer, and to that I say... No dummy.
\par Now you may think we are done, we are not. We haven't even mentioned the nature of God in causality and the teleological nature of Hamiltonian mechanics. There is also further refinements of time which must be discussed to even further define causality.
\section{Divine Law Theory of Causation}
\par You thought I was done with with causality. You thought wrong.
\par Now, what is the Divine Law Theory of Causation. It is very simple, it is that God predefines the laws of physics and nature. This differs from the ideas of occasionalism. Occasionalism is that all actions are the conscious decision of God, that the uniformity of the laws of nature is simply a coincidence.
\par Now, why do I choice Divine Law Theory of Causation? While I do not attempt to venture into the idea that I can even begin to comprehend the machinations of the God most high. Though in my mind the idea of predetermined laws make more sense, a more efficient method for God. Though this is not my primary reason for this belief. As I will mention latter in this chapter I will explain several beliefs of my predicated upon divine law theory of causation. Also, it is more in-line with scientific materialism
\section{Material, Immaterial, and Reality itself}
\par Now we are getting to the meat of things. What makes something real? What makes something itself? How do we define such things?
\par These are all great questions that I will go through one by one.
\par I will begin with what makes something real. While it is easy to look at things in the materialistic view, that of which that can be observed and have an observable effect is "real." No this has some implications, what about what about God(this is based on my assumption that he is 'real') but more importantly what about things that have an effect but of the concept of virtual particles.
\par First, the immaterial. Given the fact that by definition, these things are beyond us I will ignore them because we cannot gain any intuition from it. Just remember that it must be included within the concept of real, simply we are unable to define it accurately.\\
M= Material\\
I=Immaterial\\
$\mathbb{R}= Real$
$$
M \land I\subset \mathbb{R}
$$
\par Finally, back to virtual particles, I have kept you in suspense enough. How do we satisfy the existence of virtual particles? Just as we have several times before, say it with me, we must refine the definition. Now, how to do this. We must first realize that at the subatomic level, no particles are particles in the way we describe it, as I mentioned earlier they are simply excitations within a field. They are energy in a specific manner. rom here we can divide the material into two types; true material and partial material. The true material refers to what we generally consider matter, and the the partial(photons, gluons, and other massless but still interacting objects.) Which leads me to my final definition; material is that with observable effect[observable does not just simply mean seen, it can be any form of observation].
\section{Identity}
\par While many have abstract ideas on the identity of object, I will simple say that all subatomic particles are identical and thus it is not hard to define, and the others are simple as humans(another word defined by humans) define it. Good day gentlemen
\section{SpaceTime}
\par Space and time, the two things most clearly and least clearly understood at the same time. Things that we know do to our interactions with it every second(ha-ha) of our lives but still don't truly understand.
\par Many philosophers have questioned the reality of space and time. Are they simply human made? I would suggest not, the mathematical empiricalism and importance of them in higher level physics points to their real nature.
\par You can see this clearly through special relativity, there is an implied "proper" time at which things like electromagnetism operate correctly and without it our modern view of physics collapses.
\par Now you may ask about quantum mechanics, where such ideas of concrete space and time become fuzzy. Where particles exist within multiple place, some theories suggests 'time travel"(not real time travel) and other. Though, as I have mentioned before, I think his more evolves our ideas matter than our ideas of space and time. Also, there are examples of the objectiveness of space and time through things like quantum field theory(combines special relativity) and the Plank constants. You can't have constants of space and time without an objective reality of space and time. Now, what is the Plank length and time. They are the units of which distance and time become quantized, and some suggest the smallest units possible. Now, that seems like an odd idea, especially considering how much smaller it is that anything real and the random constants within it $c,G,\hbar$. Speed of light(or causality), gravitational constant, and reduced Plank constant. This is because as you get smaller and smaller there is a new uncertainty, a 'plank' uncertainty. Beyond this the uncertainty becomes absolute. That the realities of a new quantum gravity take hold The easiest way to look at this is through finding at what point the wave-length of light(our primary method of observation) becomes a black hole.
\section{Interpretation of Quantum Mechanics}
\par The long awaited further explanation of my views on quantum mechanics. Throughout this chapter, I have hinted at these ideas of interpretations of quantum mechanics. Also, throughout this I have primary relied on the Copenhagen interpretation of quantum mechanics. Which may lead several into seeing this as my view of quantum mechanics, it is not(I will elaborate further later.) Though let us first go through the different interpretations of quantum mechanics.
\par The Copenhagen interpretation. In essence the Copenhagen, the most popular view among scientist, is  that quantum physics is impossible to understand through our classical views of metaphysics. That our theories of quantum mechanics works, and to ponder further is a waste of time given its impossibility. It takes our ideas of wave functions and entanglement as reality, not because it is real reality, but because the model works. That at the end of the day physics is not the study of reality in the way metaphysics is, but the study of models of the universe.
\par Though this explanation isn't really fair, so I will expand upon it. The key concepts are things like wave-particle duality, that particles are both particles and waves and that things like observation can effect how they interact with itself(yes particles interact with themselves) and with other things.[Though observation is never truly defined] Quantum superposition, something we all know fairly well and requires no further explanation. That quantum physics is probabilistic by nature, and truly probabilistic not governed by hidden laws berried beneath. There is a lot more to it but given that I have already explain part of it previously and how popular it is, you can fill in the gaps on your own. While this theory has been extremely important for our modern views, and important to explaining how quantum mechanics works without esoteric 'nonsense' it has a couple flaws like observer dependence and the fact that it quite literately states that it is not the ontological truth but rather a model.
\par Next, the many worlds theory. One of the popular interpretations by the general public. Basically, it is that when probabilities collapse; whether it is an interaction, wave-particle duality, superposition, or many of the several other things; the universe splits into two(or however many are used to describe the 'whatever'). That one universe has one and the other has the other. This is quite outlandish idea; for its reliance on the observer and the fact that there is no mechanism of 'universe splitting.' One thing I would like to add, is that I have always hated how people use the idea of a multiverse in this context, it is multiple verses it is one with multiple observable components. A better term, generated by Sir Roger Penrose and a Latin Professor that he was friends with(which I think is absolutely amazing, he made up a new word because he didn't like the word being used) he now calls is an omnium(meaing of all)
\par An even more outlandish theory is many-minds... which is exaclty what it sounds like... yeah
\par The Pilot wave interpretation is that there are hidden variables; that just as things like coin flips, brownain motion, plasma instabilities, and so much more seem random from the outside, if you really break them down they aren't. Now what are these hidden variables. Now what is this hidden variable, you might ask. Well lets take a step back and explain a bit deeper. First, lets define a quantum wave function, this is different to the excitations of quantum fields mentioned earlier. The wave-function is a complex variable(I mean complex as in in the complex field not challenging to understand) that depends on the particles position and can be used for several things. It can de described:
$$
i \hbar \frac{\partial \psi(r,t)}{\partial t}=(-\frac{\hbar ^2}{2m}\nabla ^2 +V(r))\psi (r,t)
$$
\par where $\hbar$ is obviously the reduced Plank constant. r is the position vector. $\nabla ^2$ is the Laplacian operator. m is mass. v is the potential energy. t is time.
\par Once you have the wave function you can do several things with it, square it for probability distribution, collapse the wave function, etc. But I am getting away from myself, you want philosophy so I will get back on that.
\par Basically, pilot wave theory suggest that this wave function influences this particle more than we think, and thus can lead to the particle's behaviors being deterministic through equations like this for velocity, $v$:
$$
v=\frac{j}{\psi}
$$
\par Now, in all reality the equation is obviously bigger; must expand $j$ and $\psi$, but I didn't want to write it out(also there are expanded versions for relativistic and whatnot.) though back to what we had in mind, what did we have in mind, oh yeah super-determinism. Basically, quantum mechanics is deterministically not probabilistic.
\par One final thing I will say about pilot wave theory, is that some people suggest that the many worlds interpretation and pilot wave theory are one in the same just describing from different points.
\par Next, we have environmental decoherence theory. Sadly, I know very little about this one at the moment, I will attempt to learn more to better understand it better(for its own sake and given that it could convince me) But it is basically that like how a coin isn't truly random, neither is the quantum world. That there are density matrices made by interactions that effect future actions leading to a theoretically deterministic world.At the moment I don't really understand the mechanisms so I can't prove, disprove, critique, or criticize.
\par Lastly, Objective collapse theory. The objective collapse theory is similar to Copenhagen, with it's more literal view of quantum mechanics. Though, it has two differences. It has a true literal and ontological belief that goes further into everything within it. Also, it does not rely upon the observer. It relies upon 'random' collapses of wave-functions, such as gravity, scholastic field, or the interactions of other particles.
\par This is all well and good, and I do truly suggest you try to figure it out yourself(even though you don't exist), but this is a book MY philosophy. So what do I believe you might ask. I find the objective collapse theory to be best with the pilot function being a close second. Now why do I believe this?
\par First, for reasons I will explain later; the idea of non-determinism is important to other theories of mine and thus I need it for them to function. Though I am sure you all don't think that is enough, so I will go further. Second, in all reality given we cannot empirically prove any of this it doesn't really matter. Three, given that it doesn't really matter, objective collapse theory makes the most intuitive and philosophical sense to me intuitively. Four, you know what, I give up(for know) these justifications are kind of weak, even in my eyes. I plan to do more research and come up with further justifications. Right now it is basically which sounds right.
\section{*Other Questions in the MetaPhysics of Quantum Mechanics}
\par ...(will update later)
\section{Quantum Consciousness,  \\ Arminianism, Super-Gödel \\ thinking, OR Orch, why I even decided to write this book, and how long can I make the title for a section within a chapter}
\par As much as I love how hilarious the title is; with its length, random buzz word, and joke on the end. I also kind of hate it; quantum consciousness seems like some esoteric nonsense. It feels like I am admitting to believing in a specific theory just to justify my religious beliefs, but this chapter will attempt to prove that that isn't the case. That it is rigorous, while maybe not on a true empirical scale, at least on a formal logic one.
\par While there are many reasons for the creation of this book, the main ones is to evolve this very idea. It was that I had this idea at one point and it challenged a lot of previous thoughts, thus I wanted to evaluate my philosophical thoughts from the ground up, then apply that.(While it was the original reason, I plan to continue this book much further and continue it.
\par Let me first explain what my theory is in broad strokes. It is the belief(more so a belief than a full theory given its unprovable nature, unscientific background, and basis on the belief in an omni-potent God unconstrained by the confines of physics, formal logic, or even abstract-mathematics.) It is that our consciousness and free will is preserved by God through the use of quantum mechanics. This theory in it-of-itself in not groundbreaking, original, or even scientific in nature, but I intend to go about explaining some possible mechanisms of this(though sadly, by definition this theory will be unprovable, which I will explain why in a second.)
\par First, is there any justification for this idea of quantum consciousness? There is some, it is weak but existent nonetheless. It is related to Gödel's incompleteness theorem. Gödel incompleteness theorem has two part. First, "For any consistent formal system F that is capable of expressing elementary arithmetic, there exists a statement G in the language of F such that if F is consistent, then G is true, but G is not provable within F." The second is, "For any consistent formal system F that is capable of expressing elementary arithmetic, the consistency of F cannot be proven within F." This claims that formal logic has its own limits; you cannot prove an axiom with itself and you cannot prove a derivative with an axiom alone, you must prove such axioms with other axioms(because you cannot prove an axiom with itself.) Now what does this have to do with free will, well to have free will you must first have some form of consciousness, something outside of our ideas of formal logic. We have some proof of this, the fact that we are able to derive pure math at all seems to prove that we are not algorithmic based, something beyond. Now there are a lot of arguments against this, namely that our ability to do mathematics comes from our informal thinking rather than beyond-formal logic thinking. In all reality, I like this explanation better, but I place Penrose's argument of consciousness here as a sample of somewhat scientific rigure rather than simply religious thought.
\par You may now be asking how this super-Gödel thinking proves free will. It doesn't, but it implies as deeper idea of consciousness allows us to theoretically side-step determinism in order to 'allow' free will. Now, before we get into how God can protect free will, and why I think he would, let us first explain some possible mechanics of explaining this seemingly unexplainable phenomena. Though before that, let me say the fact that by definition it goes beyond Gödel's theorems, we by definition can't prove it so take all of this with a grain  of salt. It is an idea, not a law, not even a true theory, and idea.
\par Let us first go through the most popular explanation, the reason I even came up with this idea. Sir Roger Penrose's OR Orch(I seem to be mentioning him a lot) Essentially it is that microtubals, a potion of the brains neurons, and theoretically a part of consciousness, are highly subjective to quantum 'strangeness' in a way that most other complex structures aren't. That the interactions of quantum mechanics can somehow safe guard conciseness and allow super-Gödel thinks and even possibly free will.
\par While this idea is specifically for objective wave collapse theory, it also works for things like environmental decoherence and kind of work for many-worlds, but not for Copenhagen and pilot-wave.
\par For Copenhagen, no extra theories are required, by definition the idea of consciousness in 'beyond science' and thus no extra loop-holes are needed.(they don't specify what an observer is so therefore other explanations are possible, but they hold no further ground than this one.)
\par Now how does quantum 'brains' lead to free will. It doesn't, not in the most literal meaning, but it does give us extra wiggle room. While most modern views see our view of determinism as falsifying free will, but these theories give us room to say that there are other possible answers; dualism, God's choice, beyond material, and others.
\par I would say that God, by some manner, protects free-will. Whether it through some hidden ideas of quantum mechanics or it is that our consciousness is beyond the material and effects the material through the quantum world. I don't know, as I mentioned earlier, I am not even confident it is correct. It is an idea.
\par Now you may or may not be asking, why would God use quantum mechanics to protect consciousness, if by definition he is beyond formal logic and thus wouldn't need such things. While I cannot comprehend the machinations of God, I would suggest that a possible explanation would be to create our logically, self-fulfilling world that we are able to comprehend(for the most part) and live in a true free will.
\par Now to end, I want to reiterate how theoretical and disconnected from regular science this is, most of my ideas are at least respected by a decent portion of the scientific community, this is not. It is an idea, nothing more. This is an exercise to see if it holds weight, it currently does not seem to.
\section{Compatibilist Free Will}
\par Even though I wrote a whole lot about quantum libertarian free will. I don't necessarily believe in that. I view the world as either semi or super-deterministically and that that doesn't actually effect free will. Though I still wanted to examine the idea in more detail nonetheless.
\section{*Super-determinism, semi-determinism, Retro-causality, and Chaos theory}
\section{Role of God}
\par I have already gone over the role of God in many ways; creator of the universe, its laws of nature, protecting free will, but now I wil go further.
\par The first question is the relation of miracles, in the Christian faith, we believe that our God is a personal God that creates miracles for us. Now how does he do that if he predetermined the laws of nature?
\par Well that is fairly simple, God is omnipresent, meaning he not only knows what you are going to pray for before he created the universe, but he also knows how to create such miracles from the creation of the universe. Thus he 'created' the miracles when he created the universe itself; with knowledge of all that will ever transpire.
\par How does the divine foreshadow effect our idea of free will. Well, God is beyond our free will, he is beyond whatever quantum mechanisms protect our free will, so he can know without interfering with our free will.
\par Is something good because God says it is, or because it is objectively. It is because it God says it is, beyond God there is not mechanism for the creation of truly objective morals, there are logical models based upon subjective values, but they cannot stand upon themselves. God makes them right because he is beyond formal logic.
\section{Post-Axiomatic Nature of God}
\par I guess pre, would be the better prefix given what I am about to say, but Post-axiomatic sounds better. Now back to the actual concept.
\par This idea is simple. That our axioms, as later examined are derived by God rather than God being subservient to the nature of these axiomatic principles, "A line is the shortest distance between two points," "Existence exist," "consciousness in its totality," "A is A," and many more. He, God, derives them. Now this twists our mind given that axioms by definition cannot be imagined without their existence. It is like inventing a new dimension for us to see, it doesn't exist.
\par This concept enforces many ideas already present; God omni-(something that cannot exist with current axioms), God being beyond comprehension, that God is self defining(see non-infinite definition of words and axiomatic semantics), and being outside of causality, outside of formal logic. He creates existence(something that makes no sense in our current axiomatic understanding of 'existence'), he is three in one(against A is A), and much more.
\par While they don't 'solve' these questions in the traditional sense, they open the door in a more esoteric sense. While by definition we can never understand these concepts intuitively they do give rise to much of our understanding of God.
\par On to our esoteric identification(I know must of what I write is intended to be rational and examined this is one of the exceptions.) This idea of pre/post-axiomatic identification leads to our idea of faith. God, isn't just beyond us, he is ineffable and unprovable though our axiomatic thought processes. Beyond that, this further goes into my idea of the "non-material." We can further define this as that of which is based upon our known axioms and that of which is not.
\section{Metaphysics of cosmology}
\par While I could go on and on about physics of things like the big bang, entropy, arrow of time, anthropic principle and how they could theoretically effect our views of metaphysics. I don't need to, my belief in God nullifies such arguments. That all possible incites have already been satisfied by God and thus have little need for exploration.
\section{Ontology of Information and Entropy}
\par Once again, we must first define what our concepts are rigorously.
\par A classical perspective. Within this concept information is the physical quantities that en essence give you traditional information. That allow you to decode past events and even future ones. Entropy is the inverse of this affect. Geometric instabilities and 'randomness' that hide 'information' from observers.
\par From a quantum perspective; quantum information even includes superpositions, wavefunctions, density matrices, and more. The quantum bits, qubits, can exist and superpositions and exist within the same quantum state(position)
\par In this view, entropy then becomes a byproduct of entanglement, a process that 'hides' the information from view.
\par There is much more on the semantics of this. Going into Geometric identities, black hole information paradox, and quantum thermodynamics. Though, for now I will move on to the more metaphysical parts.
\par One common debate is if information is truly fundamental or simply a human made connection. That the universe has a 'computational structure' where information is the substrate and entropy the measure of the complexity. With modern understanding of the geometric identities and the fundamentalness of information, the theory of fundamentalist seems to take the lead.
\par Next is the arrow of time. The second law of thermodynamics/entropy states that in isolation, states drive towards entropy. This necessitates an arrow of time. It breaks the symmetry of time.
\par Now many question whether or not entropy is a by-product of the arrow of time or the other way around.
\par The way I see it, the direction of entropy is the consequence of the already time symmetry breaking existence of dark energy, black holes, and other similar objects in this preview. That the direction towards entropy precludes low-entropy existancess.
\par Though this is just one theory. Holographic principle, Gravitational Entropy, Quantum Mechanics and Unitary Evolution, and more. In-fact the quantum mechanics and unitary evolution makes more sense and less problems.
\par The symmetric theory ignores that some forms of entropy are consequences of statistical physics that doesn't deny time symmetry
\section{Dualism}
\par I have already mentioned this in interpretations of quantum mechanics, but in physics many possible mathematical(and thus metaphysical) interpretation exist. Different ways to look at the world. They can't both be true, can they?
\subsection{*Phase-space and the Euclidean Mind}
\section{*Ontology of Numbers}
\section{Reality and the Abstract}
\par When reading this chapter, one can easily get a esoteric/gnostic sense of reality. Of pure abstraction with little regards to reality. Though this is not the case. In fact I find the abstract "non-material/pre-axiomatic" world as inconsequential. The only true consequence being God, but he matter because he influences the material. The focus on this abstraction being the fact that the ideas of reality are largely explored and are easily understood and so large amounts of writing and exploration on the topic is relativity inconsequential.
\par The focus on abstract, structuralism, post-axiomatic is simply to satisfy my need for cognition(as later explored) rather than being core parts of who I am.  These theoretical ideas that I am not even sure if they are right or not(post-axiomatic God, Consciousness, etc) are explored not because they are core to who I am and what I believe(maybe the God one) but because it is fun.
