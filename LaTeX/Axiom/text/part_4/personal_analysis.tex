
\chapter{Personal-Analysis}
"The unexamined life is not worth living" | Socrates
\section{Introduction}
I will use this journaling of sorts to examine myself and my own personality.
\section{Disgrace and Pride}
(if anyone else reads this, don't pay too much mind. I use words differently than their actual meaning. I don't word things very well. I was simply trying to capture my own indescribable and esoteric and possibly failable thoughts in this moment on this topic, my actual feelings are very different than what can be interpreted through the word choice.)(Additional update: upon later analysis, I have found that these ideas of pride and disgrace are for the most part the idea of seeing my earlier mentioned "personal values" in a light very similar to traditional morality. This doesn't fully explain it and I will continue to work on refining these ideas beyond the esoteric concentration currently presented.)
\par Why is it that you are so accustomed to the use of disgrace in your speech and why does it effect you in such a manner. The idea of 'disgracing' yourself is so vital to your worldview, it effects everything; morals, interactions with other, and so much more. It isn't even like you are all that effected by the thoughts of others. In fact the most abnormal and extreme feelings of self-disgust and disgrace are related to a refection of the thoughts of other. Why is it that I feel like to see other, value their opinions, and let them influence men to disgrace myself. It is an odd thing, yet it is so very natural. It influences everything, and it is so much more extreme when I am in isolation as I am now. What is it about the concept of being influenced by others to be so repulsive, so disgusting that I won't just let it influence my actions but I will try to persuade others my way is best with pride. With this eternal pride that locks me in, says that I am always all of the way in all things. There are so many things to think about in this discussion so I will try to take it piece by piece and hopefully add some things.(also, what it with the strange switching from dialogue with an external, internal, and this explanation style?)"
\par Well, disgust is a natural motivating force. It motivates far better than most, it is consuming. It involves fear, pride, righteous anger, annoyance, and so many more extreme and strong emotions. It motives far more than most, and it comes so easily, especially for someone like me. There is no greater fear than falling in ones own eyes. My disgust of others is just who I am, it is what makes me who I am. It disconnects me from the petty emotions around me. I mean, can people really say that those are better; to be insecure, jealous, vapid, empty, etc? Is it really better? Take insecurity, it is such a strong emotion that consumes people so very easily, making them bicker and fight, making them claw at each other with their broken paws that do very little besides hurt themselves and hurt others by their own volition of nonsensical blight. My 'pride' can rise me above that. If I do not see them as anything, then for what reason do I have to be insecure about this vapid collection of dust. Now I am not truly some raging narcissist who sees no one beyond and object for my own use. I see people as they are, I care for others, value them. I just simply do not value them in the way others prescribe them. I don't see them as threats, competition, just broken monkeys in need of assistance(and that I am the same way and can occasionally use assistance myself, just less than most.) I still see them, I make friends just as easy, if not easier than most due to this.
\par This disgust is a strengthening endeavor that assistance me in many ways. My self-disgust pushes me further. It is what lead my to teach myself calculus in elementary, and has brought me here know with graduate level knowledge, research, and so much more. It has disconnected my from the vapid desires of others in their looks and whatever other nonsense.(though I will add, this disgust is not some obscene emotional self-distrain but rather a more abstract understanding that doesn't truly make me feel bad, just concous of 'wrong doing.' More so like a passion against, I feel no negative emotions about myself)
\par Though I must acknowledge that this does not make me better, it makes me better within my own eyes, my own values, not others. My values have no greater truth than that they are my own. So my disgust should always remain abstract, looking down at the ideas not the people. I have done great at this, never prescribing habits or concepts to people. Though as of late I have had trouble with this, seeing people beyond the moment. Being conscious of their deniers and thoughts, it can create some discontent. I should remain as I was, seeing people only in the instant. Nothing more. I am the only conscious being within the confines of my own mind, for it is better this way. You cannot accuse a rock of moral failure, only a man. For which I am the only man that I see, the only man that I know, I am to constituent for the moral blame. The disgust should be within me and me alone seeing only me. For I am the only one held to my values and the only one who could suffer the breaking of them. For they are my values alone, that is the purpose of the disconnect. Why my values must be mine alone, for I am the only one to be connected to them, the only man in my eyes.
\par For this I also hold the burden of thought and reason. The thought behind my morals, my values, my beliefs, my religion, my reality. I see the world as my land for the conquest of my own knowledge to build up it all. That is the whole purpose of this book, to fully disconnect, to find myself fully and fully alone. Because, in the end that is all that matters in the abstract. For sure, I love my family, my friends, and my fellow man, but I love myself in the way that one can only love oneself, the expectation of my own measurement, the pride in my achievements, the disgust in my faults, the understanding of my beliefs. For one cannot love others without loving oneself, for love is derived from ones understanding of their own values, who they are in all of reality.
\par Now why are these feelings and thoughts all the more present in isolation.For when with others, those are my real feelings. In isolation I attempt to derive and find, deriving and finding the strange and unwieldy emotions of the mind does not come truly and with the same accuracy of that of physics. It comes in strange botches of thought that don't mean what they literally mean but can be described through thoughtful examination of the words, other words, and the actions of the person.
\par When did I become this, so thoughtful behind it all. Seeing the world beyond the material, seeing my thoughts beyond the exact. Maybe I really am chaining, in a way that must lead to the changing of even my most basic assumptions.(this last paragraph is really stupid)






\section{On Ayn Rand}
\par It would be obvious to admit that this section would be an analysis and critique of Ayn Rand's ideas, but like the more erudite among you will notice that this would go against the structure of this chapter. This is exactly correct, this will rather be an analysis about my propensity towards Ayn Rand's ideas.
\par While this may seem useless and without purpose, it isn't. It has been a strange psychological question about my enjoyment of Ayn Rand's novels even though I starkly disagree with actual philosophy and the fact that her books lack the many of the general characteristics that generally allow me to enjoy such novels. So what is it?
\par There are several reasons, though I will start with the most fundamental. Simply put, I see myself within the characters. Most particularly Howard Roark. Many of the descriptions of his won emotions and others description of me mirror is a strange way. While not absolutely similar, it builds off in a way more closely connected than any other fictional character. How he exists as an independent entity, not noticing others but living by their moral code not out of other means but as its own mean. Because of integrity above all. The way he is an act of moral striving rather than a disgusting abstraction for those even more disgusting to connect to. He lacks those pathetic neurotic tendencies that those around me let control and give authority over them.
\par Beyond that, he gives me ways to articulate my own personal feelings in a way that I have never seen before. Being "to proud to boast" by seeing both the criticism and complements of the world to be equally insignificant because it does not come from myself. To find pride, value, truth, within myself not within the pathetic world beneath me. Seeing another person see how pathetic the social validation games are, not is disdain for others but as seeing it beneath myself.
\par The way happiness is the his natural order rather than some far flung ideal that is beyond, that negative emotions seem dulled by their pathetic attempt. That he is truly happy at all times, just as I am.
\par The way compromise, even in the slightest way seems to be evil. That this concept has been a driving force in my life thoroughly. Because it is a self-betrayal, a betrayal that can't even be thought about. An evil beyond belief and idea. Even a white lie, a broken ideal without real backing, a principle made as a child, and so much more seem evil and I can't figure out how others live with it. Though these compromises are never entertained long enough to feel anything beyond the knowledge of it.
\par The way others see him as cold and arrogant despite him clearly not being, due to there misunderstanding of him. Because he is independent, because he doesn't care about friends and what insignificant interactions they had, what their friend's did to annoy them. That they don't care about philosophy, physics, mathematics. 
\par How he is both happy in isolation and with others. Equally, because the existence of others doesn't have that effect on him.
\par The way his creative and logical thoughts of Philosophy, physics, mathematics, and other intellectual topics are all that matter, well beyond the existence so often people confine themselves to.
\par Finally, he lacks all neurotic tendencies. No desire for complements, praise, people to soften their words, bend down. These neurotic tendencies are found everywhere, in everything. People, fictional characters, and others. Though I feel such thoughts so rarely. It is a impeded idea in almost all of fictional by its own virtue, but I never see it within my own mind. He like me is truly beyond these petty neuroticism, no capacity for them at all. No vulnerabilities, no anxieties, no insecurities, no of it.
\section{On the Pursuit of Thought}
\par NFC, or Need for Cognition is a psychological concept seen clearly in this book. Though it many not be obvious the the extreme extent it is true.
\par Much of my ideas of philosophy, ethics, physics, and more may seem like a desire to know reality in its greatest extent(and this is certainly true), but in its most basic and primate way, it is my need to think.
\par I love thinking, it is my favorite thing. This is what drive me, my desire to satisfy that part of me. I think about anything complex enough; philosophy, ethics, physics, mathematic, coding, economics, political philosophy, literary analysis, world building, international relations theory, geopolitics, formal analysis, psychology, meta-cognition, and so much more. The more thinking required the better.
\par In fact, I love it the most when it is beyond me. When it takes me weeks to not even fully understand what questions to ask, when it feels beyond my comprehension, when I think for days and go no-where. I love it, the scavenger to knowledge, then to actually get it. For it to all fall into place just as it should.
\par Now very little does this, in fact most things just come. They are understood intuitively. Seem to basic to even consider. Even some of my other hobbies like psychology, politics, and such seem to basic, and most of the other things beyond my intellectual hobbies seem so basic as to not even give it time at all.
\par Another funny consequence, is despite my almost compulsion to efficiency, I still am drawn to complexity. Despite my normally physicalistic and literal tendencies I am also drawn to the abstract. I have found this most easily seen in my obsession with using higher order math. I use tensor calculus, hamiltonain mechanics, geometric identities, when similar methods can do it. I obsess over these ideas when there are more practical matters. 
\section{Nietzsche's Sovereign Man and Morality}
\par Upon further reflection, my chapter on disgrace and pride has much connection with Nietzsche's "sovereign man." Here I will first explain what that is, its connection, and divergence. 
\par Before I begin, one thing with Nietzsche's writing is that it is up to controversial interpretation. Some say the sovereign man is sincere, other ironic, others meta-ironic, others see it simply as a literary device, and there are still other interpretations out there. Luckily though, for the purposes of this exercise it doesn't matter Nietzsche's intentions, only my intentions and ideas. For instance, Nietzsche's reasoning for presenting the sovereign individual differ from mine; which I will present later.
\par Now to actually begin with the analysis. Who is the sovereign individual? The sovereign individual is the archetype of Nietzsche's will to power(the Ubermensch later takes its place as a further extreme but I care little for this idea.) It is an implied ideal in which a person creates mastery over their own life. To live their life in accordance with their values, that they themselves had created rather than inherited. They do this with their mastery over their own impulses; a mastery so extreme that they eventually shape their impulses into however they consciously wish. Further more, these values and 'morals' are created though anesthetic ideas, creativity, future-bound, and affirmation to the power of life rather than traditional fear. Finally another key part in is surprisingly forgiveness, though not in the traditional sense. Rather than forgiveness caused by God or some other values it is a combination of the acknowledgment of the fact that most people are incapable of true moral thought, self-mastery, and to let go of their resentments and impulses; another key idea is simply the fact that not forgiving hurts the sovereign individual, Nietzsche's suggests that it is better to simply forget about the trespasses rather than hold on to resentment and call it responsibility to forgive when it is hidden resentment. Essentially the sovereign individual is the archetype of the promise made flesh(or word)
\par Now how does this relate? Well before I get toe pride and disgrace, lets explore the philotimic virtue. The philotimic virtue highly mirrors the concept of creating your own values, setting your life and soul in pursuit of them, and finally holding yourself to them in a way that can be seen as fanatic to the outside. Also, creating these values individual of the existence of others(though the philotimic virtue includes the will and thought of God to yet be higher than the will to power of myself.) To live your life as you will it so. T
\par Now disgrace, the idea that my personal values that I have willed hold the same good and evil. That to betray them is evil. But both ideas have a similar them, by not hiding from it but rather carrying the burden is guilt erased. By taking the conscious action of purifying oneself one becomes worthy of forgiveness(not God's forgiveness but my personal one) and because I have achieved this, while I carry momentary understandings of mistakes unlike the sovereign man, I hold no returning 'guilt' or resentment of any kind. No neurotic tendencies, no projections onto other, none.
\par Pride, this as mentioned earlier is the same as the earlier mentioned philotimic virtue. The idea of creation and happiness as a norm that has been willed by me as an action of my pursuit of virtue and creation. This is where the disgrace truly comes in, it is reabsorbed, not as pain but as energy for the recreation of the self in the form of higher values.
\par Another key idea is that of forgiveness. Once again there is the connection between my idea and his. The fact that for the self; moral, ethical, value-driven, etc is not a same thing. It is a rapture of the very existence, but rather than waste energy on being sorry for oneself, one must use that energy on self-perfection. Though, forgiveness of others is another story. Other people are not capable of moral action, they are slaves to impulses, temperament, social constraints and much more. While people claim to be moral, much of there actions disgrace their ideals, their intentions align with physiological temperaments rather than values, and they abandon so much so easily. Their transgressions should be forgotten, or better yet not noticed at all.
\par Though, I am sure you are thinking about the differences(I mean I have been sprinkling them about these section) and while they are important to some extent. I don't want to go though a point by point refutation of Nietzsche ideas, especially when this analogy of the sovereign man is meant more an a psychological analogy to help explain intuitively my own personality rather than philosophical affirmation.(I disagree with Nietzsche on much)
\par One thing I will say though is the difference in end goal. For me, the sovereign man in a more refined sense(axiomatically driven, recursively made, God fearing, etc) is the final form. The Ubermensch is useless. The chaotic form of disparaging logic, reason, and God in forming morality disbanded the entire project and is just as disgusting as the slave morality of others.
\section{Half-Growth and Half-Death}
\par While many proclaim that my ideas of seeing people not as thinking being, not as moral agents is misanthropic and anti-social. That I should have more respect for others, that this lack of respect can lead to later personal problems with others. Though I disagree, first I will go over how I have seen others interact that makes me disagree, and my own personal evidence from my falterations with these values.(one thing I will add, is once again these are not true literal beliefs but analogies to explore esoteric identities.)
\par Think of how people treat children, then people they deem as equals, then themselves, and then those they deem as higher than them when they perceive a slight. 
\par The child's slight is either ignored or the person takes conscious action to help them, not by expecting moral action as they would with others but by understanding the child's abilities and inabilities and working around them in a kind manner.(Bar extreme causes or grotesque individuals)
\par With others there exist a range of reactions to slights, but their is a clear differentiator. They expect others that are 'equals' to exist in a semi-moral manner(this is because most people hardly understand moral manner to begin with). They try to act in kindness and understand others faults and temperaments but when push comes to shove they resent others, they expect from others. This can come in a wide variety of ranges: resentment, anger, fear, annoyance, instability, and so much more. These reactions are almost always unproductive and hurt relationships, people and such. They also use the perceived moral responsibility to negate their own moral responsibility in many cases(under my analysis, this is where most pain received by the affliction of others comes from, either consciously or unconsciously)
\par Then from here it is easy to see that the extremes of these go up and up. 
\par These are where a majority of the petty, vapid, and pathetic emotions I wrote about earlier come from.
\par Now what I suggest isn't as radical as it seems, it is just shifting the average person closer to where others see children, no one other than God as above me. That I intentionally analysis and 'handle' other people. I understand them both to better see them as pathetic not in a disgusting way but like a child. Also, to better handle them for interactions with them.

\par Though, one thing I didn't include in my earlier writing is that some people do come closer. Select people I know that I am close to enjoy(or don't depending on how you look at it) the responsibility of a human and conscious life within my own ideas.
\par Now, where does my growth and death come from? Well in recent time I have faltered, I have subconsciously found that I do expect things from others. These is so terrible in my eyes fro several reasons. One, as I have mentioned earlier, this 'equality' is the source of much of our troubles in the modern age void of true horrors. Second, for my personality and moral theory this effects me in a much more extreme sense than it does other. From a shallow perspective I am an extreme puritanical person who is obviously very prone to disdain. Beyond that I hold some values in high regards that many don't hold at all. Though, more truly than that, my in my moral theory intentions mean everything and so very few people have truly pure intentions. Now I don't mean everyone is a selfish jerk(though many are) but beyond that many are husks of people that only follow moral theory because they don't comprehend any other way, others hedonists that find minor moral action as easier, others do it because of insecurities, and so on and so forth. The concept of doing things for the mere fact they are moral/logical is lost on almost every person you will ever meet. 
\par Here I will give an illustration. Last week(as of writing this section) I had an interaction. Someone had very condescendingly given me 'advice.' Telling me in a clearly condescending and antagonistic tone to remember to unroll my selves I had rolled up to wash my hands.(this is also after of many other similar interactions with this person) Now because I knew that to explain the ethical ideas of attempting to 'assert' fake authority on another in such efforts was immoral due to the fact that on a psychological basis that such things could very easily be comprehended as attacks on their own self-determination and competency. Also, that such actions could easily be interpreted(and likely truly) as things like need for superiority, low self-esteem, extreme lack of social awareness, control issues, projection of their own incompetence, desire for a reaction, or malignant/grandiose/competitive narcissism. Though I instead let it slide and said thank you, I didn't feel it would be worth it or would give any results on the matter. Now I didn't expect them to realize their fault or anything of that extreme. I expected them to say your welcome either as a empty social connection or an understanding of the fact that I understood their game and they would either give up or hopefully be filled with self-disgust over the conscious acknowledgment of why they did what they did. Instead they didn't make eye contact and instead "hhmf" at me. I still had the rational ability to understand that this person was either too stupid or immoral to understand whatever I wanted to say on the matter, so I decided not to. Though regardless I was filled with disdain. I was disgusted by such extreme evil. I know for many this doesn't seem like evil, but in the eyes of intentions it can only be logically assumed as. Those of lacking ability of intelligence to react in moral manners wouldn't make such an extreme mistake, those powered by resentment or anger would likely be filled with shame of such actions, and so on and so forth. Likely the only remaining interpretations would be that of malignant/competitive narcissism, desire for the reaction/pain of others, and other similar grotesque ideas. Now while I see the earlier stated ideas as immoral, these take special places, especially for someone like her who has the capacity of living in dignity. 
\par Now this isn't some one shot, lately I have been more and more likely to to feel disdain for others. I don't know if it is puberty, social connection, increases in closeness with others, or whatever. All I know is that while this is something that so many other have proclaimed as what would be the greatest thing to my self-perfection is rather the worst thing.
\section{Lack of Aesthetic}
\par When I am referring to aesthetics in this section, I do not mean philosophical aesthetics, as in valuing something over another; I mean traditional ideas of beauty in music, art, and natural affairs.
\par In this sense of the word, I lack almost all aesthetic values. I have no favorite color, no concept on beauty in most things, and no care for musics, visual art, or other 'artistic' concepts.
\par This doesn't mean I have none. I have some interest in the artistic understanding in complex and intriguing literature, I can  also be temporary incapsed by complex art/music(though only as long as it takes for me to understand it, and if it is too abstract as to be useless I have little or no care for it); beyond that I can find beauty in mathematics, logic, efficiency, ideas, plans.
\par In relation with that I have no feelings of sentimentality, meek emotions, and other similar emotions.
\par Now, I have always naturally found all of these things as beneath me and childish(and still do to some extent) though I have been taking time to try to expand my horizons. 
\par As I have been doing this, I have noticed some changes. I have thought in more lofty ways than before, been more open to some experiences, and even engage with emotions in a way previously unheard of. Now here I will leave with this, but later I will explore this in more detail. What is actually causing this change, Explore what the change actually is rigorously, and whether is it actually a good or bad thing?
\section{Boredom}
\par Just as NFC is a primary driver in my life, so is aversion to boredom. 
\par The most obvious connection is simply for my NFC is grown and in a feedback loop with my aversion to boredom. I satisfy my boredom with challenging cognitive tasks. This is one of if not the largest driver in my self-education, research projects, writing this book, and so much more are driven by aversion to boredom. 
\par A little beyond that is ambition. Just as I choice challenging cognitive tasks to satisfy my boredom, so do I choice other challenging tasks. Leading, planning, mentorship, responsibility, work, physical activity, and more. This is what pushes me in SVA, Slack, JROTC, Scouts, OA, and much more. These activities satisfy my boredom.
\par I will go even further and say this desire for challenge is primary from aversion to boredom, not fear, perfectionism, and other traditional psychological reasons for people pushing themselves further than most.
\par Now, my aversion to boredom isn't just in ambition and other productive means. The most obvious is in watching TV or non-educational videos. Though this is largely to simple to take any time analyzing.
\par Though there are other things to analysis, relations. First I will go over casual relations. 
\par For some reason, as I will discuss in more detail later on, I switch between 'extroversion' and 'introversion' in a sense. When I am away from others I hold no desire to be with them, I satisfy my boredom through abstract thought, arguments with myself, physics, and other similar things. That this feels the most natural thing. Though when I am with others this switches. Getting lost in thought no longer feels natural but rather challenging and hard to focus in the same way. I naturally feel the way to satisfy my boredom is through conversation with others. It is an interesting development that I will likely look into further.


\section{Why Physics}
\par While I have many interest; math, coding, economics, psychology, philosophy, logic, and many more. Though, one interest stands above and beyond the others. Physics. Ever since I was a kid I have been obsessed with physics.
\par Now I am sure I can come up with some lofty reason on how physics is the nature of the universe in its purest form, or the most fundamental science of them all. While I am sure that it is part of the reason I like physics so much, it is not one of the primary.
\par The most easily observed is its balance of abstract thinking and practicality. It is one of the most abstract and purest forms of logic other than pure maths and some forms of philosophy, but unlike the others it has a more direct connection to real-world application. For instance, my work is working towards fusion reactors, something I think once working will be one of the most influential and greatest creations of the modern age. To combine both reduction to axioms and analytics and contructionism of creating something practical.
\par Next is the challenge. Not only is physics itself extremely challenging, with some aspects of it taking months of studying to even comprehend it, it also combines many other hard disciplines; math, coding, engineering, and even metaphysics as times. This challenge is exhilarating and as mentioned elsewhere is one of the things that i most desire.
\par On combining other things, physics is a multi-disciplinary subject. Combining and using many of my other hobbies.

\par There are many more reasons but these are the main ones.
\section{Lack of Resistance}
\par As developed throughout this book, especially in "On the Pursuit of Thought" the idea of challenge as a goal. 
\par The joy of an intellectual challenge specifically. To mull over a topic for hours. To have something beyond current comprehension. Something that doesn't make sense. Then all of a sudden explodes, not only to explain itself, but making connections all over. Like a flood. 
\par This is my greatest joy, what I live for. Though, even as a teenager I am coming to limits. I will examine this in parts, first through my self-education and then through external world interactions.
\par What is going on personally is I fear that I won't feel this feeling. Many topics like economics, psychology, geopolitics, history, philosophy, literature, and more don't give this rush anymore.
\par As an example I will examine literature. In the past year I have found interest in literature, a topic I had previously missed. It was enticing, learning about the usages of symbolic characters(and figuring out who was and what was), setting as a character, philosophy through stories, and so much more. I had that excitement, though lately I don't. While there is still more to learn, it isn't the same. Everything new, is obvious. No requirements of complex thought when the workings are already there, only new information to be feed into the pre-made algorithm.
\par Now, I still have some exiting things in the above fields and much, much more in abstract mathematics and physics, but I am only 16 at the moment. If I am already this far along now, what is to say in 10 years there will be much left, what about 40? I have already surpassed all traditional classes and all that is left is research articles and manuscripts, these will leave me for quite some time, but who knows how long that will be.
\par This also continues in my personal life, school has the intellectual engagement of watching paint dry, talking to people is a burden, leadership roles still have excitement but have less and less return,
\section{Am I Misanthropic?*}
\par My misanthropic like tendencies are clear throughout this book, but am I really misanthropic? No, well... maybe a little.
\par I do have many friends, friends that I enjoy. I like talking to people. I am truly an extrovert in nature. 
\par Though, beneath that there is a disconnect. I don't truly enjoy talking but rather staving off boredom. While in isolation I generate distrust and disgust of others. Many people I dislike more than I dislike boredom and everyday the percentage of people within that camp grows. 
\par 
\section{Depth of Feeling}
\par All of this discussion on emotions and psychology can leave a reader with the thoughts of a sensitive and emotional boy, but this is far from the case. 
\par While I use extreme words to convey my ideas, the extremity is not there. All my life my feelings have been dulled. While others are overwhelmed by feelings, for me they are hard to observe. Like they don't fully touch me. 
\par This isn't true for all emotions; my passion, excitement, fanaticism, and need for cognition are all real. This doesn't take away from my thesis. 
\par Beyond these, I have always required conscious effort to hold on to emotions. While this may sound like lamentations, it is not. It is a great thing to forget anger and pain because of there insignificance. To never feel overwhelmed or anxious, to have control over my thoughts and actions, I am in control. I love that I am this way. 
\par There was once a time that this made me feel inhuman. While others talked about there emotions, though that feeling like all has passed. Though, don't take this too far, I am by no means some robot devoid of all emotions, just simply that for any population with variance some people will be above or below average at things. For me I am simply far enough that observing others like me is a rare enough occurrence to think it truly is rare(but it really isn't.)
\section{Language Usage*}
\par Before I actually begin upon the main idea of this section I feel the best way to introduce this is to explain the inspiration of this section.
\par The inspiration is the oddest place place possible; Ted Kaczynski, or the Unabomber.
\par I remember when I was a kid I always hated the idiom "Have your cake and eat it too." The reason for this is due to the fact that the have your cake is placed first, and this fit(in a way that doesn't fit with the idiom.), because you must have your cake to eat it. So, I started to switch it around to "Eat your cake and have it to" though never really liked that. Until I read that part of the reason the Unabomber was caught was due to his strange phrase "Eat your cake, and have it all the same." When I had read this my first thought was insult that I hadn't thought of that myself.
\par Once I thought of that I realized more connection. Some more 'normal.' Starting with a thoughtfully researched and observed effect being the fact that highly educated members of fields like physics, mathematics, logisticians, computer science, and philosophy use mathematical and logic terms in regular speech. This is because their thorough and clear definitions have clear applications. Axioms, manifolds, prior, bias(in a mathematical sense), fallacy, paradox, induction, inference, deduction abduction, ambiguity, equivocation, disjunction, mapping, inversion, duality, convergence, divergence, topology, entropy, singularity, resonance, phase transition, field, fractal, lattice, span, gradient, symmetry breaking, null, second order, and more. This isn't that odd considering that it is more wide spread, I mean a couple of those terms described above have even introduced themselves into colloquial speech(though used slightly differently.)
\par Another is using words that were created in academia, then introduced into regular speech that then changed it, in their original usage.
\par Though some are a bit stranger, mainly the fact that I simply create my own definitions for words or even create my own words.
\\
\par So lets go through these one by one.
\par The first is very simple, these words are very useful in regular speech, convey abstract ideas easily, and can be applied easily. The mere fact that I understand them makes it only logical that I use it.
\par Next, the usage of academic definitions has a similar requirement. The fact that these academic definitions are generally more completely and logically defined than generally used word. 
\par Finally, the changing of words and creation has a similar efficiency and logic.
\par Though, this brings new questions, why do I feel so comfortable with this, what criteria do I use for greater 'logic,' what is with this obsession with logic, why don't I just pick either change current words or create new ones instead of do a bit of both, and much more.
\par ...

\section{Euphesus*}
\par As I mentioned earlier, there has been change in me as of recent, and it has recently come to head with the advent of 'dating'(not actually dating the girl formally[will explain more later]{Also not completely sure it is 'coming to head' just more extreme than before}).
\par What I mean, is for the first time in a long time I feel conflicted. As strange as it may sound, I have  been complete for a while now. I never feel internal conflict, neuroticism, second guessing myself, or anything of that such. I have for a very long time clearly defined my morals and values and thus never needed to second guess myself. While I am not perfect, the few times I make mistakes, I clearly identify them after the fact and rectify them. There is no lingering guilt nor do I go back and forth between what I should have done or should do for the future. 
\par Now, in the last year or so, I have had some rare conflicts; many of whom I have talked about in earlier sections. Though with this ambiguous relationship, it has become an almost daily thing.
\par I went back and forth between whether or not I should ask her out in the first place. When I did I had a lingering regret about doing it over text rather than in-person. Later when texting her after the fact, I felt the inexplicable desire to talk to her, without any real end-desire other than the conversation. First, I felt conflicted because I have never felt the such a desire, in fact I often was disgusted by others for this desire. It always seemed beneath me, while it wasn't against my values, for someone who has always been so 'complete' and whose entire personality and temperaments are recursively defined by myself the mere fact of something new is insane upon the face of it. It is like a self-betrayal which it took me a while to figure out what the betrayal was in the first place. Furthermore, this caused me to start questioning thing, why I didn't desire to talk to my friends about nothing in particular, should I start doing it, etc. It even lead me to try it. Then this started to make me think about how some random girl I am not even formally dating, has caused me to alter my own behavior. Just slightly, but nobody alters me even slightly.
\par Once I had finally gone through all of that and moved on, I decided I would talk to her over text. During that time I neurotically tried to when would be a good time and what to say. Once again, I don't do that. I don't care what other people think of me, I don't care if the message wasn't perfect, I don't feel neurotic over anything, certainly not a person. Yet I did, I thought about it. Finally, I texted her, it went nice.
\par I decided I should keep in touch, because I wouldn't see her in person for a month. So I decided to reduce ambiguity I would simply have a schedule of when to initiate a conversation. The number of days between initiations was meaningless and I knew I just needed some arbitrary number, yet I thought about what would be an embarrassingly long amount of time.
\par Later, came the day for my next initiated conversation. and once again I spent way to long figuring out what to say, but when I did, she hadn't responded. I didn't think much of it, due to the fact there were dozens of possible explanations for the lack of communication that were fine. That was until the next day, when she responded on a group text, but continued to ignore the individual one. Now even still, hours later I am still thinking of it. Still conflicted, wondering if asking her out was the right choice, or my number, or what I said. Wondering what my next steps should be. 
\par While yes, I understand these emotions are normal for an adolescent boy of my age, but they aren't for me. Not only am I rarely conflicted about anything. I am much less about people. I am the kind of person who interacts with people for amusement or out of responsibility; deep down I don't care other people in the way required to feel these neurotic thoughts and emotions.
\par It is so strange, having the last couple years of my identity and foundation so clearly defined for everything. Only to bring out something new, something that doesn't fit. Now I don't know what to do; stay rigid or change(but change what) 
\\
\par Over the last month or so I have gotten over the visceral reaction. Now that I am back from my trip we can see each other in person, but haven't very much. 
