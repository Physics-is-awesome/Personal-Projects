chapter{Epistemology}

"Man is neither infallible nor omniscient; if he were, a discipline such as epistemology—the theory of knowledge—would not be necessary nor possible: his knowledge would be automatic, unquestionable and total." | Ayn Rand
\section{Limited Objectivism}
\par Before I begin upon explaining this, I must first specify the influence of Ayn Rand is limited to Epistemology. It has not influenced my politics nor ethics. Now lets begin with the philosophy.
\\
\\
\par In a short simplified sense, objectivistic Epistemology is the inverse of transcendental idealism(though despite what many say, not the opposite). Both philosophies take both reason thought and empirical data as useful tools in the accumulation of knowledge.
\par Though they differ in importance and direction. Transcendental idealism takes reason as the most fundamental and such abstract thought can then be applied to the outside world through empericalism. Objectivism takes the inverse, that we create our abstract models based upon our interactions with the world. Now these may seem like small differences, but the expand themselves through their logical conclusions. I could explore how they differ in more detail, but I will focus on objectivism.

\par Now, where did the limited part come from. As I have mentioned before, I am a religious person. I believe there are something beyond our sense and possible our mortal comprehension, therefore we are limited by such things when acquiring knowledge. I do believe what we are capable of discovering is 'objective' and just as metaphysically true as what we can't, but that does not take away the fact that some abstractions are beyond us.
\section{Consciousness and Reality}
\par A common continental question, is the question upon whether our consciousness truly models objective reality. While these questions can be useful when directed in a purposeful and specific way; though most people don't direct it in a useful way, but rather in a simply abstract way with little refinement of knowledge than the ability to trick others through strange and useless question.
\par Now, why do I think this is a useless question? Well, for several reasons I will address categorically. Now I will prefis this by saying this is a criticism, not a critique, I may later may later add a more refining critique but for now it is an acknowledgment of the stupidity of this, and to show that all derived concepts have no true analytical framework.
\par The most basic argument against this is the pragmatic thesis. In essence, it doesn't matter that reality is consistent with the observations of the mind, the mere fact that our observations closely associate enough for practical purposes is reality enough. There is no purer form of reality than this functions.
\par Though, this isn't enough for me, I think it could be enough if it was all there was but I want more. I do believe in truly objective reality and not just our semi-subjective observations of reality.
\par A refinement of the pragmatic argument to include the existence of an objective world would be a question of statistics. If a further world with separate physical laws existed it wouldn't work with our vast and extreme observations. Though, this wouldn't work with some more extreme theories wouldn't work with more radical and complete ideas, like Recreant's dream theory. Though, this implies that whatever objective reality is it must not be capable of interacting and effecting our reality. This can be attributed to nothing that could interact with our world, or our world is independent of the interactions of reality by its own definition. This may seem like a minor and obvious reduction but nevertheless it such reductions are the essence of philosophical discussion.
\par Next, such dream theories are useless. Derive no further discussion, only exist as a way to trick people up and thus serve no purpose. For these, the original pragmatic argument is all that can be derived secularly.
\par Now, you may say this is a good way to question our scientific premises. An argument against scientific realism, but in reality it doesn't change anything. Scientific realism stands or doesn't regardless of the 'validity' of such theories.
\par My final secular argument is that such philosophical arguments are harmful. That philosophy should not start with doubt but rather with perception. Such perception must be doubted, but doubt existing in its own self-containerized munition is useless. It defines nothing and leads to strange and possibly dangerous ideas. For instance Desecrate lead to Spinoza who lead to Hegel, who lead to Italian idealism, who lead to Giovanni Gentile the philosophical founder of fascism. Now, you may criticize that this was one complex branch, you could also derive similar lines through Aristotle or just any philosopher far back enough, and this is a good argument. Though, there is a clear line, also to other dangerous ideas like communism, NAZIsm, and so much more. These things are predicated on these strange cartesian/continental questions that derive no true thought, and thus should be understood in this way. Simply as pathetic and dirty tricks and shouldn't be see as true paths to knowledge. While yes, we should still engage sometimes, truly placing it on equal footing(or higher as we have been) is dangerous for our collective understanding reality and derived sense based upon it.
\par Though, many of you may have noticed I specify secular. There is a religious argument as well. If God created the reality as reality, then it is by definition reality. The only argument otherwise is that God doesn't exist(either entirely or he is not God)

\section{Nature and Validation of Axioms}
\par First, like always, we must define axioms thoroughly. The dictionary definition is "a statement accepted as true as the basis for argument or inference." For instance, our understanding of scientific realism is the understanding of the fact that our observations of reality are accurate(there are some observations that can be morphed due to other things that are observable.)
\par A simplified and more observable definition can then be defined to say that an axiom is a statement that either cannot be proved or cannot be proved currently that is used within a greater statement that can be proved upon its edifice. These axioms, while don't require 'proof' generally desire some type of reasoning even if it is not objective or purely analytical in nature. For instance, the above arguments for the objective nature of reality aren't true objective proof, they still exist as semi-arguments. While we cannot prove the objectivity of reality with true and objective reasoning, some reasoning can be applied to get some kind of 'proof' of the statement so that further studies can be applied based upon that said axiom.
\section{Non-infinite Definition of words and Axiomatic Semantics}
\par In the field of axioms, there becomes a problem. Part of the reason that 'some' words cannot be further defined. They are defined by themselves, they can only be intuitively understood, and even there is a limit.
\par A great example is existence, because to define this thesis, you need an antithesis. That doesn't exist(non-existence by definition needs to exist to be something and thus cannot exist because it would need to exist to be in existence.), so therefor existence is defined by itself.
\par This brings up a great deal of problems, paradoxes, and much more that will be explored.
\section{Induction \& Deduction}
\par In the expansion of human knowledge there are two main ways in which they can be reached; induction and deduction.
\par Deduction is by taking premises/axioms and derive conclusions based upon this. Induction is the opposite, from observing 'effects,' define whatever knowledge is desired.
\par This can be seen in physics, deduction is taking a fundamental theory and applying it in a new way to find novel uses, predict, or whatever may be needed. Induction would be taking data to come up with a way to model reality or come up with an entirely new theory.
\par Both of these are extremely important. Without induction we could never go beyond basic logic, but without deduction we can never find objective truth. The way I have found to be the most useful is to use induction to attempt to find our 'axioms' and then use these to deduct the reality around us, though this is only really works for my fields of interest. In other fields, like psychology this strategy would fail.

\section{The Problem of Induction}
\par Nothing can be proven true. As controversial of an idea as this may seem, it isn't. This basic axiom is the foundation of modern scientific epistemology. That theories can only be proven wrong, never correct. That is why they will always remain theories.
\par Now, you might think something as foundational as something like Newton is true. Though, there are questions through MOND(Modified Newtonian Mechanics) and others. Given, Newtons empirical evidence, there would need to be a lot of evidence for MOND for it to be taken anywhere near the same level
