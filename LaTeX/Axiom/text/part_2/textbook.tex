\chapter{Textbooks}
\section{Introduction}
Here I will write my notes from books I have read or am reading.
\section{Theoretical Physics by Georg Joos}
\par The vector analysis portion is intresting due to the fact that it takes something as simple as vector analysis and brings rigorous ideas to it like that $\oint ds=0 $ for close surfaces. 
\par I also never really thought about using vector analysis instead of tensors. 
\par The rest of curl, gauss, and such is very simple. Though, now it is getting into tensors though vectors.
$$dv_x=ds\nabla v_x$$
$$dv_y=ds\nabla v_y$$
$$dv_z=ds\nabla v_z$$
\par So therefore $dv=ds \nabla v$
\par To calculate v, three vectors or nine scaler must be known.
\par Another interesting addition is the fact that in physics, symmetric tensors can be represented as a surface to the second degree.
\par One thing I have noticed is I need a more intuitive grasp into the relationships between curl, div, laplance, and grad. They use it a lot to simplify the calculations. 
$$\nabla^2 f = \nabla \cdot (\nabla f)$$
The Laplace operator is equal to the divergence of the gradient.
$$\nabla \times (\nabla f) = 0$$
For smooth scaler fields. The curl of the gradient is zero.
$$\nabla \cdot (\nabla \times \mathbf{F}) = 0$$
The same is true for the divergence of the curl.
$\nabla^2 \mathbf{F} = \nabla(\nabla \cdot \mathbf{F}) - \nabla \times (\nabla \times \mathbf{F})$
For vector fields, the Laplace operator has different relations.
\\
\par Next, onto calculus of variations. I will derive Euler-Lagrange differential equation.
\par Let: $\tilde{y}$ be a neighboring function to $y$. Where $\in$ be a small quantity and $\eta(x)$ be a arbitrary function of x. so if $\tilde{y}=y+\epsilon \eta$ then $\tilde{y'}=y'+ \epsilon \eta'$. Here we stipulate that the two functions $\tilde{y}$ and $y$ converge at the beginning and end. Thus, $\eta$ must vanish at the ends. So if we substitute an integral $I$,, we find that it becomes a function of $\epsilon$. Then we require that $I(\epsilon)$ must have an extreme value of $\epsilon=0$. Here it is in mathematical terms:
$$I(\epsilon)=\int^{x_1}_{x_0}F(x,y+\epsilon \eta, y'+\epsilon\eta')dx=extremum \ for \ \epsilon=0$$
\par This gives us a simple way of determining the extreme value for a given integral. The condition is:
$$(\frac{dI}{D\epsilon})_{\epsilon=0}=0$$
\par We can then expand the integrad function $F$ in Taylor's series, according to the powers of $\epsilon$.
\par The differentiate with respect to $\epsilon$.
\par This expression then vanishes for $\epsilon=0$. Thus then simply remains the condition for the extremum.
\par Integrate this to get the Euler-Lagrange differential equation:
$$\frac{d}{dx}\frac{\partial F(x, y, y')}{\partial y'}-\frac{\partial F(x, y, y')}{\partial y}=0$$
\\
\par For writing constraining forces, we find it to be 
$$Z=\lambda \ grad \ G$$
\par Where $G(x,y,z)$ is the equation of the surface
\\

\section{On Sympathetic Reduction in Classical Mechanics}
\section{Magnetic Fields and Magnetic Diagnostics for Tokamak Plasmas by Alan Wooton}
\section{Advanced MHD with Applications to Laboratory and Astrophysical Plasmas by Cambridge }
\par For wide variety of MHD instabilities operating in tokamaks, represented by normal modes of the form(assuming that  cylindrical approximation and the toroidal representation may be ignored):
$$f(\psi,\vartheta, \varphi, t)=\sum\nolimits_m \tilde{f}(\psi) e^{i(m\vartheta + \eta \varphi - \omega t)} $$
\par Is only unstable for perpendicular wave vectors.
\par The reason is the enormous field line bending energy of the Alfven waves

\section{Hamiltonian description of the ideal Fuid by P.J.Morrison}
 \section{Classical Dynamics a Modern Perspective}
 \subsection{Chapter 6}
 A lot more importation stuff I forgot to write down, well I will start in chapter 6:
 If multiple canonical transformations, directly transforming would be canonical. $w \to w' \to w''$ then $w \to w''$. This is the group Composition Law.
 
 Next is the Associativity of Composition Law: 
 if $w’= \phi(w)$, $w” = \Phi(w’)$, and $w’'' = \xi(w")$
 so no matter what order of transformation they will be the same, whether ($T_3(T_2T_2)$ or $(T_3T_2)T_1$ or anything of the such.
 
 Identity: The identity transformation w’ = w obviously obeys the conditions for being canonical.

Inverses: If $\omega \to \omega’$ is canonical, then so is the transformation 
id $\omega’ \to \omega$. 

"Thus all the properties needed to define a group are trivially obeyed. (In 
Chapter 12 we discuss group structure in a little more detail). "

The canonical group is characterized by the dimension, 2k, 
of the phase space. 

The subgroups of such a phasespace is very large, though three can be easily defined.

Another subgroup is the contact transformations.(when a poission bracket can be formed)

Note: changing coordinate systems can satisfy phasespace rules, but isn't generally considered to be a transformation

"Intuitively speaking, a finite canonical transformation that can be 
connected continuously to the identity should be built up as a succession of 
infinitesimal transformations."

To express the relation between Poission Brackets and canonical transformations through differential equations:
$$\frac{d \omega}{d \theta}= \{\omega^\mu, \phi(\omega)\}_\omega= \epsilon^{\mu v}\frac{\partial \phi(\omega)}{\partial \omega^v}$$

Let us first clarify the meaning of this equation. We look upon the $w^\mu$ as 
functions of $\theta$; $\phi$ is a function with a fixed functional form, and $w^\mu$ are the unknowns in the differential equation. Thus on the right-hand side of this equation, the arguments of $\phi$ are just the quantities we are trying to solve for. From the theory of differential equations, we know that there is a unique solution for $w^\mu$ if we are given the values of $w^\mu$ at $\theta = 0$. Calling these boundary values of $w^\mu$ as $w^\mu_0$, the solution of the equation can be written as: 
$$\omega^\mu= \varphi^\mu(\omega_0; \theta), \space \space \space \varphi^\mu(\omega_0; 0) = \omega^\mu_0$$

\subsection{Chapter 7}
