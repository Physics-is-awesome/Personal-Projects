
\chapter{Textbooks}
\section{Introduction}
Here I will write my notes from books I have read or am reading.
\section{Theoretical Physics by Georg Joos}
\par The vector analysis portion is intresting due to the fact that it takes something as simple as vector analysis and brings rigorous ideas to it like that $\oint ds=0 $ for close surfaces. 
\par I also never really thought about using vector analysis instead of tensors. 
\par The rest of curl, gauss, and such is very simple. Though, now it is getting into tensors though vectors.
$$dv_x=ds\nabla v_x$$
$$dv_y=ds\nabla v_y$$
$$dv_z=ds\nabla v_z$$
\par So therefore $dv=ds \nabla v$
\par To calculate v, three vectors or nine scaler must be known.
\par Another interesting addition is the fact that in physics, symmetric tensors can be represented as a surface to the second degree.
\par One thing I have noticed is I need a more intuitive grasp into the relationships between curl, div, laplance, and grad. They use it a lot to simplify the calculations. 
$$\nabla^2 f = \nabla \cdot (\nabla f)$$
The Laplace operator is equal to the divergence of the gradient.
$$\nabla \times (\nabla f) = 0$$
For smooth scaler fields. The curl of the gradient is zero.
$$\nabla \cdot (\nabla \times \mathbf{F}) = 0$$
The same is true for the divergence of the curl.
$\nabla^2 \mathbf{F} = \nabla(\nabla \cdot \mathbf{F}) - \nabla \times (\nabla \times \mathbf{F})$
For vector fields, the Laplace operator has different relations.
\\
\par Next, onto calculus of variations. I will derive Euler-Lagrange differential equation.
\par Let: $\tilde{y}$ be a neighboring function to $y$. Where $\in$ be a small quantity and $\eta(x)$ be a arbitrary function of x. so if $\tilde{y}=y+\epsilon \eta$ then $\tilde{y'}=y'+ \epsilon \eta'$. Here we stipulate that the two functions $\tilde{y}$ and $y$ converge at the beginning and end. Thus, $\eta$ must vanish at the ends. So if we substitute an integral $I$,, we find that it becomes a function of $\epsilon$. Then we require that $I(\epsilon)$ must have an extreme value of $\epsilon=0$. Here it is in mathematical terms:
$$I(\epsilon)=\int^{x_1}_{x_0}F(x,y+\epsilon \eta, y'+\epsilon\eta')dx=extremum \ for \ \epsilon=0$$
\par This gives us a simple way of determining the extreme value for a given integral. The condition is:
$$(\frac{dI}{D\epsilon})_{\epsilon=0}=0$$
\par We can then expand the integrad function $F$ in Taylor's series, according to the powers of $\epsilon$.
\par The differentiate with respect to $\epsilon$.
\par This expression then vanishes for $\epsilon=0$. Thus then simply remains the condition for the extremum.
\par Integrate this to get the Euler-Lagrange differential equation:
$$\frac{d}{dx}\frac{\partial F(x, y, y')}{\partial y'}-\frac{\partial F(x, y, y')}{\partial y}=0$$
\\
\par For writing constraining forces, we find it to be 
$$Z=\lambda \ grad \ G$$
\par Where $G(x,y,z)$ is the equation of the surface
\\

\section{On Sympathetic Reduction in Classical Mechanics}
\section{Magnetic Fields and Magnetic Diagnostics for Tokamak Plasmas by Alan Wooton}
\section{Advanced MHD with Applications to Laboratory and Astrophysical Plasmas by Cambridge }
\par For wide variety of MHD instabilities operating in tokamaks, represented by normal modes of the form(assuming that  cylindrical approximation and the toroidal representation may be ignored):
$$f(\psi,\vartheta, \varphi, t)=\sum\nolimits_m \tilde{f}(\psi) e^{i(m\vartheta + \eta \varphi - \omega t)} $$
\par Is only unstable for perpendicular wave vectors.
\par The reason is the enormous field line bending energy of the Alfven waves

\section{Hamiltonian description of the ideal Fuid by P.J.Morrison}
 \section{Classical Dynamics a Modern Perspective}
 \subsection{Chapter 6}
 A lot more importation stuff I forgot to write down, well I will start in chapter 6:
 If multiple canonical transformations, directly transforming would be canonical. $w \to w' \to w''$ then $w \to w''$. This is the group Composition Law.
 
 Next is the Associativity of Composition Law: 
 if $w’= \phi(w)$, $w” = \Phi(w’)$, and $w’'' = \xi(w")$
 so no matter what order of transformation they will be the same, whether ($T_3(T_2T_2)$ or $(T_3T_2)T_1$ or anything of the such.
 
 Identity: The identity transformation w’ = w obviously obeys the conditions for being canonical.

Inverses: If $\omega \to \omega’$ is canonical, then so is the transformation 
id $\omega’ \to \omega$. 

"Thus all the properties needed to define a group are trivially obeyed. (In 
Chapter 12 we discuss group structure in a little more detail). "

The canonical group is characterized by the dimension, 2k, 
of the phase space. 

The subgroups of such a phasespace is very large, though three can be easily defined.

Another subgroup is the contact transformations.(when a poission bracket can be formed)

Note: changing coordinate systems can satisfy phasespace rules, but isn't generally considered to be a transformation

"Intuitively speaking, a finite canonical transformation that can be 
connected continuously to the identity should be built up as a succession of 
infinitesimal transformations."

To express the relation between Poission Brackets and canonical transformations through differential equations:
$$\frac{d \omega}{d \theta}= \{\omega^\mu, \phi(\omega)\}_\omega= \epsilon^{\mu v}\frac{\partial \phi(\omega)}{\partial \omega^v}$$

Let us first clarify the meaning of this equation. We look upon the $w^\mu$ as 
functions of $\theta$; $\phi$ is a function with a fixed functional form, and $w^\mu$ are the unknowns in the differential equation. Thus on the right-hand side of this equation, the arguments of $\phi$ are just the quantities we are trying to solve for. From the theory of differential equations, we know that there is a unique solution for $w^\mu$ if we are given the values of $w^\mu$ at $\theta = 0$. Calling these boundary values of $w^\mu$ as $w^\mu_0$, the solution of the equation can be written as: 
$$\omega^\mu= \varphi^\mu(\omega_0; \theta), \space \space \space \varphi^\mu(\omega_0; 0) = \omega^\mu_0$$

\subsection{Chapter 7}
...

\section{Numerical Hamiltonian Problems}
\subsection{Chapter 1: Hamiltonian Systems}

MacKay and Meiss have compiled great papers on the subject in 1987, Marsen (1992) offers a more geometric introduction.

Usually in mechanics, q is generalized coordinates while p is conjugated generalized momenta. H being total mechanical energy.

I like how they show the nature of the Hamiltonian system to be a logical nessesity based upon how the phase space is constructed, rather than purely physical construction.

It is sometimes useful to combine dependent variables into a 2d $ y = (p,q) $.
$$ \dv{y}{t} =J^{-1} \grad H $$

Where, $\grad $ is the gradient operator and $J$ is skew-symmetric Matrix


\subsection{Chapter 2: Symplecticness}

As mentioned, Hamiltonian problems are exeptional. Most systems do not allow for area preservation(energy conservation, time symmetry, etc), which is a requirment of Hamiltonian systems.
That is why exploration into other systems like Nambu, Metriplectic, etc is needed.

This area preservation is not simply required as a general rule but is the basis of the structure. All other properties of Hamiltonian dynamics is derived from the area property.

One can check for the preservation ogarea by checking if the Jacobian is exactly 1 or a simpler way is to find it is if $ dp* \wedge dq* = dp \wedge dq $ or that the pull back of the 2-form is equal to the origional 2-form.

In multiple-diminsions. Look for multiple area preservations rather than multi-diminsional preservation.

\section{Chapter 2: Numerical Methods}

Stiff problems are defined to be where numerical stability, not accuracy, forces you to take extremely small steps.

Explicit methods requires s number of evaluations per steps while implicit provides coupled system of $s \times D $ algebraic systems and $s \times D $ components of stage vectors.

After reading this I wrote a program in Julia to compare the time it takes for both an explicit RK and Implicit to solve a ODE. It was more challenging than it should have been.

\subsection{Chapter 4: Order Condition}

More so Computer science stuff, not so important at the moment but something to look at later.

\subsection{Chapter 5: Implementation}
Really just the stuff I did earlier when writing my own implicit RK.

\subsection{Chapter 6: Symplectic Integration}

If $ b_i a_{ij} + b_j a{ji} - b_i b_j = 0 $ for $i,j = 1,...,s$. Then the method is symplectic. They have a nice proof that I can use as a reference.

Most all symplectic methods are implicit.

They go through solutions of various families of symplectic RK integration.

\subsection{Chapter 7: Symplectic Order Conditions}

I still don't get order conditions. I will end it at one point.

\subsection{Chapter 8: Numerical Experimentations}

Page 115 lists multiple symplectic numerical methods, the rest of the chapter examines them. Too much for me to write but useful and interesting
information nonetheless.

One thing I will say is that they claimed that Calvo is clearly the best and very little improvement is likely possible.

\subsection{Chapter 10: Properties of Symplectic Integrators}
\subsubsection{Backward error analysis}
How much must the numerical problem be perturbed so that the numerical problem is equal to the analytical.
This is especially useful in uncertain models, in which perturbation is the best we can do.

This only holds if the numerical solution of time $t_n$ is computed by iterating $n$ times $1$ in the same symplectic maps. Thus, variable-step integrations fail.

\subsubsection{Alternative Approach}
There is another method through repeating periodically; though, this system is now non-autonomous and is discontinous.

\subsubsection{Conserved Energy and Quantities}
While there are exceptions for unique cases, generally one must choice whether to have conserved energy or sympletic structure. Obviously, choosing must
be done by the researcher, but generally sympletic is superior for the fact that it does not confine the nature of the dynamics.

There is a great deal of research into methods that exactly conserve H. This is not covered in the book.

There are several ways to come with close approximations of H though.

\subsubsection{KAM}
KAM theory proves that symplectic methods can show that nonlinear effects are not destabilizing.

\subsection{Chapter 11: Generating Functions}
Mentioned earlier, $S$ can be used to express symplectic mappings and transformations. $S$ is a single real value function. This function it called
the \emp{generating function}.

\subsubsection{First kind}
Let $(p*,q*) = \phi(p, q)$ be a sympletic transformation defined in a simply connected $\Omega$ domain. For each closed path $\gamma$:
$$ \int_\gamma p dq-\int_\gamma p*dq* = 0$$
Where $pdq$ is the differential form.

Making it so that
$$ dS= pdq - p*dq* $$

Further assume that $q$ and $q*$ are functions independent of $\Omega$. If this is true we can express $S(p, q) $ as function $S^1$ of q and q*.. Thus,
\[
\begin{aligned}
p &= \frac{\partial S^1}{\partial q},
\end{aligned}
\qquad
\begin{aligned}
p* &= \frac{\partial S^1}{\partial q*}
\end{aligned}
\]

\subsubsection{Third kind}
What happended to \emp{second}, I don't know.

If q and q* are not independent. Then,if p and q* are independent you can express $S^3(p, q*). The rest is self-explanitory.

\subsubsection{All kinds}
One can use the Poincare generating function.
\subsubsection{Hamiltonian-Jacobi}

