\chapter{Numerical Hamiltonian Problems}
\section{Chapter 1: Hamiltonian Systems}

MacKay and Meiss have compiled great papers on the subject in 1987, Marsen (1992) offers a more geometric introduction.

Usually in mechanics, q is generalized coordinates while p is conjugated generalized momenta. H being total mechanical energy.

I like how they show the nature of the Hamiltonian system to be a logical nessesity based upon how the phase space is constructed, rather than purely physical construction.

It is sometimes useful to combine dependent variables into a 2d $ y = (p,q) $.
$$ \dv{y}{t} =J^{-1} \grad H $$

Where, $\grad $ is the gradient operator and $J$ is skew-symmetric Matrix


\section{Chapter 2: Symplecticness}

As mentioned, Hamiltonian problems are exeptional. Most systems do not allow for area preservation(energy conservation, time symmetry, etc), which is a requirment of Hamiltonian systems.
That is why exploration into other systems like Nambu, Metriplectic, etc is needed.

This area preservation is not simply required as a general rule but is the basis of the structure. All other properties of Hamiltonian dynamics is derived from the area property.

One can check for the preservation ogarea by checking if the Jacobian is exactly 1 or a simpler way is to find it is if $ dp* \wedge dq* = dp \wedge dq $ or that the pull back of the 2-form is equal to the origional 2-form.

In multiple-diminsions. Look for multiple area preservations rather than multi-diminsional preservation.

\section{Chapter 2: Numerical Methods}

Stiff problems are defined to be where numerical stability, not accuracy, forces you to take extremely small steps.

Explicit methods requires s number of evaluations per steps while implicit provides coupled system of $s \times D $ algebraic systems and $s \times D $ components of stage vectors.

After reading this I wrote a program in Julia to compare the time it takes for both an explicit RK and Implicit to solve a ODE. It was more challenging than it should have been.

\section{Chapter 4: Order Condition}

More so COmputer science stuff, not so importaint at the moment but something to look at later.

\section{Chapter 5: Implementation}
Really just the stuff I did earlier when writing my own implicit RK.

\section{Chapter 6: Symplectic Integration}

If $ b_i a_{ij} + b_j a{ji} - b_i b_j = 0 $ for $i,j = 1,...,s$. Then the method is symplectic. They have a nice proof that I can use as a reference.

Most all symplectic methods are implicit.

They go through solutions of various families of symplectic RK integration.

\section{Chapter 7: Symplectic Order Conditions}
