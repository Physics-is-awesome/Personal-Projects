\chapter{Video classes}
\section{Differential Geometry - Robert Davie}
\subsection{1-forms}
These are linear functionals that map tangent vectors at a point on a manifold to real numbers. In local coordinates on a manifold $M$, a 1-form can be written as $\omega = \omega_i dx^i$, where $\omega_i$ are smooth functions and $dx^i$ are basis 1-forms (dual to coordinate basis vectors). For example, the differential of a function $df = \frac{\partial f}{\partial x^i} dx^i$ is a 1-form. They are used to measure vectors, like gradients or line integrals.
\subsection{2-forms}
These are antisymmetric bilinear maps that take two tangent vectors and produce a real number. In coordinates, a 2-form looks like $\omega = \omega_{ij} dx^i \wedge dx^j$, where $\wedge$ denotes the wedge product (ensuring antisymmetry: $dx^i \wedge dx^j = -dx^j \wedge dx^i$). 2-forms are used to measure oriented areas, such as in surface integrals or the electromagnetic field tensor in physics.
\subsection{3-forms}
These are antisymmetric trilinear maps, taking three tangent vectors to a real number. In coordinates, a 3-form is $\omega = \omega_{ijk} dx^i \wedge dx^j \wedge dx^k$, with antisymmetry in all indices. They measure oriented volumes and are used in integrals over 3-dimensional submanifolds, like in fluid dynamics or general relativity.
\subsection{k-form}
This can then be made general as
$$\omega=\Sigma_{i_1..i_k}\omega_{i_1...i_k} dx^{i_1}... \wedge dx^{i_k}$$

\subsection{Introduction to Exterior calculus}
An exterior derivative maps k-forms into k+1-forms.
$$d: \wedge^k(M) \to \wedge^{k+1}(M)$$
and must follow these rules
Linearity: $d(\alpha + \beta) =d\alpha + d\beta$
Leibniz(product rule): $\wedge(\alpha + \beta)= d\alpha \wedge \beta + (-1)^k \alpha \wedge d\beta $
Nilpotency: $d(d\alpha)=0$
$$\therefore d \omega=\Sigma_{i<j} (\frac{\partial \omega_i}{\partial x_j} - \frac{\partial \omega_j}{\partial x_i})dx^i \wedge dx^j$$
In 3D, the exterior derivative of the 2-form correspondence to the divergence in vector field.

Allows for integration of manifolds. Shows that the integral of the exterior derivative of a differential form over a manifold of the integral of the form itself over the boundary of the manifold: 
$$\int_M d \omega = \int_{\partial M} \omega$$
This unifies vector calculus in areas like divergence or green's theorem.

It also show vector flow

Lie derivatives measure the change of forms
\subsection{Exterior calculus-2}
The exterior for a k-form is
$$d \omega=\sum  (\sum_{n=1} \frac{\partial x_{i.....k}}{\partial x^{i_n}}dx^{i_n})\wedge dx^{i_1} \wedge .... \wedge  dx^{i_k}$$

4-form is the highest form

Back to to integration of manifolds with stokes theorem.
While $\int_M d \omega = \int_{\partial M} \omega$ is true $\int dx \wedge dy$ and $\int dxdy$ represent slightly different mathematical concepts. In diff forms, it is the integral of an oriental area. In Multi-variable calculus it is the integral of over a region in the xy-plane
\subsection{Wedge Product-2}
Properties of the wedge product:
Anti-symmetry: $\omega \wedge \eta= (-1)^{pq} \eta \wedge \omega$. Where p \& q are the dimensionality of $\omega, \eta$

Symmetric group: $S_{p+q}$ is the group of all possible permutations. 

Permutation and sign of permutations: $\sigma$ is essentially a specific way of ordering a set of elements. The sign of permutation $sign(\sigma)$ tells us if the permutation can be achieved in an even or odd number of swaps.
Where odd is negative and even is positive

Role of these two things: The sign tells us the sign of a wedge product given a permutation. $dx^1 \wedge dx^2\to -dx^2 \wedge dx^1$. The symmetric groups tells us all possible permutations. 
\begin{align*}
    \therefore (\omega \wedge \eta)(X_1..., X_{p+q})=\frac{1}{p!q!}\sum_{\sigma \in S_{p+q}}sign(\sigma)\omega(X_{\sigma(1)}..., X_{\sigma(p)}), \\ \eta(X_{\sigma(p+1)}...X_{\sigma(p+q)})
\end{align*}
This $\frac{1}{p!q!}$ ensures correct normalization.
\subsection{Introduction to the Hodge Star Operation}
This operator allows us to relate manifolds. Crucial in formulating coordinate free theories. It transforms k-forms into (n-k)-forms.
Specifically it is an isomorphism between the spaces of k-forms and (n-k)-forms on an n-dimensional Riemannian(M,g). Where g is the metric tensor.
It is denoted as:
$$\star: \wedge^k(T^*M)\to \wedge^{n-k}(T^*M)$$

The way it works is hard to write out, so just remember it.
To construct a hodge star
$$\omega \wedge \star \eta = \langle \omega, \eta \rangle vol_M$$
Where $\langle .,. \rangle$ is the inner product
and Vol is the volume form of M. Defined as $vol_M=\sqrt{|g|} dx^1 \wedge .. \wedge dx^n$ Where $|g|$ is the determinate of the Metric Tensor.

The equation is long, so just look it up, it is simple but long.

\subsection{Hodge star 2}
Hodge star establishes duality between geometric theories.
Involution: Doing it twice returns it to the original 
Linear transformations: You know what this means
Interacts like multiplication with exterior derivative: $\star(d \omega) = d(\star \omega)$
The co-differential is defined using the Hodge Star.
\subsection{The push forward of vectors on manifolds}
A pushforward is a way to map a vecotr on a tangent space onto another manifold's tangent space.

Properties:

Linearity: The pushforward $ \phi_*: T_p M \to T_{\phi(p)} N $ is a linear map. For vectors $ v, w \in T_p M $ and scalars $ a, b $,

$$\phi_*(a v + b w) = a \phi_* v + b \phi_* w$$

Chain rule: If you have another smooth map $ \psi: N \to P $, the pushforward satisfies the composition rule:

$$(\psi \circ \phi)_* = \psi_* \circ \phi_*$$

This follows from the chain rule for derivatives.

Action on curves: If you think of a vector $ v \in T_p M $ as the tangent to a curve $ \gamma: (-\epsilon, \epsilon) \to M $ with $ \gamma(0) = p $, then $ \phi_* v $ is the tangent vector to the curve $ \phi \circ \gamma $ at $ \phi(p) $.

The pushforward $ \phi_* v \in T_{\phi(p)} N $ is given by:
$$\phi_* v = v^i \frac{\partial \phi^j}{\partial x^i} \frac{\partial}{\partial y^j}$$
Here, $ v^i $ are the components of $ v $, and $ \frac{\partial \phi^j}{\partial x^i} $ are the Jacobian entries. The result is a vector in $ T_{\phi(p)} N $, expressed in the basis $ \left\{ \frac{\partial}{\partial y^1}, \dots, \frac{\partial}{\partial y^n} \right\} $.

\subsection{Pull-Back in k-forms}
Given a smooth map $ \phi: M \to N $ between manifolds $ M $ (dimension $ m $) and $ N $ (dimension $ n $), and a $ k $-form $ \omega \in \Omega^k(N) $ on $ N $, the pullback $ \phi^* \omega \in \Omega^k(M) $ is a $ k $-form on $ M $. The pullback essentially "transfers" $ \omega $ from $ N $ to $ M $ by composing it with the map $ \phi $.
The pullback is defined pointwise. For a point $ p \in M $, and $ k $ tangent vectors $ v_1, \dots, v_k \in T_p M $, the pullback $ \phi^* \omega $ at $ p $ is:
$$(\phi^* \omega)_p (v_1, \dots, v_k) = \omega_{\phi(p)} (\phi_* v_1, \dots, \phi_* v_k)$$
Here:

$ \phi_*: T_p M \to T_{\phi(p)} N $ is the pushforward (differential) of $ \phi $, which maps each tangent vector $ v_i $ to $ \phi_* v_i $.
$ \omega_{\phi(p)} \in \wedge^k T_{\phi(p)}^* N $ is the $ k $-form $ \omega $ evaluated at $ \phi(p) $, acting on the pushed-forward vectors.

This definition ensures that $ \phi^* \omega $ is a $ k $-form on $ M $, as it takes $ k $ vectors in $ T_p M $ and produces a number.

Key Properties of the Pullback

Linearity: The pullback $ \phi^*: \Omega^k(N) \to \Omega^k(M) $ is a linear map.

Preserves degree: The pullback of a $ k $-form is a $ k $-form.

Functoriality: For maps $ \phi: M \to N $ and $ \psi: N \to P $,
$$(\psi \circ \phi)^* = \phi^* \circ \psi^*$$

Commutes with exterior derivative: For any $ k $-form $ \omega $,
$$d (\phi^* \omega) = \phi^* (d \omega)$$
This is a powerful property, making pullbacks compatible with the exterior calculus.

Wedge product: The pullback respects the wedge product:
$$\phi^* (\omega \wedge \eta) = (\phi^* \omega) \wedge (\phi^* \eta)$$



Computing the Pullback in Coordinates
To compute $ \phi^* \omega $ in practice, we typically use local coordinates. Here’s the step-by-step process:

Choose coordinates:

On $ M $, use coordinates $ (x^1, \dots, x^m) $ around $ p $.

On $ N $, use coordinates $ (y^1, \dots, y^n) $ around $ \phi(p) $.

The map $ \phi $ is expressed as $ y^i = \phi^i(x^1, \dots, x^m) $.

Express the k-form:

Let $ \omega \in \Omega^k(N) $ be written in coordinates as:
$$\omega = \sum_{I} a_I(y) \, dy^{i_1} \wedge \dots \wedge dy^{i_k}$$
where $ I = (i_1, \dots, i_k) $ is a multi-index with $ 1 \leq i_1 < \dots < i_k \leq n $, and $ a_I(y) $ are smooth functions on $ N $.


Pull back the form:

The pullback is:
$$\phi^* \omega = \sum_{I} (a_I \circ \phi) \, \phi^* (dy^{i_1} \wedge \dots \wedge dy^{i_k})$$

Compute $ a_I \circ \phi $: Replace $ y $ with $ \phi(x) $, so $ a_I(y) = a_I(\phi(x)) $.

Compute the pullback of the basis $ k $-form:
$$\phi^* (dy^{i_1} \wedge \dots \wedge dy^{i_k}) = \phi^* (dy^{i_1}) \wedge \dots \wedge \phi^* (dy^{i_k})$$

For a 1-form $ dy^j $, the pullback is:
$$\phi^* (dy^j) = d (y^j \circ \phi) = d (\phi^j(x)) = \frac{\partial \phi^j}{\partial x^\ell} dx^\ell$$
(using the chain rule, summing over $ \ell $).
Thus:
$$\phi^* (dy^{i_1} \wedge \dots \wedge dy^{i_k}) = \left( \frac{\partial \phi^{i_1}}{\partial x^{\ell_1}} dx^{\ell_1} \right) \wedge \dots \wedge \left( \frac{\partial \phi^{i_k}}{\partial x^{\ell_k}} dx^{\ell_k} \right)$$

This is a $ k $-form on $ M $, expressed in the $ dx^\ell $ basis.


Simplify:

Expand the wedge products and collect terms.
The result is a sum of terms involving $ dx^{\ell_1} \wedge \dots \wedge dx^{\ell_k} $, with coefficients that depend on the Jacobian $ \frac{\partial \phi^j}{\partial x^\ell} $ and $ a_I(\phi(x)) $.

\subsection{Pull-Back of volume forms}
The pullback of a volume form $ \omega = f(y) \, dy^1 \wedge \dots \wedge dy^n $ on $ N $ under a map $ \phi: M \to N $ is an $ n $-form on $ M $, computed as:
$$\phi^* \omega = (f \circ \phi) \cdot \det \left( \frac{\partial \phi^i}{\partial x^{j_k}} \right) \, dx^{j_1} \wedge \dots \wedge dx^{j_n}$$
If $ \dim M = \dim N $, and $ \phi $ is a local diffeomorphism, $ \phi^* \omega $ is a volume form on $ M $. If $ \dim M < \dim N $, the pullback is zero. The Jacobian determinant captures the volume scaling, making this operation central to integration and geometric transformations.
\subsection{Integration on Manifolds Using the Pullback of Volume Forms}
The pullback of volume forms allows for integration of manifolds. Which is useful in the integration of curved coordinate systems.

volumes generalize "dx" or "dxdy."

The steps in integration of manifolds using volume forms is:

\textbf{Step 1:} $   M   $: The domain of integration (often $   \mathbb{R}^n   $ or a simpler manifold).
$   N   $: The target manifold with a volume form $   \omega   $.
$   \phi: M \to N   $: A smooth map, typically a parametrization or diffeomorphism.
$   f: N \to \mathbb{R}   $: The function to integrate (if $   f = 1   $, we’re computing the volume).


\textbf{Express the volume form:}

Write $   \omega = f_\omega(y) \, dy^1 \wedge \dots \wedge dy^n   $ in coordinates on $   N   $.


\textbf{Compute the pullback:}

\textbf{Compute $   \phi^* \omega   $, which involves:}

Substituting $   y = \phi(x)   $ into $   f_\omega(y)   $.
Computing the Jacobian determinant of $   \phi   $.
Forming $   \phi^* \omega = (f_\omega \circ \phi) \cdot \det \left( \frac{\partial \phi^i}{\partial x^j} \right) \, dx^1 \wedge \dots \wedge dx^n   $ (if $   m = n   $).




\textbf{Set up the integral:}

The integral becomes:
$$\int_{\phi(M)} f \, \omega = \int_M (f \circ \phi) \cdot \phi^* \omega$$

In coordinates, if $   \phi^* \omega = g(x) \, dx^1 \wedge \dots \wedge dx^n   $, and $   M   $ corresponds to a region $   U \subset \mathbb{R}^n   $:
$$\int_U (f \circ \phi)(x) \cdot g(x) \, dx^1 \dots dx^n$$



\textbf{Evaluate the integral:}

Perform the integral over $   U   $ using standard multivariable calculus techniques, accounting for the Jacobian determinant and any orientation changes.
