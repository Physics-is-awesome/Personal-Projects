
\chapter*{Introduction}
% your text here
"The mind is everything. What you think you become." | Budda \\
\par The following introduction is slightly outdated, but will remain. The reason I claim it to be outdated is because I have decided to change the foundation of my book. Previously I planned for this to be a philosophical manuscript, but I have since expanded it to all of my ideas; physics, language, personal anecdotes, and many more that I will examine through the Socratic dialogue I mentioned in my original introduction. Also, the Socratic dialogue is later either added or discarded by whim alone. Now, why do I keep my earlier introduction? I feel that it does explain my thoughts very well, and it is a documentation of my thoughts at that time.
\par

\par This text is an examination and explanation of my own philosophy through the use of Socratic dialogue and other similar methods of description through the use of dialogue like prose with questions and answers, obviously it will be a very one sided conversation but nonetheless is will be structured in a way similar to that of a conversation. 

\par "Now, you may be asking why I choice this style, you aren't (considering you don't exit, I have no intention of actually having someone reading this) but I will answer it anyways. That is the the purpose exactly, by asking and answering simple questions it will hopefully lead to a clearer explanations and possibly even a better understanding myself. 

\par It is also related to the fact that I find this form of communication seems more natural to me. Which is quite odd considering the fact that in actual conversation I do not speak in this manner, I am much more curt, short spoken, and 'robotic.' If I were to guess it would have something to do with the fact that I understand myself to a better degree, thus I am able to use more precise language when conversing with myself(or less persistence depending on the context). That it is precisely because I have no intentions of anyone else reading my writing, hearing my thoughts, or even hearing the words that occasionally come about out load when I speak to myself that I am comfortable in speaking manner. A manner captures the concept beyond the bare bones, that explains the entire idea without any room for interpretation, with only the exceptions of the time purposely meant to be left up to interpretation. Lastly, the strongest reason is because of the time I have, when speaking with other I have to be quick and witty with my speech. Using the bare minimum so they do not gloss over my words but rather understand them at least to the degree to have a dialogue. With this I am free to take the time to mule over my thoughts rather than speak on instinct alone. When speaking to myself, I realize I speak them again and again trying to come with the perfect way to say it. That for as quick witted as I am, I am not enough, I am not as quick as I am intelligent and not as intelligent as I believe(don't know why I wrote that last part, while likely true it has little relation to that that was written before it.)

\par This writing is a great example of the strength of this Socratic dialogue, through these questions that would seldom come up without the purposeful slowing and ...pausing(haha) [haha, who writes *haha* in their own book, especially only one they would read]((anyways I am getting distracted)) Not only have begone(I know that is misspelled, I do not care you pedant) to understand why I speak in a different manner to myself, but it also begins to explain why I also seem to have a different sense of humor while in isolation. While when with others, I normally focus on irony, juxtaposing ideas to illicit laughter, and I also use a lot of absurdism. I primary use the exaggeration of my own paradoxical nature. I also tend to use a lot of sarcasm and insults. Though when in isolation I primary create more complex and abstract forms of irony, such as metairony, post-irony, but rarely post-metairony(I don't like it as much). That I fine humor in jokes that have many many meanings, that take time to comprehend all of the facites and possible interpretations, and also more abstract humor, humor that is funny not for the sake of itself but because of its own sake.(That makes no sense, but I know what I mean, I should try to come up with a better way of wording such ideas.)"

\par Back to me from the future(present, past?), anyways. This will be a collection of thoughts covering several branches of thoughts and correlations. In effect, it will be writing down the conversations that I generally have with myself.


\chapter{Metaphysics}
The verbal interpretation, on the other hand, i.e. the metaphysics of quantum physics, is on far less solid ground. In fact, in more than forty years physicists have not been able to provide a clear metaphysical model. | Erwin Schrodinger 
\section{Scientific Materialism and God}

\par Despite the fact that there exist no 
formal term for such \\
philosophies(I should come up with a name for it, Christian Scientific Materialism?)it in a broad stroke is most likely the most common view of meta-physics of all, though I plan to expand upon the concepts more deeply than that which is common. In essence this view can be summarized by the combination of two seemingly contradictory concepts:

\par    First, is scientific materialism, this is the idea that there is nothing further than the material, that reality can be derived empirically through the laws of science and nature.

\par    Second, the idea of the Christian God. That there exist an external omnibenevolent, omnificent, omnipotence, omnipresence, and omniscience as described by the Christian Bible. Later I will explain my faith in more detail, but for now this is sufficient.

\par You may be wondering how these concepts can be combined due to their seemingly contradictory nature. This is a simple enough question, if the material is all that exist how can a non-material God exist? The answer in all reality is very complex, but it can be simplified to the fact that I view the material world to follow the philosophy of scientific materialism but there to also be a non-material "world" that God exists within. The reality of such a non-material world is sadly beyond my grasps due to my life within the material, I make no assertion that I understand such things. Partly because by definition such things are not possible to understand, like a 4th spacial dimension. While we can understand it through analogies and abstraction based upon prior more concrete idea, but sense we have no clear or objective priors we have no capacity to do such. \\
You can simplify such thoughts through formal logic \\

Let: \\
X=Material \\
Y=Non-Material \\
C=Laws of nature and reality \\
C(X)=Matter/Energy/Both interactions(all material action) \\
C(X,Y)= Observable phenomena
A=Scientifically explainable phenomena \\
$$
X \implies C
$$
$$
C(X) \iff C(X) \models A(X) \top
$$
$$
\forall C(X,Y) \exists C
$$
$$
\nexists Y(C) \therefore \nexists A(Y)
$$
\section{Scientific Realism vs Anti-realism}
\par What is scientific realism and anti-realism? It is the debate between whether our theories of science are ontologically real, or simple predictive models. 
\par My view is that it depends, when it comes to macroscopic and easily observable and prove able; these theories aren't true. You may be thinking the opposite, that it is those that are most definitively true and the rest that aren't. Though I would argue these are the models; a ball don't fall; a collection of subatomic particles follow their geodistic, which is restricted by other particles and forces, and so much more. This can be applied to all things other than the most fundamental of physics. It can especially true for the softer sciences; chemist, biology, psychology, and so many more.
\par While I think science does study the metaphysical truth through fundamental physics and abstract mathematics, on a broad stroke it does not.
\section{Causality}
\par Now I am sure you are confused, I mentioned I would dive into deeper questions related to free will, determinism, what is real, divine voluntarism, occasionalism, how quantum mechanics plays within this and other 'deeper' concepts, but we must take things one at a time.
\par So, let us dive in. What is causality, it is very simple. It is the concept that things happen because other things had happened first. That all things  have a 'cause' and that that thus causes the effect(then they later become the cause of a later thing)
\par I can hear you screaching that this is non-sense, who in their right mind would question something as basic as cause and effect? Why even look into it. First, that is extremely intellectually lazy, all things have debates against it, you should always look into it. Though, more importantly, there actually are debates against it.
\par The easiest to dismiss is Hume's skepticism, that causality is simply a human made concept. That things doesn't really happen because of another action but simply that both actions are just sequences of events. Though, I would argue that this pedantic argument of definitions doesn't truly work. On an epistemological basis, causality is easier for humans to understand, and though breaking things into chunks we may better understand the world. That either concept is equally true, but one concept as a greater pragmatic 'truth' and you can argue that that correlates to a better empirical truth. That there is even a speed of causality written in the nature of reality.(The speed of light) A bit more deeply, once you quantize to the plank time, things truly do divide. Something happens, then a next thing happens. The complexity and abnormal reality of quantum mechanics makes this argument a little fuzzy, and thus leads me to my next paragraph.
\par Now those unfamiliar with quantum mechanics may be confused, how does quantum mechanics may be a little confused. I will attempt to explain it to the best of my ability, but the reality of quantum mechanics is extremely complex and without a clear understanding of differential equations, discrete mathematics, linear algebra, and much more it is impossible to understand. I know many science communicators claim to teach "quantum physics and mechanics" to they layman, but this is a lie. These half backed analogies are not quantum mechanics. It does not come close, but luckily for me I am the only one reading this book I will be able to understand my attempt of analogizing(?) the philosophical question.
\par In essence, in quantum mechanics. Things are not local, for many reasons. The most basic is the idea that particles do not exist in the way we can imagine. They exist in multiple places at the same time[not really but useful analogy], simply with different probabilities(don't pretend to understand, nobody does, they just prove the math and accept as is.) You know what, that analogy was not enough. I will explain in more detail. Particles do not exist(not really). Simply quantum fields exist. The electron field for electrons, the photon for photons, and so on. The particles are not particles as we imagine, but excitations within the field. It is a bit more complex than that, they can act as particles when observed but that is a whole other can of worms. Now back to the matter at hand, now waves act in certain ways that is hard to explain. Due to their quantum nature they must be decomposed first through Fourier modes to under stand more clearly. This is an example of a very simple form:
$$
\psi(x) = \frac{1}{2\pi} \int \phi(p)^{ipx}dp
$$
\par Now I won't explain the equation in its entirety but I will say this. That you cannot solve it for an absolute certainty of position(not that you could get a "true" answer even if you could given that is is a Taylor Mode and not an analytical equation). Now you may think this is just a quirk of the math, but it is not. It is real for reason that could take up the entirety of this book if I truly understood them.
\par Next, quantum mechanics goes against causality through entanglement. Now what do I mean, when people refer to the speed of light, they are actually referring to the speed of causality. To explain this is relatively simple with a good understanding of Special Relativity. It is obvious that the universe is invariant to position and velocity(this has been previously proven thoroughly through experiments in high speeds, specifically through electromagnetism but others can apply), but for this to be true there must be a finite constant cosmic speed(because it requires a Lorenz transformation[there is more explained why this si the only answer but I will leave this hear because I have much much more physics to talk about in a philosophy book.])  This is the speed of causality, the speed of information. If something were to go above this speed it would in essence go back in time. This can be proven rather simply through simple algebra:
$$t'=\frac{1}{\sqrt{1-\frac{v^2}{c^2}}}(t-\frac{vx}{c^2})
$$
\par From here you can look at the square root and easily find that if $v>c$ then $\gamma $(the value of the first part with the $\frac{1}{\sqrt{1-\frac{v^2}{C^2}}}$ would be imaginary. Thus the t value(of time would then become negative, this obviously breaks causality.
\par Finally, back to quantum mechanics, as you know entanglement transfers information at faster than the speed. This seems to break causality. 
\par Now I could keep on talking about arguments between quantum physics and causality, but I will only talk about one more. Virtual particles. Now first of all, what are virtual particles. Virtual particles are non-existent but existent particles that exist within the math of quantum field theory(and thus above my pay grade) but I will attempt to explain it. The idea in essence is that due to the uncertainty principle there exist small amounts of excitements within their fields. These excitements are not enough for an actual particle to exist in the way we imagine, but occasionally a particle can appear, primary through the use of "borrowing" energy from another particle or through creating a negative particle(not to be confused with anti-particles those are negatively charged, these have negative energy/mass) These virtual particles are especially important to particle interactions, what I mean is the fact that when two particles are "close" enough together they excite the fields around them to create bosons(force carriers) to cause the actual interaction. As an analogy, imagine two electrons, when close enough together they excite the fields around them to create a virtual photon, the photon then interacts with one electron(and also effects the other electron negatively[it is almost impossible to explain]) and cause the electromagnetic repulsion between the two. Now, back to philosophy, what in the world does this ramble have to do with causality. That is a great question. It is that all of our interactions are based upon something that isn't real. Electrons don't actually create virtual photons, these photons don't create negative photons. Despite having a tangible effect, they don't actually exist. How can something cause another thing if when you break it down to its component parts, particles are behaving 'irrationally' being effected by mathematical objects.

\par Now after I have listed the critiques, you are probably very aware about how you assumption of the obvious state of causality was intellectually lazy, but you probably also realize I wouldn't have written all of this unless I have an answer to these critiques, or at the very least a possible answer. In all reality I could be wrong about this and causality could be false. Though I am never wrong, so that must not be true.

\par The uncertainty of particles doesn't necessarily falsify causality, it simply evolves it. We previously see causality as something that is caused by a specific thing, but it can be caused by a probability of things(if that make sense). 

\par Next I will talk about virtual particles, and yes I hear you saying that I should go in the order of which I introduced the ideas, but I do not care. Now, virtual particles. How does causality work with things that are not 'real'? Now the most astute among you might claim, why would causality not apply, and you are exactly right! You all may feel a sense of betrayal,  that it was simply a trick of the light. But now, listen carefully, this is how philosophy works. Though asking questions that are hard to understand, questions that play tricks on our language, on our mind. Though these questions we refine our thoughts. This is the problem with modern philosophy, they confuse critiques with criticism. While criticism has a place within philosophy, it should not be as big as it is now. Now before I continue I will define why I consider a critique and a criticism as different things. A criticism takes a part an idea, concept, action, or really anything for the sake of taking it down. This can be justified if such ideas have no worth and the criticism is an attempt to prove it. A critique is a question used to refine a thought not destroy it.(though if the critique can not be resolved then it may.) This critique of virtual particles is not supposed destroy causality by any means, but like the earlier critiques, it is meant to refine our thoughts. It asks us to question what a cause is, what can be a cause. Can virtual particles be causes, modern science suggests yes, but this requires a refinement of our concept of causality. Of what is 'real.' In a later section I will delve into this in a future section, but now I will leave you to ponder.
\par Now finally, to entanglement. Does this refine our definition or break it? It breaks it, but due to the evidence surrounding causality, including the fact that the special and general relativity are simply logical derivatives of the idea of a finite special to causality suggests that our idea of entanglement is flawed not our idea of causality. Now what is the answer? In short, nobody knows. There are many theories trying to come up with a solution. These include Einstein-rosen bridge, super-determinism, and local hidden variables theory. Though since these ideas are both extremely complex and offer little beyond the problem at hand I have elected to move on. Now you may say that my Faustian arrogance has lead me astray to ask questions I can't answer, and to that I say... No dummy.
\par Now you may think we are done, we are not. We haven't even mentioned the nature of God in causality and the teleological nature of Hamiltonian mechanics. There is also further refinements of time which must be discussed to even further define causality. 
\section{Divine Law Theory of Causation}
\par You thought I was done with with causality. You thought wrong. 
\par Now, what is the Divine Law Theory of Causation. It is very simple, it is that God predefines the laws of physics and nature. This differs from the ideas of occasionalism. Occasionalism is that all actions are the conscious decision of God, that the uniformity of the laws of nature is simply a coincidence. 
\par Now, why do I choice Divine Law Theory of Causation? While I do not attempt to venture into the idea that I can even begin to comprehend the machinations of the God most high. Though in my mind the idea of predetermined laws make more sense, a more efficient method for God. Though this is not my primary reason for this belief. As I will mention latter in this chapter I will explain several beliefs of my predicated upon divine law theory of causation. Also, it is more in-line with scientific materialism
\section{Material, Immaterial, and Reality itself}
\par Now we are getting to the meat of things. What makes something real? What makes something itself? How do we define such things?
\par These are all great questions that I will go through one by one.
\par I will begin with what makes something real. While it is easy to look at things in the materialistic view, that of which that can be observed and have an observable effect is "real." No this has some implications, what about what about God(this is based on my assumption that he is 'real') but more importantly what about things that have an effect but of the concept of virtual particles. 
\par First, the immaterial. Given the fact that by definition, these things are beyond us I will ignore them because we cannot gain any intuition from it. Just remember that it must be included within the concept of real, simply we are unable to define it accurately.\\
M= Material\\
I=Immaterial\\
$\mathbb{R}= Real$
$$
M \land I\subset \mathbb{R}
$$
\par Finally, back to virtual particles, I have kept you in suspense enough. How do we satisfy the existence of virtual particles? Just as we have several times before, say it with me, we must refine the definition. Now, how to do this. We must first realize that at the subatomic level, no particles are particles in the way we describe it, as I mentioned earlier they are simply excitations within a field. They are energy in a specific manner. rom here we can divide the material into two types; true material and partial material. The true material refers to what we generally consider matter, and the the partial(photons, gluons, and other massless but still interacting objects.) Which leads me to my final definition; material is that with observable effect[observable does not just simply mean seen, it can be any form of observation].
\section{Identity}
\par While many have abstract ideas on the identity of object, I will simple say that all subatomic particles are identical and thus it is not hard to define, and the others are simple as humans(another word defined by humans) define it. Good day gentlemen
\section{SpaceTime}
\par Space and time, the two things most clearly and least clearly understood at the same time. Things that we know do to our interactions with it every second(ha-ha) of our lives but still don't truly understand.
\par Many philosophers have questioned the reality of space and time. Are they simply human made? I would suggest not, the mathematical empiricalism and importance of them in higher level physics points to their real nature. 
\par You can see this clearly through special relativity, there is an implied "proper" time at which things like electromagnetism operate correctly and without it our modern view of physics collapses.
\par Now you may ask about quantum mechanics, where such ideas of concrete space and time become fuzzy. Where particles exist within multiple place, some theories suggests 'time travel"(not real time travel) and other. Though, as I have mentioned before, I think his more evolves our ideas matter than our ideas of space and time. Also, there are examples of the objectiveness of space and time through things like quantum field theory(combines special relativity) and the Plank constants. You can't have constants of space and time without an objective reality of space and time. Now, what is the Plank length and time. They are the units of which distance and time become quantized, and some suggest the smallest units possible. Now, that seems like an odd idea, especially considering how much smaller it is that anything real and the random constants within it $c,G,\hbar$. Speed of light(or causality), gravitational constant, and reduced Plank constant. This is because as you get smaller and smaller there is a new uncertainty, a 'plank' uncertainty. Beyond this the uncertainty becomes absolute. That the realities of a new quantum gravity take hold The easiest way to look at this is through finding at what point the wave-length of light(our primary method of observation) becomes a black hole. 
\section{Interpretation of Quantum Mechanics}
\par The long awaited further explanation of my views on quantum mechanics. Throughout this chapter, I have hinted at these ideas of interpretations of quantum mechanics. Also, throughout this I have primary relied on the Copenhagen interpretation of quantum mechanics. Which may lead several into seeing this as my view of quantum mechanics, it is not(I will elaborate further later.) Though let us first go through the different interpretations of quantum mechanics.
\par The Copenhagen interpretation. In essence the Copenhagen, the most popular view among scientist, is  that quantum physics is impossible to understand through our classical views of metaphysics. That our theories of quantum mechanics works, and to ponder further is a waste of time given its impossibility. It takes our ideas of wave functions and entanglement as reality, not because it is real reality, but because the model works. That at the end of the day physics is not the study of reality in the way metaphysics is, but the study of models of the universe. 
\par Though this explanation isn't really fair, so I will expand upon it. The key concepts are things like wave-particle duality, that particles are both particles and waves and that things like observation can effect how they interact with itself(yes particles interact with themselves) and with other things.[Though observation is never truly defined] Quantum superposition, something we all know fairly well and requires no further explanation. That quantum physics is probabilistic by nature, and truly probabilistic not governed by hidden laws berried beneath. There is a lot more to it but given that I have already explain part of it previously and how popular it is, you can fill in the gaps on your own. While this theory has been extremely important for our modern views, and important to explaining how quantum mechanics works without esoteric 'nonsense' it has a couple flaws like observer dependence and the fact that it quite literately states that it is not the ontological truth but rather a model.
\par Next, the many worlds theory. One of the popular interpretations by the general public. Basically, it is that when probabilities collapse; whether it is an interaction, wave-particle duality, superposition, or many of the several other things; the universe splits into two(or however many are used to describe the 'whatever'). That one universe has one and the other has the other. This is quite outlandish idea; for its reliance on the observer and the fact that there is no mechanism of 'universe splitting.' One thing I would like to add, is that I have always hated how people use the idea of a multiverse in this context, it is multiple verses it is one with multiple observable components. A better term, generated by Sir Roger Penrose and a Latin Professor that he was friends with(which I think is absolutely amazing, he made up a new word because he didn't like the word being used) he now calls is an omnium(meaing of all)
\par An even more outlandish theory is many-minds... which is exaclty what it sounds like... yeah
\par The Pilot wave interpretation is that there are hidden variables; that just as things like coin flips, brownain motion, plasma instabilities, and so much more seem random from the outside, if you really break them down they aren't. Now what are these hidden variables. Now what is this hidden variable, you might ask. Well lets take a step back and explain a bit deeper. First, lets define a quantum wave function, this is different to the excitations of quantum fields mentioned earlier. The wave-function is a complex variable(I mean complex as in in the complex field not challenging to understand) that depends on the particles position and can be used for several things. It can de described:
$$
i \hbar \frac{\partial \psi(r,t)}{\partial t}=(-\frac{\hbar ^2}{2m}\nabla ^2 +V(r))\psi (r,t)
$$
\par where $\hbar$ is obviously the reduced Plank constant. r is the position vector. $\nabla ^2$ is the Laplacian operator. m is mass. v is the potential energy. t is time.
\par Once you have the wave function you can do several things with it, square it for probability distribution, collapse the wave function, etc. But I am getting away from myself, you want philosophy so I will get back on that.
\par Basically, pilot wave theory suggest that this wave function influences this particle more than we think, and thus can lead to the particle's behaviors being deterministic through equations like this for velocity, $v$:
$$
v=\frac{j}{\psi}
$$
\par Now, in all reality the equation is obviously bigger; must expand $j$ and $\psi$, but I didn't want to write it out(also there are expanded versions for relativistic and whatnot.) though back to what we had in mind, what did we have in mind, oh yeah super-determinism. Basically, quantum mechanics is deterministically not probabilistic. 
\par One final thing I will say about pilot wave theory, is that some people suggest that the many worlds interpretation and pilot wave theory are one in the same just describing from different points.
\par Next, we have environmental decoherence theory. Sadly, I know very little about this one at the moment, I will attempt to learn more to better understand it better(for its own sake and given that it could convince me) But it is basically that like how a coin isn't truly random, neither is the quantum world. That there are density matrices made by interactions that effect future actions leading to a theoretically deterministic world.At the moment I don't really understand the mechanisms so I can't prove, disprove, critique, or criticize.
\par Lastly, Objective collapse theory. The objective collapse theory is similar to Copenhagen, with it's more literal view of quantum mechanics. Though, it has two differences. It has a true literal and ontological belief that goes further into everything within it. Also, it does not rely upon the observer. It relies upon 'random' collapses of wave-functions, such as gravity, scholastic field, or the interactions of other particles.
\par This is all well and good, and I do truly suggest you try to figure it out yourself(even though you don't exist), but this is a book MY philosophy. So what do I believe you might ask. I find the objective collapse theory to be best with the pilot function being a close second. Now why do I believe this?
\par First, for reasons I will explain later; the idea of non-determinism is important to other theories of mine and thus I need it for them to function. Though I am sure you all don't think that is enough, so I will go further. Second, in all reality given we cannot empirically prove any of this it doesn't really matter. Three, given that it doesn't really matter, objective collapse theory makes the most intuitive and philosophical sense to me intuitively. Four, you know what, I give up(for know) these justifications are kind of weak, even in my eyes. I plan to do more research and come up with further justifications. Right now it is basically which sounds right.
\section{*Other Questions in the MetaPhysics of Quantum Mechanics}
\par ...(will update later)
\section{Quantum Consciousness,  \\ Arminianism, Super-Gödel \\ thinking, OR Orch, why I even decided to write this book, and how long can I make the title for a section within a chapter}
\par As much as I love how hilarious the title is; with its length, random buzz word, and joke on the end. I also kind of hate it; quantum consciousness seems like some esoteric nonsense. It feels like I am admitting to believing in a specific theory just to justify my religious beliefs, but this chapter will attempt to prove that that isn't the case. That it is rigorous, while maybe not on a true empirical scale, at least on a formal logic one.
\par While there are many reasons for the creation of this book, the main ones is to evolve this very idea. It was that I had this idea at one point and it challenged a lot of previous thoughts, thus I wanted to evaluate my philosophical thoughts from the ground up, then apply that.(While it was the original reason, I plan to continue this book much further and continue it.
\par Let me first explain what my theory is in broad strokes. It is the belief(more so a belief than a full theory given its unprovable nature, unscientific background, and basis on the belief in an omni-potent God unconstrained by the confines of physics, formal logic, or even abstract-mathematics.) It is that our consciousness and free will is preserved by God through the use of quantum mechanics. This theory in it-of-itself in not groundbreaking, original, or even scientific in nature, but I intend to go about explaining some possible mechanisms of this(though sadly, by definition this theory will be unprovable, which I will explain why in a second.)
\par First, is there any justification for this idea of quantum consciousness? There is some, it is weak but existent nonetheless. It is related to Gödel's incompleteness theorem. Gödel incompleteness theorem has two part. First, "For any consistent formal system F that is capable of expressing elementary arithmetic, there exists a statement G in the language of F such that if F is consistent, then G is true, but G is not provable within F." The second is, "For any consistent formal system F that is capable of expressing elementary arithmetic, the consistency of F cannot be proven within F." This claims that formal logic has its own limits; you cannot prove an axiom with itself and you cannot prove a derivative with an axiom alone, you must prove such axioms with other axioms(because you cannot prove an axiom with itself.) Now what does this have to do with free will, well to have free will you must first have some form of consciousness, something outside of our ideas of formal logic. We have some proof of this, the fact that we are able to derive pure math at all seems to prove that we are not algorithmic based, something beyond. Now there are a lot of arguments against this, namely that our ability to do mathematics comes from our informal thinking rather than beyond-formal logic thinking. In all reality, I like this explanation better, but I place Penrose's argument of consciousness here as a sample of somewhat scientific rigure rather than simply religious thought.
\par You may now be asking how this super-Gödel thinking proves free will. It doesn't, but it implies as deeper idea of consciousness allows us to theoretically side-step determinism in order to 'allow' free will. Now, before we get into how God can protect free will, and why I think he would, let us first explain some possible mechanics of explaining this seemingly unexplainable phenomena. Though before that, let me say the fact that by definition it goes beyond Gödel's theorems, we by definition can't prove it so take all of this with a grain  of salt. It is an idea, not a law, not even a true theory, and idea.
\par Let us first go through the most popular explanation, the reason I even came up with this idea. Sir Roger Penrose's OR Orch(I seem to be mentioning him a lot) Essentially it is that microtubals, a potion of the brains neurons, and theoretically a part of consciousness, are highly subjective to quantum 'strangeness' in a way that most other complex structures aren't. That the interactions of quantum mechanics can somehow safe guard conciseness and allow super-Gödel thinks and even possibly free will.
\par While this idea is specifically for objective wave collapse theory, it also works for things like environmental decoherence and kind of work for many-worlds, but not for Copenhagen and pilot-wave.
\par For Copenhagen, no extra theories are required, by definition the idea of consciousness in 'beyond science' and thus no extra loop-holes are needed.(they don't specify what an observer is so therefore other explanations are possible, but they hold no further ground than this one.)
\par Now how does quantum 'brains' lead to free will. It doesn't, not in the most literal meaning, but it does give us extra wiggle room. While most modern views see our view of determinism as falsifying free will, but these theories give us room to say that there are other possible answers; dualism, God's choice, beyond material, and others.
\par I would say that God, by some manner, protects free-will. Whether it through some hidden ideas of quantum mechanics or it is that our consciousness is beyond the material and effects the material through the quantum world. I don't know, as I mentioned earlier, I am not even confident it is correct. It is an idea.
\par Now you may or may not be asking, why would God use quantum mechanics to protect consciousness, if by definition he is beyond formal logic and thus wouldn't need such things. While I cannot comprehend the machinations of God, I would suggest that a possible explanation would be to create our logically, self-fulfilling world that we are able to comprehend(for the most part) and live in a true free will. 
\par Now to end, I want to reiterate how theoretical and disconnected from regular science this is, most of my ideas are at least respected by a decent portion of the scientific community, this is not. It is an idea, nothing more. This is an exercise to see if it holds weight, it currently does not seem to.
\section{Compatibilist Free Will}
\par Even though I wrote a whole lot about quantum libertarian free will. I don't necessarily believe in that. I view the world as either semi or super-deterministically and that that doesn't actually effect free will. Though I still wanted to examine the idea in more detail nonetheless.
\section{*Super-determinism, semi-determinism, Retro-causality, and Chaos theory}
\section{Role of God}
\par I have already gone over the role of God in many ways; creator of the universe, its laws of nature, protecting free will, but now I wil go further.
\par The first question is the relation of miracles, in the Christian faith, we believe that our God is a personal God that creates miracles for us. Now how does he do that if he predetermined the laws of nature?
\par Well that is fairly simple, God is omnipresent, meaning he not only knows what you are going to pray for before he created the universe, but he also knows how to create such miracles from the creation of the universe. Thus he 'created' the miracles when he created the universe itself; with knowledge of all that will ever transpire.
\par How does the divine foreshadow effect our idea of free will. Well, God is beyond our free will, he is beyond whatever quantum mechanisms protect our free will, so he can know without interfering with our free will.
\par Is something good because God says it is, or because it is objectively. It is because it God says it is, beyond God there is not mechanism for the creation of truly objective morals, there are logical models based upon subjective values, but they cannot stand upon themselves. God makes them right because he is beyond formal logic.
\section{Post-Axiomatic Nature of God}
\par I guess pre, would be the better prefix given what I am about to say, but Post-axiomatic sounds better. Now back to the actual concept.
\par This idea is simple. That our axioms, as later examined are derived by God rather than God being subservient to the nature of these axiomatic principles, "A line is the shortest distance between two points," "Existence exist," "consciousness in its totality," "A is A," and many more. He, God, derives them. Now this twists our mind given that axioms by definition cannot be imagined without their existence. It is like inventing a new dimension for us to see, it doesn't exist.
\par This concept enforces many ideas already present; God omni-(something that cannot exist with current axioms), God being beyond comprehension, that God is self defining(see non-infinite definition of words and axiomatic semantics), and being outside of causality, outside of formal logic. He creates existence(something that makes no sense in our current axiomatic understanding of 'existence'), he is three in one(against A is A), and much more.
\par While they don't 'solve' these questions in the traditional sense, they open the door in a more esoteric sense. While by definition we can never understand these concepts intuitively they do give rise to much of our understanding of God.
\par On to our esoteric identification(I know must of what I write is intended to be rational and examined this is one of the exceptions.) This idea of pre/post-axiomatic identification leads to our idea of faith. God, isn't just beyond us, he is ineffable and unprovable though our axiomatic thought processes. Beyond that, this further goes into my idea of the "non-material." We can further define this as that of which is based upon our known axioms and that of which is not.
\section{Metaphysics of cosmology}
\par While I could go on and on about physics of things like the big bang, entropy, arrow of time, anthropic principle and how they could theoretically effect our views of metaphysics. I don't need to, my belief in God nullifies such arguments. That all possible incites have already been satisfied by God and thus have little need for exploration.
\section{Ontology of Information and Entropy}
\par Once again, we must first define what our concepts are rigorously.
\par A classical perspective. Within this concept information is the physical quantities that en essence give you traditional information. That allow you to decode past events and even future ones. Entropy is the inverse of this affect. Geometric instabilities and 'randomness' that hide 'information' from observers. 
\par From a quantum perspective; quantum information even includes superpositions, wavefunctions, density matrices, and more. The quantum bits, qubits, can exist and superpositions and exist within the same quantum state(position)
\par In this view, entropy then becomes a byproduct of entanglement, a process that 'hides' the information from view.
\par There is much more on the semantics of this. Going into Geometric identities, black hole information paradox, and quantum thermodynamics. Though, for now I will move on to the more metaphysical parts.
\par One common debate is if information is truly fundamental or simply a human made connection. That the universe has a 'computational structure' where information is the substrate and entropy the measure of the complexity. With modern understanding of the geometric identities and the fundamentalness of information, the theory of fundamentalist seems to take the lead.
\par Next is the arrow of time. The second law of thermodynamics/entropy states that in isolation, states drive towards entropy. This necessitates an arrow of time. It breaks the symmetry of time. 
\par Now many question whether or not entropy is a by-product of the arrow of time or the other way around.
\par The way I see it, the direction of entropy is the consequence of the already time symmetry breaking existence of dark energy, black holes, and other similar objects in this preview. That the direction towards entropy precludes low-entropy existancess.
\par Though this is just one theory. Holographic principle, Gravitational Entropy, Quantum Mechanics and Unitary Evolution, and more. In-fact the quantum mechanics and unitary evolution makes more sense and less problems.
\par The symmetric theory ignores that some forms of entropy are consequences of statistical physics that doesn't deny time symmetry
\section{Dualism}
\par I have already mentioned this in interpretations of quantum mechanics, but in physics many possible mathematical(and thus metaphysical) interpretation exist. Different ways to look at the world. They can't both be true, can they?
\subsection{*Phase-space and the Euclidean Mind}
\section{*Ontology of Numbers}
\section{Reality and the Abstract}
\par When reading this chapter, one can easily get a esoteric/gnostic sense of reality. Of pure abstraction with little regards to reality. Though this is not the case. In fact I find the abstract "non-material/pre-axiomatic" world as inconsequential. The only true consequence being God, but he matter because he influences the material. The focus on this abstraction being the fact that the ideas of reality are largely explored and are easily understood and so large amounts of writing and exploration on the topic is relativity inconsequential.
\par The focus on abstract, structuralism, post-axiomatic is simply to satisfy my need for cognition(as later explored) rather than being core parts of who I am.  These theoretical ideas that I am not even sure if they are right or not(post-axiomatic God, Consciousness, etc) are explored not because they are core to who I am and what I believe(maybe the God one) but because it is fun.
\section{*Questions and Debates in Modern Metaphysics}
\section{*Conclusion}
\chapter{Epistemology}

"Man is neither infallible nor omniscient; if he were, a discipline such as epistemology—the theory of knowledge—would not be necessary nor possible: his knowledge would be automatic, unquestionable and total." | Ayn Rand
\section{Limited Objectivism}
\par Before I begin upon explaining this, I must first specify the influence of Ayn Rand is limited to Epistemology. It has not influenced my politics nor ethics. Now lets begin with the philosophy.
\\
\\
\par In a short simplified sense, objectivistic Epistemology is the inverse of transcendental idealism(though despite what many say, not the opposite). Both philosophies take both reason thought and empirical data as useful tools in the accumulation of knowledge.
\par Though they differ in importance and direction. Transcendental idealism takes reason as the most fundamental and such abstract thought can then be applied to the outside world through empericalism. Objectivism takes the inverse, that we create our abstract models based upon our interactions with the world. Now these may seem like small differences, but the expand themselves through their logical conclusions. I could explore how they differ in more detail, but I will focus on objectivism.

\par Now, where did the limited part come from. As I have mentioned before, I am a religious person. I believe there are something beyond our sense and possible our mortal comprehension, therefore we are limited by such things when acquiring knowledge. I do believe what we are capable of discovering is 'objective' and just as metaphysically true as what we can't, but that does not take away the fact that some abstractions are beyond us.
\section{Consciousness and Reality}
\par A common continental question, is the question upon whether our consciousness truly models objective reality. While these questions can be useful when directed in a purposeful and specific way; though most people don't direct it in a useful way, but rather in a simply abstract way with little refinement of knowledge than the ability to trick others through strange and useless question. 
\par Now, why do I think this is a useless question? Well, for several reasons I will address categorically. Now I will prefis this by saying this is a criticism, not a critique, I may later may later add a more refining critique but for now it is an acknowledgment of the stupidity of this, and to show that all derived concepts have no true analytical framework.
\par The most basic argument against this is the pragmatic thesis. In essence, it doesn't matter that reality is consistent with the observations of the mind, the mere fact that our observations closely associate enough for practical purposes is reality enough. There is no purer form of reality than this functions.
\par Though, this isn't enough for me, I think it could be enough if it was all there was but I want more. I do believe in truly objective reality and not just our semi-subjective observations of reality.
\par A refinement of the pragmatic argument to include the existence of an objective world would be a question of statistics. If a further world with separate physical laws existed it wouldn't work with our vast and extreme observations. Though, this wouldn't work with some more extreme theories wouldn't work with more radical and complete ideas, like Recreant's dream theory. Though, this implies that whatever objective reality is it must not be capable of interacting and effecting our reality. This can be attributed to nothing that could interact with our world, or our world is independent of the interactions of reality by its own definition. This may seem like a minor and obvious reduction but nevertheless it such reductions are the essence of philosophical discussion.
\par Next, such dream theories are useless. Derive no further discussion, only exist as a way to trick people up and thus serve no purpose. For these, the original pragmatic argument is all that can be derived secularly.
\par Now, you may say this is a good way to question our scientific premises. An argument against scientific realism, but in reality it doesn't change anything. Scientific realism stands or doesn't regardless of the 'validity' of such theories. 
\par My final secular argument is that such philosophical arguments are harmful. That philosophy should not start with doubt but rather with perception. Such perception must be doubted, but doubt existing in its own self-containerized munition is useless. It defines nothing and leads to strange and possibly dangerous ideas. For instance Desecrate lead to Spinoza who lead to Hegel, who lead to Italian idealism, who lead to Giovanni Gentile the philosophical founder of fascism. Now, you may criticize that this was one complex branch, you could also derive similar lines through Aristotle or just any philosopher far back enough, and this is a good argument. Though, there is a clear line, also to other dangerous ideas like communism, NAZIsm, and so much more. These things are predicated on these strange cartesian/continental questions that derive no true thought, and thus should be understood in this way. Simply as pathetic and dirty tricks and shouldn't be see as true paths to knowledge. While yes, we should still engage sometimes, truly placing it on equal footing(or higher as we have been) is dangerous for our collective understanding reality and derived sense based upon it.
\par Though, many of you may have noticed I specify secular. There is a religious argument as well. If God created the reality as reality, then it is by definition reality. The only argument otherwise is that God doesn't exist(either entirely or he is not God)

\section{Nature and Validation of Axioms}
\par First, like always, we must define axioms thoroughly. The dictionary definition is "a statement accepted as true as the basis for argument or inference." For instance, our understanding of scientific realism is the understanding of the fact that our observations of reality are accurate(there are some observations that can be morphed due to other things that are observable.)
\par A simplified and more observable definition can then be defined to say that an axiom is a statement that either cannot be proved or cannot be proved currently that is used within a greater statement that can be proved upon its edifice. These axioms, while don't require 'proof' generally desire some type of reasoning even if it is not objective or purely analytical in nature. For instance, the above arguments for the objective nature of reality aren't true objective proof, they still exist as semi-arguments. While we cannot prove the objectivity of reality with true and objective reasoning, some reasoning can be applied to get some kind of 'proof' of the statement so that further studies can be applied based upon that said axiom.
\section{Non-infinite Definition of words and Axiomatic Semantics}
\par In the field of axioms, there becomes a problem. Part of the reason that 'some' words cannot be further defined. They are defined by themselves, they can only be intuitively understood, and even there is a limit.
\par A great example is existence, because to define this thesis, you need an antithesis. That doesn't exist(non-existence by definition needs to exist to be something and thus cannot exist because it would need to exist to be in existence.), so therefor existence is defined by itself.
\par This brings up a great deal of problems, paradoxes, and much more that will be explored.
\section{Induction \& Deduction}
\par In the expansion of human knowledge there are two main ways in which they can be reached; induction and deduction. 
\par Deduction is by taking premises/axioms and derive conclusions based upon this. Induction is the opposite, from observing 'effects,' define whatever knowledge is desired. 
\par This can be seen in physics, deduction is taking a fundamental theory and applying it in a new way to find novel uses, predict, or whatever may be needed. Induction would be taking data to come up with a way to model reality or come up with an entirely new theory. 
\par Both of these are extremely important. Without induction we could never go beyond basic logic, but without deduction we can never find objective truth. The way I have found to be the most useful is to use induction to attempt to find our 'axioms' and then use these to deduct the reality around us, though this is only really works for my fields of interest. In other fields, like psychology this strategy would fail.
\section{*Reason, Empiricism, and intuition}
\section{The Infinite Pursuit of Truth*}
\par Nothing can be proven true. As controversial of an idea as this may seem, it isn't. This basic axiom is the foundation of modern scientific epistemology. That theories can only be proven wrong, never correct. That is why they will always remain theories.
\par Now, you might think something as foundational as something like Newton is true. Though, there are questions through MOND(Modified Newtonian Mechanics) and others. Given, Newtons empirical evidence, there would need to be a lot of evidence for MOND for it to be taken anywhere near the same level 
\section{*Wesleyan Quadrilaterals}
\section{*Virtue Epistemology}
\section{*The Burden of Justification}
\section{*Structure of Knowledge}
\section{*Reductionism and Constructionism}
\section{*The Problem of Priori}
\section{Other}
Small essays on random topics.
\subsection{An Axiomatic Induction into Non-Equality of Correlation and Causality}
\par "Correlation Does not Imply Causation," an idiom commonly used since the early twentieth century. Yet despite its age and apparent common-sense appeal, the depth of its consequences are often overlooked. When, understanding this fact is crucial to our pursuit of truth in a way hard to ignore. Thus, an axiomatic examination of causality and correlation is required. First through a preview into the required axioms, then through the use of logical induction finding the natural consequences of said axioms.This is established due to the fact that rigorous and systematic definitions lead to greater understanding of a concept whose consequences are generally lost.

\par First, to begin any rigorous examination, a preview into the basic required axioms is required. The primary axiom is that causality exist, that some kind of 'cause' or action leads to an effect, and that then by definition these will be correlated. The next is that we exist in a world with various physical causes and observable effects in a complex multi-dimensional way. This complexity leads to multiple causes producing the same effect, a single cause producing multiple effects, and relationships that are entirely disconnected. It also permits recursive causality, in which two events can cause each other.

\par From these basic axioms, a myriad of possible cases of which that correlation does not equal causation. The first being reverse causality, the basis of this idea being that the first axiom of causality being divided into cause and effect creates a directionality. Then, based upon the second axiom it is very logical to understand that there exists non-recursive cause effect relations, making it so that if someone were to make the conclusion that B causes A because of correlation potentially wrong given that A causes B. Next, the common-causal variable states that given the second axiom, that one cause can have multiple effects, this lead to the possibility of C causing both A and B and thus A is not the cause of B but rather C is. Next, bidirectional causation is a consequence of recursive relationships where both A and B are causes of one another, so the statement that B is the cause of A is true but not nuanced enough. Finally, The relationship between A and B is coincidental; given the complexity of the examined world, coincidences can exist. Correlation can be caused by poor sampling or simply chance. These examples show that correlation isn't enough to prove causation due to the numerous other potentialities where correlation does not prove causation.

\par The statement that "Correlation Doesn't Equal Causation" isn't simply a term that that exists without logical basis, but rather a natural consequence of the physical world. With only two axioms; with one being that causality exists and the other that being that the physical world is extremely, through these basic axioms four different examples are clear consequences of these axioms. In summary, a rigorous definition of the common idiom has been established to create a deeper understanding.

\section{*Conclusion}
%%%%%%٪%%%%%%%%%٪%%‰%%%%
\chapter{Ethics}
%%%%%%%%%%%%%%%%%%%%%%%%%
"Finally, brothers and sisters, whatever is true, whatever is noble, whatever is right, whatever is pure, whatever is lovely, whatever is admirable—if anything is excellent or praiseworthy—think about such things." | Philippians 4:8

\section{Introduction to ethics}
\par Many that know me, know the question I love to ask, "what is your moral philosophy?" I love this question for many reasons. First, it intrigues people, they love to talk about their own morals and beliefs, but engages their curiosity with these newish words. Second, Morality is something that is both simple and fundamental enough that everyone has thoughts of it but still complex enough that it can have engaging discourse. Three, it get to explain stuff, most people haven't heard of virtue ethics, deontology, egoism, etc.

\par Despite this fundamentalness to ethics, many questions are still unanswered, which fundamental theory? Many questions are raised for each theory that must be addressed? Even people question if Morality is even objective or not? 
\section{Virtue Ethics}
\par My fundamental theory is virtue ethics
\par Virtue ethics is based upon the idea that Morality is derived from living a virtuous life rather than the adherence to rules or a logical stance. That you must pick a set of virtues and live by those principles.
\par Now why do I choice this over the other types of theories of morality and ethics. To understand this I will first criticize(yes criticize not critique) other theories then I will defend virtue ethics.
\par First, deontology. Deontology comes from the Greek word deon meaning duty. That you have a duty to follow certain 'rules.' That morality is based upon a set of rules and concepts rules and concepts to follow. Say, don't kill, don't steal, give to others.
\par My problem with this is not that it is too ridged, but that it is impossible. It would be impossible. You can't make rules for all possible interactions so must theories of deontology tend to just be contorted virtue ethics. Also, that both the action and consequences should be included when making an ethical decision but generally rules can lose this complexity.
\par Next, egoism. Yeah no
\par OK, fine, I'll do a little more. Most theories of egoism fall into either three camps; nonsense hedonism(which needs no criticism), logical consequentiality based concepts, or altruistic egoism. First, the consequentialism type eventually leads to normal consequentialism but with more steps to find the conclusion so I will focus on that when I get to it.
\par Next, consequencialism. Frist, for consequencialism to work a set of values must first be prescribed(so it isn't fundamental) there are others like utilitarianism but they aren't self-justifying nor truly justified by other means. There are some that exist but I will show that to analyses all in-depth is a waste of time. My reasoning is that both actions and consequences should be analyzed. Why, good question... Well, first there is religion obviously. Though a more secular reasoning is that ethics is used as a binding agent, people emotionally don't bind well to 'immoral' actions, and those that do tend not to bode well with other things.
\par Now for justification of virtue ethics. The first and most important is that I am a Christian, from my analysis most Christian ethics is highly tided in virtue ethics. The fruits are virtues. Most sins are vices. Most ethical advice is based upon abstract virtues rather than a simple value(consequentialism) or rule(deontology). While there are many rules, most of them are based upon the virtues later mentioned rather than justified by themselves.
\par Next, on a more secular level, virtue ethics are far more complete and level for people to analyze, create, and apply to your life. They can included anything you want to have in them without creating paradoxes and problems(as long at the two concepts are not are paradoxical in it of themselves)
\par I will show this through the future sections that go through a list of virtues and why I feel they should represent my ideas.
\par One final thing I will talk about, is something that may be a strength or weakness depending on how you look at it. Virtue ethics has a much more 'subjective' connentation. A lot of virtues that I value another may not, also these virtues can be analyzed in your own way.
\par The way I see it, some virtues are fundamental and should be applied to all given they are principles derived by God, others I feel as personal virtues and feel no sense that they should be forcefully applied to others(though I feel things would be better for all that way.)
\par In essence, the idea of virtue ethics is to pursue perfection of the mind, body, moral thought, etc in order to create a purity of intention. This is the goal, not absence of sin, but the perfection of moral intentions. To create yourself in the image of your own ideal and to pursue it. That morality isn't just a responsibility but ontological alignment. To pursue the ideal of Jesus.
\section{Stewardship}
\par Throughout the rest of this chapter, many times I reflect on morality as my values. Values that are observed because of some 'internal promise.' This is because it is the way I have naturally thought, though I am trying to evolve past this.
\par I have always seen the internal promise as more pure than any other form of moral enforcement, though I have found this not to be true in the highest extent.
\par This obsession with dignity and following through with my own values alone is cornered in the vice of pride. It sits upon it, being thus controlled by it, and by proxy then so am I.
\par Now I have found, not responsibility, but existence in stewardship. In stewardship to the will of God, his grace, his mercy, his creation, and most importantly each other.
\par You see, God brought Adam and Eve into the world, as images of himself: builders, molders, and beings capable of reason. A creation to tend the garden, name and care for the animals, and to care for each other. This was the reason for humanities existence.
\par After the fall of mankind, humanities purpose shifted, but not much. We are still arbiters of his will and creation, by authority he has given us. Though, because of the imperfections made by us, our responsibility is to fix them. To perfect the world around us through the words that God has authored for us.
\par For God himself has given us this divine commission; though he hasn't left us to do it alone. He gives us gifts, hope, and healing. He feeds us and nurtures us so we made be sanctified in his love for us.
\par Now in practice, this becomes our purpose of moral perfection. Both for ourselves and for those around us. To create, not only pure intentions for ourselves but for the society and people around us. I will discuss this more throughout this book, especially this chapter, but here I will leave with this.
\par Though our actions, we may be able to co-create with God to bring Heaven on Earth. Bring about social holiness caused by our own actions with others. This is the purpose of Christianity, of ourselves. Only secondary to a relationship with God.
\section{Categorical Good}
\par Based upon my religious beliefs I propose this axiom, "that all of reality is designed in such a way in that "good" can be logically inferred and the actions to find this good is what we call virtue." 
$$\forall x( R(x)\rightarrow \exists P(I(P,G(x))) \land \forall A(V(A)  \iff \exists y(L(A, y) \land G(y))))$$
\par In essence, there exists a good for all categories(ex. communication, get information delivered clearly). Thus, there are good ways to do this, virtue(ex. use precise language when needed but don't use overtly complex language when not). Finally, not only can iterations of this build up an idea for how to "be a good person" based upon a complex model of different circumstances throughout life, but that what virtues exist reveal what makes a person "good" by what God has consciously divined, giving the actor "wisdom". This is all centered around the Stewardship given to us by God as mentioned in the previous section.
\par Obviously, this isn't a purely religious view. Aristotle came through with an idea very similar idea, but even his idea had a pantheistic view, and other secular versions generally make it an unquestionable axiom without further backing other than the fact it creates a non-paradoxical system. Also, this system allows further development, like founding it in stewardship or having all things reveal "wisdom" to further develop virtue.
\section{Cardinal Virtues}
\par Now, while my cardinal virtues have basis in the traditional cardinal virtues of the catholic faith they are not the same. The name comes from the fact that cardinal comes from the Latin word 'cardo' which means hinge. In essence all other virtues hinge on these concepts.
\par Here is a short list of the main ones
\begin{itemize}
    \item Temperance: Excess in most matters proves detrimental; self-control is paramount in one’s life.
    \item Prudence: Deliberation is essential before action in all circumstances.
    \item Fortitude: Mastery over one’s emotions is a paramount virtue, as the inability to do so inflicts harm upon oneself and others.
    \item Faith: Maintain strong belief not only in God but also in oneself and one’s values and beliefs; these should remain immutable unless confronted with supreme evidence.
    \item  Duty: One has obligations to oneself, family, others, and one’s values, which should drive one’s life. Duty to oneself includes ambition and adherence to personal values; duty to family involves support and politeness; duty to others is similar but to a lesser extent.
    \item  Individualism: Embrace self-reliance, self-respect, and ambition, and adhering to personal values.
\end{itemize}
\par As you can see all of these virtues are simple and widely acknowledged and don't require much additional justification. The only true exception being individualism....
\section{*Salient Virtues}

\section{The Philotimic Virtue}
\par Now Philotimic is a word I have made up, I will go into more detail about what exactly it means later, for now assume it means pride.    
\par Now, I hear your vapid and lost mind screaming, "isn't pride a vice, what are you doing" Though I will examine why I think pride in some forms is a virtue not a vice(at least certain types of pride)
\par First to examine we must first define pride and what it is. One definition is ,"feeling of deep pleasure or satisfaction derived from one's own achievements, the achievements of those with whom one is closely associated, or from qualities or possessions that are widely admired." or "consciousness of one's own dignity." Now what does this mean and how do they fit together.
\par Let us start with the first one, a deep feeling of pleasure in ones own achievements. This can manifest itself in many ways; work, morality, intellectual achievement, and so many others. Beyond that it can effect people's actions in both positive and negative ways. Let's start with negative, it can lead to obsession with ones own capacity and the product of such. On the flip side, a healthy obsession will lead to personal growth on whatever they feel pride in. Also, it can give them confidence in their own capacity. An example would be with morality, if they feel pride in their own sense of morality it can lead to self-assurance in their own senses leading to them being able to apply and do them, but it can lead to close-mindedness and looking at just your own thoughts and nothing more.
\par Now lets look at the second definition, consciousness of one's own dignity. Meaning that they are aware and influenced by concepts of their own dignity. The belief that somethings are within their ability and others are beneath them.
\par Now what I mean by philotimic, as essentially the positive aspects of this. That to have philotimic virtue is to take your own life seriously. To be principled, to be absolute. Not just in behaving moral in a moral and dignified manner, though this is definitely a part of it. In fact, even within this manner this virtue is partly about fanatical pursuing such moral perfection. Though, outside of this there are many things. Most notably is having such fanatic belief in things beyond morals, other values like being early, being prepared, working hard, being mature, and being other similar concepts. That while there are values that God doesn't arbitrate, but you do, and thus you should keep them to almost the same esteem.
\par This concept, as mentioned earlier is about taking your own dignity, self-esteem, self-respect, virtue, and above all life seriously. Because that is what you should do, take it seriously in a dignified and ridged manner. To never compromise upon even the most minimalist value, because to compromise on such things is far worse than any other conceivable interaction.
\par Another concept within it is control. Control over your inner-mind. Your thoughts, emotions, temperament, and even personality being bent to the will of your own 'ego.' Not ego in the regular sense, but as in the inner consciousness of your own self. The self of morality, values, reason, and more(more closely related to the idea of 'superego')
\par In essence, recursively creating your own self based your higher values rather than letting your environment impact your supposed values.
\par Philotimia in its highest sense, is not merely pride, but the love of what is worthy of love. To find what is worthy of such esteem and pursue it with fanaticism and create your soul around such values. To take responsibility for your own existence, morality, self-esteem, and so much more.
\par An even further simplification is it is taking upon yourself the responsibility to be human. The responsibility to think, to fear God, to live accordance with your values, to have values, and to live in the image of your highest self. Not only that but bend the world into the image of your highest self.
\\
\\
\par One thing I will add, is in the conflict of universal morality(derived by God) and personal values. This conflict doesn't really exist. Both of these values exist in my mind. While universal morality is obviously stronger, this doesn't dissipate the importance of personal values. The largest differentiator is others; for universal morality, others disobeying. is an infringement of morality, while for personal values only you 'break them,' no one else.

\section{*The Internal Promise and Dignity}
\section{Good Life and Eternal Struggle}
\par There are two parts to this. Obviously the good life and the eternal struggle as defined in the title. This are both extremely connected, as I will show momentary.
\par First, the good life. To live a virtuous life is to live a good life, but there is challenge. One, you should train yourself to desire this good and enjoy it. as defined in the philotimic virtue, take this training upon yourself rather than simply passively let culture and external figures train for you. While yes, culture can do a good job, training yourself creates a better feedback loop and consequence; beyond that, training yourself is your own moral responsibility as a human being. 
\par Find this good to be your highest desire, so that you not only can but will naturally follow it and ignore sin and vice around you. To desire it fanatically. While many consider fanaticism to be a bad thing, in the pursuit of virtue, it isn't just a good thing but a lack of it is evil. 
\par Beyond that, this struggle will be enteral. Virtue by its very nature is never perfected. Therefore, you should always struggle. To take this further, you should love the struggle against it.
\section{*Personal Virtues, Values, and God's Morality}

\section{Intentions}
\par In my theory of morality, intentions matter just as much as the action in it of itself. "The LORD does not see as man does. For man sees the outward appearance, but the LORD sees the heart." 1 Samuel 16:7.
\par I could go on and on, but this idea is relatively intuitive and while many don't consciously think this way it isn't far from our minds.
\section{*Morality and Religion}
\section{*The Ethics of Knowledge}

\section{Recursive Connection of Civilization \& Virtue}
\par Our capacity towards civilization and virtue are strictly linked in a depth rarely acknowledged. 
\par Let's begin with the start, for civilization to happen, we must be able to do it well. This is the heart of virtue, while virtue can be connected with any doing any good thing well, the ability to be civilized and do civilization "well" is the very heart of it.
\par Now, how do we do civilization well? To make sure people do not become distrustful, honesty should be brought about. To bring about any positive outcome, discipline becomes a need. To expect pro-social activities, empathy is not only an ideal but a requirement. While I won't explore every angle, it is clear where this leads. Our modern ideas of culture and morality is primary based upon how to build a civilization, a society.
\par Though, how to develop virtue on a mass scale? Society. Through community virtue is developed, the most obvious being how we enforce guilt and pride into those that exhibit virtues and vices. To teach them young, how ethical theory works. 
\par A bit deeper, through habits. Virtues are all about who we are, our attributes, our character, presence, intellect, and much more. We become who we are through our habits; habits of action, ideas, thoughts, and much more. Society creates these habits, even when we don't realize this. To explore this idea, I will present a case study: a commonly criticized culture rule is to not curse. It may seem illogical, they are words, sometimes not even directed at the person offended. Now, before I begin on this analysis, I will say that this particular analysis will focus on social rules as habit forming$\implies$ virtue forming. It will ignore the dozens of other reasons and arguments.
\par From this idea of it being illogical, you can clearly see the reason by what the person is actually doing. They are \textit{tempering} their own \textit{impulses} for the sake of \textit{social harmony} and \textit{empathy} towards others. Having \textit{interpersonal tact} and learning from \textit{moral authority and teachers} without rebelling because you didn't see the big picture. These habits are both skill and virtue building so we may be civilized adults. This is also why it is so extremely emphasized during childhood.
\\
\par Now, while these parts are extremely important, they aren't simply circular self-defining concepts. Virtues exist beyond civilization and their primary justification being to mimic Christ and fulfill our duty to steward his creation.
\subsection{*Social Rules and Morality}
\subsection{*Culture as Moral Education}
\subsection{Intensity of Duty*}
\par While on a strictly moral view, ethics and their virtues are extremely important. I have gone over again and again, how these ideas should be pursued fanatically, but individually. While they should be taught to others, expecting purely moral actions from other is not something to be expected from them.
\par Virtues of civilization are another story. Some lenience is required, but to much leads to a denegation, a slippery slope. Things like duty, respect, loyalty, 
\section{*Expectations of Others}
\section{*Simulation and Questions}
\subsection{*Genetic Modification}
\section{*Conclusion}
%%%%%%%%%%%%%%%%%%%%%%%
\chapter{Religion}
\section{Methodist Traditions}
\subsection{Introduction}
\par About a year ago my church has split from the Methodist denomination. Though, my beliefs are still strongly rooted in Methodist tradition and thought. Here I will explain in broad strokes for further analysis later on.
\subsection{Axioms}
\par To start with axioms is extremely important to anything. To be able to think clearly in an abstract way, defining these axioms is key. So here:
\begin{itemize}
    \item God is Holy, loving, just, and relational
    \item God is the Trinity of the Father, Son, and Holy Spirit
    \item Human beings are made in the image of God, but corrupted by sin
    \item Grace is universal and prevenient
    \item Faith must be active in love
    \item Truth in known though quadrilateral: Scripture, reason, tradition, and experience
\end{itemize}
\par These, while not the only, are the primary axioms of Methodist thought.
\subsection{Doctrine of God}
\par due to the first axiom of who God is, he is personal and morally perfect. Not only that, but he demands us to pursue moral perfection, both personally and socially.
\par Obviously, beyond that our understanding of God is imperfect, but through the final axiom of knowledge through quadrilaterals, we find some understanding of God, but I will explain more later.
\par He is also dynamic and affecting our world. Though, it is more of the fact that he created the universe with his effects already designed within the world. Not that this is the only way, but it is the primary way. 
\subsection{Human Nature and Sin}
\par From axiom three we can say:
\par Humans are made with moral freedom and intrinsic dignity, but sin has corrupted our will, reason, and desire.
\par Then through the grace of God, we pursue moral perfection. We will not achieve it, but the pursuit is the purpose in it of itself. This morality is based upon virtue ethics created by Jesus, his disciples, and what we ourselves derive through reason and experience.
\par We are moral agents with responsibility to moral action, but due to our imperfections built in, we still require God's grace.
\subsection{Grace and Salvation}
\par From axiom four: 
\par First, there is Prevenient Grace, where God draws every person towards Himself before they respond.
\par Justifying Grace, when a person repents and believes, they are forgiven.
\par Sanctifying Grace, the process of being made holy by cooperating with God's spirit. This is the pursuit of Moral perfection as mentioned earlier.
\par In essence, salvation is not a one-time thing. It is a lifelong journey of gradual transformation to virtue, not simple radical transformation or constant struggle against without progress. 
\subsection{Free Will and Moral Responsibility}
\par From both axiom 3 and 4:
\par Human beings are free to accept or reject God's grace, but God does desire us to choice us.
\par As he does desire us, he still respects human choice.
\par Thus salvation can be lost through persistent rejection.
\subsection{Christian Perfection \& Sanctification}
\par As moral perfection has been discussed, the idea goes further. As mentioned in my section of ethics, we pursue virtue in order to create purity of intentions. This is the goal, not sinless action but purity of intention. This is further developed in my section on ethics.
\subsection{Means of Grace and Sacraments}
\par God gives grace though means: prayer, scripture, communion, fasting, fellowship, good works, etc. 
\par These actions bring us closer to him and create a relationship.
\par This relationship is symbolically expressed through communion and Baptism. Outward signs of inward grace.
\subsection{Wesleyan Quadrilateral}
\par Scripture is our foundational text and where our imperfect understanding of our world, God, and doctrine. Though due to our imperfections this has faults.
\par Reason helps us interpret scripture, understand God through the world around us, think in abstract terms of theology, ethics, and God, and helps us apply these ideas to the real world.
\par Tradition connects us to the historical wisdom of others and their reason.
\par Experience ensure that our doctrine is lived and personally meaningful. It also amplifies our ability to reason for all of the above.
\subsection{Ethics and Social Justice}
\par Christian love extends to our neighbors and enemies. We must pursue change the world for the better. Be teachers, leaders, and workers to compate the evils of the world.
\par Faith without works is dead.
\subsection{Church Structure}
\par Through a connective body we can organize, stay accountable, and use our collective discernment to understand greater than our individual reason.
\par Though, this organization shouldn't be overtly hierarchical but share the authority.
\par Through servant-based leadership, we perfect the world with ourselves.
\subsection{End Times}
\par This is something I don't have true thoughts of. Some say there will be no rapture but an ethical transformation of the world.
\par I am not sure, but I live the idea of the future that belongs to faithful action, not apocalyptic fear. Though, I am not God.
\subsection{Liturgical}
\par The Methodist tradition suggest that we must be flexible to both high church and low church
\par Though, there is a sense of some some tradition. Having ideas of plain dressing, dignified actions, lecture based liturgy, and such.
\par I personally like this. I like dignity and rules, and I enjoy lecture based learning.
\par Another thing is the idea of trying to teach the people so they may be theologically informed with an emphasis on biblical literacy so people and interpret openly to better understand through their quadrilateral

\section{*Stewardship}
\section{*In the Image of God}
\section{*The Trinity}

\section{Matthew}
\subsection{1}
\par The first chapter of Matthew is quite the question. Going through the genealogy of Jesus, but through Joseph's side. It is curious as to why, given that Joseph is not Jesus's real father, rather God is. The most obvious would be a combination of proving the connection between Jesus and David, and that genealogy of the father was tradition. Another thing is Mary's Daviatic lineage is already applied within the book of Luke. Further more, these legalistic paternal statement coupled with the feature of Jesus's virgin birth creates a nice literary tool and analogy to his fully man and fully God.
\par Beyond simply the stated facts, the genealogy is divided in three sets of fourteen. From Abraham to David, from David to the exile of Babylon, finally from the exile to the Messiah. This showcases God's chosen symmetry. That he had already laid this out and had planned it from the beginning. A way to even further underscore this is the fact that the numerical value of David’s name in Hebrew gematria is fourteen. (D=4, V=6, D=4; 4+6+4=14)
\par One thing I hadn't thought of, but I read about was the addition of women that otherwise would have been skipped due to scandalous origins are mentioned and used. This showcases God's grace. 
\par Finally, the story of Joseph's accepting Jesus. 
\par We see how God takes physical steps to ensure the faith of Joseph. By visiting him in a dream, by divine intervention. 
\par Another key theme is that God asks Joseph to go against Jewish customs. To go against traditional morality and purity in order to act in a truly moral light.
\subsection{2}
\par It is interesting that we generally think of three 'kings' or wise men when that is not what is stated. All that is known is that it is a group of scholars, nothing beyond that. Not really sure what it means beyond be aware of your theological beliefs. Furthermore, it does point to the salvation of the gentiles through the fact that the magi were gentiles.
\par Afterwards, King Herod was disturbed and desired to end Jesus's life. This first shows humanities arrogance in trying to thwart the will of God, their desire to be placed above, and more. 
\par Due to this threat, God takes Jesus's family to Egypt. This shows God taking divine guidance rather than leaving them or doing something overt like kill King Herod.
\par Next, Herod killing the children points to humanities depravity when searching beyond God.
\par Finally, we see how the prophecies have been fulfilled. Still pointing to God's architecture of the Universe.
\subsection{3}

%%%%%%%%%%%%%%%%%%%%%%%
\chapter{Personal Philosophy}
"Only through the radical examination of one's own life and the reality surrounding may one gain authority over it. Through such authority their own happiness becomes clear. To have the freedom to be responsible for themselves and the integrity of their own mind. This is the principle and highest form of human action, to divine their own morality and values then shape themselves and reality into the image of their highest values.” | AJ Cason
\section{*God as Foundation}
\section{Will as Identity}
\par All throughout history, the question of identify has been constant. Who we are: are we are personality, outward actions, inward thoughts, natural impulses, cognitive style?
\par I propose a different method, we are our will. That is our method of identity. This proposition has many consequences.
\par Before I get to that, I must define your will. Your will is your highest self within you, that which pushes you to your highest ideals. The image that burns within you of what you could become, that whispers how you may achieve it. That is your will, your will to your own power over yourself.
\par First, it allows acting through purpose to be acting authentically. Many people find paradox between our need to live 'authentically'(a moronic ideal) and fulfilling their purpose and duty to others and themselves. Though, from my view these are the same. To act and bend your personality to your personal will, to fulfill your duty, isn't unauthentic, but rather the purest form of authenticity. 
\par To 'act' and author yourself is your responsibility as a human, a creation of God the most high that has been made in his image as a being with a conscious mind. Don't live passively letting your whims and childhood define you, but rather your will is your definition. 
\par Second, a devotion to your will is your  responsibility. To 'falter' is to kill yourself and let a husk(personality, impulses, etc_ take the place of the human mind that God has put within you. To let the flesh control you is dishonorable, evil, disgusting, and never to be considered. It is not understandable, it is not acceptable, never surrender, never compromise against anything beyond your will.
\par This can be observed through giving up, compromising, complaining, etc. These are evils. People seem to accept complaining as a natural reaction to 'unfairness' and that you are nothing more than your "childhood traumas," don't become this subhuman husk. Become more, become a thinking human.
\par Four, to let others to your thinking for you is to outsource your soul. To degenerate God's creation. Think for yourself, think logically and analytically, derive your 'first principles' clearly.
\section{*Fanatical Ethics}
\section{The Duty as an Actor}
\par As mentioned in "Will as Identity" you must in a sense become an "actor" in your life. What I mean by this is that you must control your reactions(be civilized and dignified) to fulfill your duty. 
\par Be cheerful within the misk of irksome tasks and weighty responsibilities, hold your head up high, never complain, don't profane what is sacred, and so much more. 
\par One, through "acting" you will become that in which you pursue, virtuous. You will become happy, become moral, become dignified, become who your will decides you must be. It is through this "act" you will become that image of your highest self within you. 
\section{*The Aesthetic Pursuit of Truth}

%%%%%%%%%%%%%%%%%%%%%%
\chapter{Philosophy of Logic}
\section{*Nature of Logic}
\section{*Types of Logic}
\section{*Theory of Logic}
\section{*Theory of Math}
\section{*Metaphysics of Logic}
\section{*Logic and Epistemology}

\chapter{*Philosophy of Mathematics}
%%%%%%%%%%%%%%%%%%%%%%%
\chapter{The Structure of Motion: A Philosophical Journey Through Hamiltonian Mechanics}
“The Hamiltonian formalism of mechanics, especially in its canonical form, has been an inexhaustible source of inspiration for modern theoretical physics.” | Albert Einstein
\section{Introduction}
\par This chapter is planned to be an extreme examination of Hamiltonian mechanics, especially from a philosophical, logic, and abstract way. Eventually, it will go from the basic logic and axioms underlying it, to the metaphysics and epistemology that are derived, and finally to advanced and modern usages. The primary purpose of this exercise is for me to find the gaps in my knowledge and gain a more intuitive and deep understanding of these concepts.
\par The reasoning for this is due to the extremely complex and abstract nature of such mechanics. To see the world through 'flows' instead of forces.
\subsection{Physical Reality and Frameworks}
\par Now, there are many theories, models, descriptions, framework, and so much more. Though, what really differentiates them? What makes one thing one and the other the other. While I won't truly go into extreme linguistic detail for every single one, I will go into moderate detail within the art to get the general idea down.
\par Let's focus on frameworks, because that is what a Hamiltonian is(others define it as schemes which is another great word and arguably more accurate, but I will refer to it as a framework because it has a more intuitive grasp). It is an entirely different framework than Newtonian. It comes from an entirely different perspective, looking into geometric identities rather than the causal relation of forces and such interactions. Also, other theories and identities are derived through it, independent of the actual reality of it. For instance, the flow of electromagnetic interactions, relativistic, or even gravitational are all derived from this concept of the flow of energies through sympathetic geometry. 
\par Even beyond that, Hamiltonian mechanics can represent any function that changes; statistics, viruses, and many others. This is due, to that Hamiltonian's is a pure mathematic relational concept rather than based upon physical axioms. For instance, Newtonians is based upon the axiom of inertia. Though, I will add one interesting involvement is the idea of privileging Hamiltonian over Lagrangian and stating that Lagrangian is a subset of Hamiltonian rather than its own framework.
\par Then this begins to ask our questions of scientific realism vs anti-realism(I found out the more neutral language is instrumentalism). Basically, what makes such theories more fundamentalist, predictive modeling or metaphysical truth. Further, this now engages the complexities of mathematical axiomatic reduction or physical.
\par Let's start with realism vs instrumentalism. I have already made my stance clear, having metaphysical truth in important but not in the way that non-metaphysical schemes should be thrown away. Though, this then brings about the question of what frameworks have greater metaphysical truth, but I will later look into this in later sections. 
\par This now leads to the question of mathematical abstraction vs physical realism. Though, one thing I will add is that the physical axioms of Newton's laws are based upon falsified ideas like absolute space and time(but you can get around such semantics by employing Whewell's axioms of Mechanics[1st: “Every change is produced by a cause.”
2nd says that: “Causes are measured by their effects.” Finally, the third remain unchanged from Newton's formulation.] Now back to the main event, I feel that the structuralism of Hamiltonian is better due to the fact that when you break down physics to the quantum the classical forces, inertia, and such break down and you can see that the structure is what remains. That while forces still interact as abstract 'causes of change" through bosons, their classical and 'physical' realism fails to encapsulate their extreme complexity.

\subsection{Philosophical Context*}
\par The Kantian influence on Sir Hamilton is clear....
\subsection{Notion of Time, States, and Trajectories}
\par Let's start with time. I have already explained my thoughts on time, but here I will deepen my explanation. Time, is the factor on which states evolve. It is then evolution of entropy. This axiom is very simple and intuitive. While objective empirical proving the nature of time is extremely challenging due to its abstract nature. For a more in-depth analysis of the nature of time visit Chapter 1, section 3\&7.
\par One thing I will discuss is time reversibility. Now, Hamiltonian mechanics works best for time reversal symmetries(it can still work in some use cases but rarely.). In fact, time reversal symmetry is a corner stone of much of modern physics, especially quantum mechanics, but this interferes with our intuitive grasp of reality, the second law of thermal dynamics, and even dark energy; in essence the emergent properties of reality seem to interfere with our 'fundamental mathematical' derivations. So I will examen these questions; both for their own state and because it continues our question of the fundamental nature of reality. I will also attempt to discuss imaginary time with regards to quantum mechanics.
\par Actually, upon further thought the nature of time will be further developed through after the derivation of analytical mechanics.
\par Next will the on trajectories; now like many other ideas presented within this book, that may seem simple but the derivation of the ideas will be thorough and show its true complexity. Within Hamiltonian mechanics, the trajectories hold special geometric identities due to the nature of motion, but what can we derive from this simple observations

\subsection{Introduction to Phase Space}
\par Another question that arises from a Hamiltonian view of metaphysics is Phase Space. The fact that, according to our theories, phase space almost seems real, but yet it is so very different than what we observer. Or, we think so.
\par Obviously, there is the idea of the conflict between the teleology that comes as a consequence and our view of causality. I will address this in a second.
\par A simpler idea is that, from a perspective view, change becomes fundamental and any identity becomes simply emergent. This isn't truly new, simply more emergent in phase space, so I will move on.
\par One new thing, is that Phase Space allows for entropy irreversibility. Traditional theories find time reversibility to be a corner stone, this going against it.
\par This is due to the fact that entropy increases
$$S=-k_B\Sigma_i p_i logp_i$$
Where the volume of the phase space stays the same
$$Vol(\phi_t(\Omega_M))=Vol(\Omega_M)$$

\par Lastly, the 6D phase space seems very different to ours. Having generalized position and momentum coordinates is odd. An instrumentalist view makes this simple, but a scientific realism makes this difficult. Sadly, I don't have the knowledge to come up with.
\subsection{Teleology vs Causality}
\par My privileging of causal relations has already become clear, but now I will justify my thoughts through a Hamiltonian view point. While many claim that the nature of Hamiltonian mechanics necessitates a teleology, and my stated view on structuralism seems to contradict my views on causality, but I will address these claims categorically.
\par First, the nature of Hamiltonian mechanics necessitates a teleological view of the natural world. This statement will seemingly logical, lacks the tautological strength required of such an extreme statement. For one, Hamilton himself had a clear belief in causality(while this obviously isn't enough, I will explain why he and I see it that way.) First, the concepts of causal forces(forces are by definition causation principles) can be derived through the inverse gradient of potential energy. Second, the connection of the 'principle of least' action and teleology is flimsy at best. This is because their main argument is based upon intuition of mathematic alone not physical realism(I know this goes against stated structuralism but I will come back to that), the reason I state this against due to the fact that, mathematical intuition in the face of metaphysical truth. For instance, as stated earlier the teleological 'energy' is found as a derivation of causal force. Finally, returning to our arguments placed in the first chapter.
\subsection{Role of Variational Principles in Mechanics}
\section{Axiomatic Mechanics}
\subsection{Axiomatic Derivation of ... }
\subsubsection{Introduction}
\par For any mathematical physical axiomatic derivation, a mathematical derivation is first required. I won't go super in depth but I will still go over it for rigors sake. I may go back and do a deeper dive but for now I will simply put these things.
\subsubsection{Axiom of Real numbers}
Why needed: Physical quantities like position, velocity, time, energy, and action are represented by real numbers. The calculus used in mechanics relies on the properties of real numbers.
\\
Key axioms:
\begin{itemize}
    \item Commutative, associative, and distributive properties for addition and multiplication.\
    \item Existence of additive and multiplicative identities (0 and 1).
    \item Existence of additive and multiplicative inverses (for non-zero numbers).
    \item Completeness axiom: Every non-empty set of real numbers with an upper bound has a least upper bound (ensures continuity, critical for calculus).
\end{itemize}



Role: These axioms enable arithmetic operations, inequalities, and the construction of functions like the Lagrangian $ L(q, \dot{q}, t) $.

\subsubsection{Axiom of Euclidean Geometry}
Why needed: If the system involves spatial coordinates (e.g., particles moving in 3D space), Euclidean geometry provides the framework for defining positions and distances.
\\
Key axioms(informal):
\begin{itemize}
    \item Points, lines, and planes exist.
    \item A straight line can be drawn between any two points.
    \item Distance between points is defined (e.g., via the Pythagorean theorem).

\end{itemize}

Role: Defines generalized coordinates $ q $ (e.g., Cartesian or polar coordinates) and kinetic energy terms like $ \frac{1}{2} m \dot{x}^2 $.
Note: For abstract systems (e.g., in generalized coordinates), geometry may be less critical, but it’s foundational for physical intuition.

\subsubsection{Axiom of Set Theory (Basic)}
Why needed: Sets are used to define the domain of variables (e.g., time $ t \in \mathbb{R} $, coordinates $ q \in \mathbb{R}^n $) and functions.
\\
Key axioms(informal):
\begin{itemize}
    \item Existence of sets: Sets can be formed to represent collections of objects (e.g., possible paths of a system).
    \item Union, intersection, and Cartesian product: Allow combining and manipulating sets.
    \item Axiom of choice (implicitly): Ensures a choice of path exists in variational problems.

\end{itemize}

Role: Provides the language for defining functions, spaces, and the configuration space of a system.

\subsubsection{Axiom of Calculus (Differential and Integral Calculus)}
Why needed: The principle of stationary action involves integrals (action $ S = \int L \, dt $) and derivatives (in Euler-Lagrange equations).
\\
Key concepts (built on axioms of real numbers):
\begin{itemize}
    \item Limits: Define continuity and differentiability of functions like $ L(q, \dot{q}, t) $.
    \item Derivatives: Partial derivatives (e.g., $ \frac{\partial L}{\partial q} $) are used to describe rates of change.
    \item Integrals: The Riemann integral defines the action $ S $.
    \item Fundamental theorem of calculus: Links derivatives and integrals, essential for variational calculus.

\end{itemize}

Role: Enables the formulation of the action and the variation $ \delta S = 0 $.



\end{itemize}

Role: Provides the language for defining functions, spaces, and the configuration space of a system.

\subsubsection{Axioms of Variational Calculus}
Why needed: The principle of stationary action requires finding the path that makes the action stationary, which is a problem in variational calculus.
\\
Key axioms(informal):
\begin{itemize}
    \item Functional: The action $ S $ is a functional (a function of functions, e.g., paths $ q(t) $).
    \item Variation: Small changes in the path $ \delta q(t) $ are used to compute $ \delta S $.
    \item Stationary condition: The path satisfies $ \delta S = 0 $, leading to the Euler-Lagrange equations.

\end{itemize}

Role: Provides the mathematical machinery to derive the equations of motion from the action.

\subsubsection{Axioms of Linear Algebra (Basic)}
Why needed: For systems with multiple coordinates or degrees of freedom, vectors and matrices describe generalized coordinates $ q_i $ and momenta $ p_i $.
\\
Key axioms:
\begin{itemize}
    \item Vector space axioms: Addition and scalar multiplication of vectors (e.g., coordinates in $ \mathbb{R}^n $).
    \item Linearity: Operations like partial derivatives in Hamilton’s equations are linear.


\end{itemize}
Role: Supports the formulation of the Hamiltonian $ H(q, p, t) $ and phase space (the space of $ (q, p) $).

\subsubsection{Axioms of Variational Calculus}
Why needed: The principle of stationary action requires finding the path that makes the action stationary, which is a problem in variational calculus.
\\
Key axioms(informal):
\begin{itemize}
    \item Functional: The action $ S $ is a functional (a function of functions, e.g., paths $ q(t) $).
    \item Variation: Small changes in the path $ \delta q(t) $ are used to compute $ \delta S $.
    \item Stationary condition: The path satisfies $ \delta S = 0 $, leading to the Euler-Lagrange equations.

\end{itemize}

Role: Provides the mathematical machinery to derive the equations of motion from the action.

\subsubsection{Physical Assumptions (Not Strictly Mathematical Axioms)}
While not mathematical axioms, certain physical assumptions are mathematically formalized:
\begin{itemize}
    \item Time is continuous and one-dimensional ($ t \in \mathbb{R} $).(Though Hamiltonains can work in other forms)
    \item Energy is well-defined: Kinetic energy $ T $ and potential energy $ V $ are functions of coordinates and velocities.
    \item Differentiability: The Lagrangian $ L $ is sufficiently smooth (at least twice differentiable) to allow partial derivatives
    \item 

\end{itemize}
Role: These ensure the mathematical framework applies to physical systems.
\subsection{Axiomatic Derivation of Hamiltonian Mechanics}
% Introducing Hamilton's contribution
Two hundred years ago, William Rowan Hamilton reformulated classical mechanics using the \textbf{principle of stationary action}, a single axiom that unifies the dynamics of physical systems. This principle states that the path taken by a system between two times makes the action \( S \) stationary.

% Defining the action and Lagrangian
The \textbf{action} is defined as:
\[
S = \int_{t_1}^{t_2} L(q, \dot{q}, t) \, dt,
\]
where \( L = T - V \) is the \textbf{Lagrangian}, with \( T \) as kinetic energy, \( V \) as potential energy, \( q \) as generalized coordinates, and \( \dot{q} \) as their time derivatives.

% Deriving the Euler-Lagrange equations
The system follows the path where \( \delta S = 0 \). Varying the action:
\[
\delta S = \int_{t_1}^{t_2} \left( \frac{\partial L}{\partial q} \delta q + \frac{\partial L}{\partial \dot{q}} \delta \dot{q} \right) dt = 0.
\]
Integrating by parts on the second term:
\[
\int_{t_1}^{t_2} \frac{\partial L}{\partial \dot{q}} \delta \dot{q} \, dt = \left. \frac{\partial L}{\partial \dot{q}} \delta q \right|_{t_1}^{t_2} - \int_{t_1}^{t_2} \frac{d}{dt} \left( \frac{\partial L}{\partial \dot{q}} \right) \delta q \, dt.
\]
Since \( \delta q = 0 \) at \( t_1, t_2 \), we get the \textbf{Euler-Lagrange equations}:
\[
\frac{d}{dt} \left( \frac{\partial L}{\partial \dot{q}} \right) - \frac{\partial L}{\partial q} = 0.
\]

% Introducing the Hamiltonian
Hamilton defined the \textbf{Hamiltonian} as:
\[
H(q, p, t) = \sum_i \dot{q}_i p_i - L(q, \dot{q}, t),
\]
where \( p_i = \frac{\partial L}{\partial \dot{q}_i} \) are generalized momenta. Typically, \( H = T + V \).

% Deriving Hamilton's equations
Using the Legendre transform, the dynamics are governed by \textbf{Hamilton's equations}:
\[
\dot{q}_i = \frac{\partial H}{\partial p_i}, \quad \dot{p}_i = -\frac{\partial H}{\partial q_i}.
\]

% Example application
\textbf{Example}: For a particle in gravity, \( L = \frac{1}{2} m \dot{h}^2 - mgh \). The momentum is \( p = \frac{\partial L}{\partial \dot{h}} = m \dot{h} \). The Hamiltonian is:
\[
H = \frac{p^2}{2m} + mgh.
\]
Hamilton's equations yield:
\[
\dot{h} = \frac{p}{m}, \quad \dot{p} = -mg,
\]
reproducing the equation of motion \( m \ddot{h} = -mg \).

% Significance
The principle of stationary action simplifies dynamics, applies to diverse systems, and underpins modern physics, including quantum mechanics and relativity. Hamilton's work remains a cornerstone of theoretical physics.
\subsection{*Hamiltonian Mechanics ad a Geometric Theory}
\subsection{*Poisson Brackets and the Algebra of Dynamics}
\subsection{*Canonical Transformations: Symmetry, Simplicity, an Structure}

\section{Philosophical Implications*}
\subsection{*Time in Essence}
\subsection{*Hamiltonian Constraints and the Problem with Time}

\section{Modern Hamiltonian}
\subsection{Modern Research}
\subsubsection{*Hamiltonian Formulation in Plasma Physics}
\subsubsection{*Hamiltonian and Metriplectic Mechanics}
\subsubsection{*Symplectic Integrators: Preserving the Structure of Nature}
\subsubsection{*HNNN(Hamiltonian Neural Networks)}
\subsection{Fun Mess Around}
\subsubsection{Non-Canonical Poisson Brackets}
\par A Poisson bracket is a bilinear, antisymmetric operation $[F, G]$ that defines the time evolution of a functional $F$ via $\dot{F} = [F, H]$, where $H$ is the Hamiltonian.  Non-canonical brackets arise when the phase space variables (e.g., density, velocity, magnetic field in MHD) do not follow the standard canonical structure $\{q_i, p_j\} = \delta_{ij}$. 
\\
Properties: The bracket must satisfy:
\\
\begin{itemize}


    \item Antisymmetry: $[F, G] = -[G, F]$
    \item Leibniz Rule: $[F, GH] = [F, G]H + G[F, H]$
    \item Jacobi Identity: $[F, [G, H]] + [G, [H, F]] + [H, [F, G]] = 0$
\end{itemize}
\par For this I will be defining my own Poisson bracket specifically for plasma thrusters.
\par For this I will obviously need to define the variables. Things like:
\begin{itemize}
    \item Mass density: $\rho(\mathbf{r}, t)$
    \item Velocity field: $\mathbf{v}(\mathbf{r}, t)$
    \item Magnetic field: $\mathbf{B}(\mathbf{r}, t)$
    \item Entropy or internal energy: $s(\mathbf{r}, t)$ or $\epsilon(\rho, s)$
\end{itemize}
\par Now define the Hamiltonian, for this I will use the pre-established ideal-MHD, but in all reality for plasma thrusters I should use a more expanded model.
$$H[\rho, \mathbf{v}, \mathbf{B}, s] = \int \left( \frac{1}{2} \rho v^2 + \rho \epsilon(\rho, s) + \frac{B^2}{2\mu_0} \right) d^3x$$
\par Next, I must define the phase spaces and constraints. Incompressibility or magnetic field divergence. 
\par Next I propose a bracket system, with many of them being 
$$[F, G] = \int \sum_{i,j} \frac{\delta F}{\delta \xi_i} J_{ij} \frac{\delta G}{\delta \xi_j} d^3x$$
\par Such that Ideal MHD is:
$$[F, G] = -\int \Bigg\{
\rho \left[ \frac{\delta F}{\delta \rho}, \frac{\delta G}{\delta \mathbf{v}} \right]
+ \left[ \frac{\delta F}{\delta \mathbf{v}}, \frac{\delta G}{\delta \mathbf{v}} \right] \cdot \left( \frac{\mathbf{B}}{\rho} \times \nabla \times \frac{\delta G}{\delta \mathbf{B}} \right)
+ \frac{1}{\rho} \left( \nabla \times \frac{\delta F}{\delta \mathbf{B}} \right) \cdot \left( \frac{\delta G}{\delta \mathbf{v}} \times \mathbf{B} \right)
+ s \left[ \frac{\delta F}{\delta s}, \frac{\delta G}{\delta \mathbf{v}} \right]
\Bigg\} d^3x$$
\par Another approach is using lie-groups. The Lie-Poisson approach is a method to construct non-canonical Poisson brackets by reducing a canonical Hamiltonian system on a large phase space (e.g., particle coordinates) to a smaller phase space of collective variables (e.g., fluid fields in MHD). It is rooted in the symmetry properties of the system’s configuration space, described by a Lie group.
\par ...
\par The other way is through Casmir invariants. Casimir invariants are functionals $C$ that commute with all functionals $F$ under the Poisson bracket: $[C, F] = 0$. They are conserved quantities that arise from the degeneracy of the non-canonical bracket and provide constraints on the system’s dynamics.
\par Degeneracy: Non-canonical Poisson brackets are degenerate, meaning their Poisson tensor $J_{ij}$ has a non-trivial kernel. Casimirs live in this kernel, satisfying:
$$J_{ij} \frac{\delta C}{\delta \xi_j} = 0$$

\par Physical Role: Casimirs represent invariants tied to the system’s topology or symmetries, such as helicity in MHD, which constrain the evolution of plasma in thrusters.
\par Identification: Casimirs are found by solving the functional equation $[C, F] = 0$ for all $F$, often using the Lie algebra’s cohomology.
\par 
\chapter{My Physics Research}
"I have no special talents. I am only passionately curious." - Albert Einstein
\section{Previous Research}
\par These are brief/lazy copies/explanations of things I have written previously made projects.
\subsection{3D Modeling of Non-Equilibrium Dynamics in Compressed Plasma Using Lattice Boltzmann Method}
\par The construction of a theoretical model, magnetohydrodynamic lattice boltzmann method(MHD-LBM) model for 3D compressed plasma, using a finite volume scheme is constructed. The hyperbolic Maxwell equations, which satisfy the elliptic constraints of Maxwell's equations and the constraint of charge conservation, are used to simulate the electromagnetic field. The flow field and electromagnetic field are coupled to simulate a compressible plasma through the electromagnetic force and magnetic induction equations. This model can further be applied to create a quantitative simulation to model complex nonequilibrium effects of compressed plasma to provide mesoscopic physical insights into the flow mechanism of a shock wave in a supersonic plasma.
\par {Theoretical model, Finite volume scheme, hyperbolic maxwell equations, and nonequilibrium effects}

\subsubsection{3D Modeling of Non-Equilibrium Dynamics in Compressed Plasma Using Lattice Boltzmann Method
Introduction}
\par The study of nonequilibrium dynamics in compressible plasma is a critical area of research in plasma physics, with significant implications for both theoretical understanding and practical applications. Compressible plasmas are found in a variety of contexts, such as astrophysical phenomena(1), industrial processes(2,3), and fusion. Understanding the complex interactions and behaviors of plasma under nonequilibrium conditions is essential for advancing these fields.
\par Previous research has primarily focused on two-dimensional models(4) While these models have provided valuable insights, they often fall short in capturing the full complexity of three-dimensional plasma dynamics. The limitations of these models highlight the need for more advanced simulation techniques that can accurately represent the intricate behaviors of compressible plasma in three dimensions.
\par The Lattice Boltzmann Method (LBM) offers a powerful tool for modeling fluid dynamics at the microscopic level.(4,5) Unlike traditional computational fluid dynamics methods, such as the Navier-Stokes equation, the LBM is particularly well-suited for simulating nonequilibrium effects and complex boundary conditions. Its ability to handle multiphase flows and incorporate microscopic interactions makes it an ideal choice for studying compressible plasma dynamics.(5)
\par This paper aims to develop a comprehensive 3D model using the LBM to simulate nonequilibrium effects in compressible plasma. By leveraging the strengths of the LBM, we seek to overcome the limitations of previous models and provide a more accurate and detailed representation of plasma behavior. Our research focuses on the coupling of flow and electromagnetic fields, exploring the interactions and dynamics that arise under nonequilibrium conditions.
\\
\\
\subsubsection{Physical Model}
Kinetic Equation
\par Li and Zhong(10) introduced the potential energy distribution function as well as a compressed DDF Lattice Boltzmann equation. A potential energy distribution function can be added so the Boltzmann BGK can obtain an adjustable specific heat ratio or Prandtl number(4)
\par This new Boltzmann Kinetic equation can be written as followed:

$$\frac{\partial f_k}{\partial t} + (\mathbf{e}_k \cdot \nabla) f_k + \mathbf{a} \cdot \nabla_e f_k = -\frac{1}{\tau_f} (f_k - f_k^{eq})$$

$$\frac{\partial h_k}{\partial t} + (\mathbf{e}_k \cdot \nabla) h_k + \mathbf{a} \cdot \nabla_e h_k = -\frac{1}{\tau_h} (h_k - h_k^{eq}) + \frac{z_k}{\tau_{hf}} (f_k - f_k^{eq})$$

Where $k$ is the direction of discrete velocity. $f_k$ is the density distribution function. $h_k$ is the potential energy distribution function, while $f_k^{eq}$ is the equilibrium distribution function. $\mathbf{e}_k$ is the discrete velocity component, and $\tau_f$ is the relaxation time of the density distribution function. $\tau_h$ is the relaxation time of the potential energy distribution function.

This can also be defined as:

$$\tau_{fh} = \frac{\tau_f}{\tau_h}$$

The force term can be approximated as:

$$\mathbf{a} \cdot \nabla_e f \approx \mathbf{a} \cdot \nabla_e f^{eq} = -\frac{a \cdot (\mathbf{e}_k - \mathbf{u})^2}{RT} f^{eq}$$

\subsubsection{Density Distribution Function}
$$f_k^{n+1}(x_i, y_j, z_l) = f_k^n(x_i, y_j, z_l) - \Delta t \left( \frac{F_{k,i+1/2,j,l} - F_{k,i-1/2,j,l}}{\Delta x} + \frac{F_{k,i,j+1/2,l} - F_{k,i,j-1/2,l}}{\Delta y} + \frac{F_{k,i,j,l+1/2} - F_{k,i,j,l-1/2}}{\Delta z} \right)$$
$$- \Delta t \cdot \frac{1}{\tau} (f_k - f_k^{eq})$$

\subsubsection{Potential Energy Distribution}
$$h_k^{n+1}(x_i, y_j, z_l) = h_k^n(x_i, y_j, z_l) - \Delta t \left( \frac{G_{k,i+1/2,j,l} - G_{k,i-1/2,j,l}}{\Delta x} + \frac{G_{k,i,j+1/2,l} - G_{k,i,j-1/2,l}}{\Delta y} + \frac{G_{k,i,j,l+1/2} - G_{k,i,j,l-1/2}}{\Delta z} \right)$$
$$- \Delta t \cdot \frac{1}{\tau_h} (h_k - h_k^{eq}) + \frac{z_k}{\tau_{hf}} (f_k - f_k^{eq})$$

\subsubsection{Hyperbolic Maxwell Equations}
The previously used Maxwell equations have been enhanced with Lagrange multipliers, $\Psi$ and $\Phi$, to combine with the evolution equation. These are the new Maxwell equations in derivative form:

$$\frac{\partial \mathbf{B}}{\partial t} + \nabla \times \mathbf{E} + \gamma \nabla \psi = 0$$
$$\frac{\partial \mathbf{E}}{\partial t} - c^2 \nabla \times \mathbf{B} + \chi c^2 \nabla \phi = - \frac{\mathbf{J}}{\epsilon_0}$$
$$\frac{\partial \Psi}{\partial t} + \gamma c^2 \nabla \cdot \mathbf{B} = 0$$
$$\frac{\partial \Phi}{\partial t} + \chi \nabla \cdot \mathbf{E} = 0$$

Where $\mathbf{E}$ is the electric field, $\mathbf{B}$ is the magnetic field, $\Psi$ and $\Phi$ are introduced divergence variables, $\gamma$ and $\chi$ are divergence error propagation speeds, $c$ is the speed of light, and $\epsilon_0$ is the vacuum permittivity.

\subsubsection{Kinetic Non-Equilibrium Method}
The nonequilibrium effects on compressed plasma are observed through differences in molecular speed. Each non-equilibrium kinetic moment can be represented as the difference between the corresponding kinetic moment and the local equilibrium kinetic moment. The equations of kinetic moments may be used to find the nonequilibrium quantities.

$$M_{f,m}^{neq} = M_{f,m} - M_{f,m}^{eq}$$
$$M_{h,m}^{neq} = M_{h,m} - M_{h,m}^{eq}$$

\subsubsection{Discussion}
In this paper, a 3D MHD-LBM analytical model was constructed for compressed kinetic plasma. Future research must be done in developing a computer model and validating said computer model. A third-order finite volume MUSCL could be applied to this analytical model combined with a D3Qx density equilibrium distribution function to create an easily solvable numerical solution. While electromagnetic fluxes were evaluated through Steger-Warming flux vector splitting. MHD-LBM models have shown to have greater accuracy when it comes to plasma shockwaves, therefore MHD-LBM models have good reason for continual development.
\subsection{A Hamiltonian Framework on ICF Implosions Rocket Equation Based on Rayleigh–Taylor Instabilities
}
\subsubsection{Background}
\par Inertial Confinement Fusion (ICF) Implosions:
One of the primary methods of heating fusion environments is through lasers. In this approach, high-energy lasers heat spherical fuel capsules, causing them to implode. This implosion leads to an increase in pressure and heat on the fuel, creating the necessary environments for fusion.

\par Hamiltonian:
A Hamiltonian is a function in classical mechanics that describes the total amount of energy in the system. From here, equations of motion are derived.

\par Rayleigh--Taylor Instabilities:
Rayleigh--Taylor instabilities are specific types of instabilities caused by forceful interactions between higher and lower density fluids. In the context of ICF Implosions, when the higher density outer shell interacts with the lower-density inside, it creates perturbations through finger-like structures and bubbles that interfere with the efficiency of the implosion.

\par Mass Ablation (Rocket Effect):
During the process of ICF Implosion, the interaction between the outer shell and the inner fuel causes mass to be sprayed off, giving a rocket-like effect.



\subsubsection{Problem Statement}
\par Achieving commercial fusion energy through inertial-confinement-fusion (ICF) remains one of the most significant scientific and engineering challenges of our day. One of the key obstacles is optimizing the interactions between high-energy lasers and fuel capsules, which are prone to Rayleigh-Taylor instabilities during implosions. These instabilities can lead to inefficient energy confinement and hinder the overall success of the fusion process. Traditional numerical simulations are computationally expensive and often lack the accuracy needed to address this issue. Therefore, there is a need for a robust analytical framework to model these instabilities and provide new insights for optimizing the interactions. This research aims to develop a Hamiltonian framework for ICF implosions, specifically focusing on Rayleigh--Taylor instability and mass ablation. By leveraging this theoretical approach, we seek to provide deeper insights and practical solutions for optimizing laser-capsule interactions in fusion experiments.

\subsubsection{Assumptions}
\par Thin-Shell Approximation:
The thickness of the shell is assumed to be insignificant compared to the size of the imploding object. This approximation simplifies the creation of the model and the subsequent numerical analysis. It is valid given that the perturbations caused by the instabilities are much larger than the actual thickness of the shell.

\par Deceleration:
This model is specifically for an imploding spherical shell that decelerates as it converges onto the compressed fluid within its interior.

\par Acceleration:
This model does not consider the acceleration phase of an ICF capsule implosion at the beginning of the capsule-laser interaction.

\subsubsection{Framework}
In order to analyze the dynamics of the imploding shell, it must first be parameterized through two Lagrangian coordinates $\vartheta$ and $\phi$. Here, $\vartheta$ corresponds to the polar angle $\theta$, and $\phi$ corresponds to the azimuthal angle. The position vector is $\mathbf{X} = \mathbf{X}(t, \phi, \vartheta)$. Thus, the parameterization can be expressed as:

$$\mathbf{X} = \mathbf{X}(t, \phi, \vartheta)$$

The derivative of the surface is given by:

$$\frac{d\mathbf{X}}{dt}$$

\textit{Reference:} D. E. Ruiz; Degradation of performance in ICF implosions due to Rayleigh--Taylor instabilities: A Hamiltonian perspective. Phys. Plasmas 1 December 2024; 31 (12): 122701.

\subsubsection{Shell Kinematics}
The force differential can be described by:

$$dF = p(t) \cdot \mathbf{dA} \times \mathbf{n}$$

where $p(t)$ is the function of pressure with respect to time, and the cross product involves the vectors following the surface of the shell. This leads to the derivation of several important quantities, such as the velocity vector field, position vector, and centrifugal forces, which are relevant to the shape and dynamics of the shell.

An example of the centrifugal force is:

$$F_{\text{centrifugal}} = m \cdot \omega^2 \cdot r$$

\subsubsection{Shell Areal Density}
The change in density of the shell is critical to its dynamics. Given that total mass is not conserved in this equation, deriving it becomes more complex.

$$\rho_{\text{areal}} = \frac{M(t)}{A}$$

where $M(t)$ is the mass at time $t$, and $A$ is the area of the shell's surface.

\subsubsection{Compressed-Fuel Pressure}
The pressure that decelerates the shell can be described in several ways depending on the known information and boundary conditions. One example is:

$$P = P_0 \left( \frac{V_0}{V} \right)^\gamma$$

where $P_0$ is the initial pressure, $V_0$ is the initial volume, $\gamma$ is the adiabatic index, and $V$ is the current volume.

\subsubsection{Variational Principles}
To mathematically solve the Hamiltonian framework, given that the equation is asymptotic, variation principles are applied.

\subsubsubsection{Phase-Space Lagrangian}
This formulation describes the dynamics of a system by combining its configuration space (position variables) and momentum space into a single framework, called phase space.

\subsubsubsection*{Euler--Lagrange Equations}
Through the principle of least action, a set of differential equations are derived to provide an analytical solution to complex nonlinear functions. For example, for radial velocity:

$$\frac{d}{dt} \left( \frac{\partial L}{\partial \dot{r}} \right) - \frac{\partial L}{\partial r} = 0$$

where $L$ is the Lagrangian of the system.

\subsubsubsection{Conservation Laws}
Generally, conservation laws are employed to show that certain parts of the Euler-Lagrange or phase-space Lagrangian are conserved. However, due to mass ablation, parts of the fuel and capsule are lost, which complicates these conservation laws.

\subsubsection{Conclusion}

\par Summary: A Hamiltonian framework was developed to include both Rayleigh--Taylor Instabilities and mass ablation in an inertial confinement fusion implosion, leading to a set of variational principles for future work on optimizing ICF implosions in practical applications.

\par Significance:
Analytical analysis can provide future insights into optimizing ICF implosions in fusion reactors. Additionally, nonlinear numerical models can be developed based on these principles, leading to less computationally expensive models.

\par Limitations:
\begin{enumerate}
    \item This model does not account for more complex laser-plasma interaction phenomena (e.g., bremsstrahlung x-ray losses, alpha heating).
    \item This model does not consider the initial acceleration phase of the interaction.
    \item This model uses the thin-shell approximation, which may not be accurate when perturbations caused by instabilities are smaller than the shell thickness.
\end{enumerate}

\par Future Works:
\begin{enumerate}
    \item Quasilinear Models
To study additional insights from this model, a quasilinear model must be developed for analytical study.

    \item Acceleration Phase
Incorporating the initial acceleration phase into the model will increase its accuracy.

    \item Nonlinear Growth Calculations
Developing nonlinear numerical simulations will be useful in generating further insights and advancing current models.
\end{enumerate}

\section{Basic MHD}
\subsection{MHD Module}
\par The first section of it is initializing the values of things like magnetic field, pressure, and velocity. Future work will be done to have it so such values can be edited in a config file.
\par Next is computing the current density. The analytical equation can be written as 
$$J =\frac{1}{\mu} \nabla \times B$$
Which in finite difference is:
$$J_z(i,j)=\frac{1}{\mu_0}[\frac{B_y(i+1,j)-B_y(i-1,j)}{2\Delta x}-\frac{B_x(i+1,j)-B_x(i-1,j)}{2\Delta y}$$
\\
\\
Next we have
\\
\\

\par Update: I have advanced the MHD simulations time steps, magnetic field calculations, current, pressure calculations, and density calculations. Beyond that I have created animations based on magnetic field and made graphical outputs for heat, velocity, temperature, and magnetic field.
\par I also created another one specifically made for Tokamak geometry.
\subsection{Hamiltonain MHD}
\par My next project was creating a Hamiltonian based MHD simulation also in FORTRAN. 
\par I explicitly created simulation of plasma thrusters with it.
%%%%%%%%%%%%%%%%%%%%%%%

\section{A Metriplectic Formulation of Reduced Magnetohydrodynamics with Resistivity}
\subsection{Introduction}
\par Here I will use the Strauss formulation of RMHD. Then I will incorporate resistivity. This resistivity typically breaks Hamiltonian structure, but I will use metriplectic to preserve the geometric identities through adding a separate antisymmetric bracket.
\subsection{Defining the Hamiltonian}
\par For Strauss MHD, a reduced MHD model that works for strongly magnetized Plasmas, we will formulate the Hamiltonian system.
$$\mathcal{H}[\omega, \psi]=\frac{1}{2} \int(\phi \omega + \psi j)dxdy$$
\par Here the $\phi \omega$ represents the kinetic energy with $\psi j$ represents the magnetic energy.
\par Now onto the Noncannonical Poisson Structure for this ideal Hamiltonian.
$$ \{ F,G \}=\int \omega[\frac{\delta F}{\delta \omega},\frac{\delta G}{\delta \omega}]dxdy+ \int \psi ([\frac{\delta F}{\delta \omega},\frac{\delta G}{\delta \psi}]-[\frac{\delta G}{\delta \omega},\frac{\delta F}{\delta \omega}])dxdy  $$
\subsection{Entropy and Dissipation}
\par Now to add dissipation. While generally adding this will break Hamiltonian structure, we will use a couple of tools. 
\par Though first we must define our function we are using, or more accurately funcional. This is our functional for entropy:
$$S[\psi]=\frac{1}2 \int \psi^2dxdy$$
\subsection{Metric Bracket and Metriplectic Structure}
\par To incorporate we define our symmetric bracket. We make sure to construct it so that it only affects the magnetic flux variable $\psi$, since resistivity only affects the magnetic field lines and not vorticity directly.
$$(F,G)=\int 
\frac{\delta F}{\delta \psi}\eta \nabla^2 \frac{\delta G}{\delta \psi}dxdy$$
\par This has 
\begin{itemize}
    \item Symmetry $((F,G))=((G,F))$
    \item Positive semi-definiteness: $((S,S))\leq0$
    \item Energy Conservation: $((\mathcal{H}, F))=0$ for any $F$
\end{itemize}
\subsection{Combined Dynamics}
\par We know that the metriplectic function of $F$ is always defined as
$$\frac{dF}{dt}= \{F, \mathcal H \}+ ((F, S))$$
\section{Applying Metriplectic 4-bracket algorithm}
\subsection{Introduction}
\par My next big project will be applying metriplectic 4-bracket algorithm to MHD, especially ELMs. I will be writing my notes and such here.

$$
    \{ F, G \} = -\int \left[ \frac{\delta F}{\delta \rho} \nabla \cdot \left( \rho \frac{\delta G}{\delta \mathbf{v}} \right) - \frac{\delta G}{\delta \rho} \nabla \cdot \left( \rho \frac{\delta F}{\delta \mathbf{v}} \right) \right] d^3x
    - \int \left[ \frac{\delta F}{\delta \mathbf{v}} \cdot \left( \left( \frac{\delta G}{\delta \mathbf{v}} \cdot \nabla \right) \mathbf{v} - \left( \frac{\delta F}{\delta \mathbf{v}} \cdot \nabla \right) \mathbf{v} \right) \right] d^3x
    - \int \left[ \frac{\delta F}{\delta s} \left( \frac{\delta G}{\delta \mathbf{v}} \cdot \nabla s \right) - \frac{\delta G}{\delta s} \left( \frac{\delta F}{\delta \mathbf{v}} \cdot \nabla s \right) \right] d^3x
    - \int \left[ \frac{\delta F}{\delta \mathbf{B}} \cdot \left( \nabla \times \left( \frac{\delta G}{\delta \mathbf{v}} \times \mathbf{B} \right) - \nabla \times \left( \frac{\delta F}{\delta \mathbf{v}} \times \mathbf{B} \right) \right) \right] d^3x.
$$

$$(F, G; S, H) = \int_\Omega \left[ \frac{D}{T} \left( \nabla \frac{\delta F}{\delta \rho} \cdot \nabla \frac{\delta G}{\delta \rho} \right) + \frac{\mu}{T} \left( \nabla \frac{\delta F}{\delta \mathbf{m}} : \nabla \frac{\delta G}{\delta \mathbf{m}} \right) + \frac{\kappa}{T^2} \left( \nabla \frac{\delta F}{\delta s} \cdot \nabla \frac{\delta G}{\delta s} \right) + \frac{\eta}{\mu_0^2 T} \left( \nabla \times \frac{\delta F}{\delta \mathbf{B}} \cdot \nabla \times \frac{\delta G}{\delta \mathbf{B}} \right) \right] \rho T d^3\mathbf{x}.$$
\par Where $F^h = F|_{v_h}$, and the same for the other functionals. These brackets are thus antisymmetric brackets and not Poisson brackets, because it fails to satisfy the Jacobi Identity.
\par This property is essential to Poission brackets. However, no grid based discretizations has been found for fluid Poisson bracket(or many for that matter)
\section{Omega-X}
\subsection{Pre-planing}
\subsubsection{Ideas \& Overarching Plan}
\par Start with replicating 1D thermal fluid model, then start with more complex models, then experiment with other ways to do it, then dive deeper into the mathematics and hopefully come up with something(maybe metriplectic integrator), then create a code base, compare these systems to other traditional models, and create tokomak simulation(maybe, not sure if possible yet) that is it so far. Hopefully these ideas will evolve. 

\subsubsection{Metriplectic 4-bracket algorithm for constructing thermodynamically consistent
 dynamical systems}
\paragraph{*Math}
\subsubsection{A thermodynamically consistent discretizations of 1D thermal-fluid
models using their metriplectic 4-bracket structure}
\paragraph{Math}
Edit: so I don't keep forgeting, $\rho$ is mass density, $m$ is momentum density, and $\sigma$ is entropy density. Then for the thermodynamic quantities: $\eta$ is specific entropy, $u$ is velocity, and $T$ is temperature 
 Let $V_h = v_h \subset H^1(Ω)$ be the degree-p continuous
 Galerkin finite element space defined over a uniform grid, $\tau_h $, on Ω: i.e.
 $$V_h = {v_h ∈ H^1(Ω) : vh|K ∈ Pp(K), ∀K ∈ Th}$$
  The discretizations is accomplished
 using the method of lines by positing that all dynamical fields have spatial dependence modeled
 in this Galerkin subspace. However, rather than discretizing the equations of motion themselves,
 we discretize the weak forms implied by the metriplectic formulation.
\\
 Let $(\rho_h. m_h. \sigma_h) \in V_h \times V_h \times V_h$
 So that the discretized Hamiltonian and entropy can be given as:
 $$H^h[\rho_h. m_h. \sigma_h]= \int_\Omega [\frac{1m^2_h}{2 \rho_h}+\rho_h U (\rho_h, \frac{\sigma_h}{\rho_h}]dx$$
 $$S^h[\sigma_h]=\int_\Omega \sigma_h dx$$
 Then the Metriplectic 4 Bracket can be expressed through:
 $$(F^h, K^h, G^h, N^h)_h = -\frac{1}{\text{Re}} \int_{\Omega} \frac{1}{T_h} \left[ \left( K^h \partial_x F^h_{m_h} - F^h \partial_x K^h_{m_h} \right), $
 $$\left( N^h \partial_x G^h_{m_h} - G^h \partial_x N^h_{m_h} \right) \right. \\
\left. +, $
$$ \frac{1}{\text{Pr} \gamma - 1} \frac{1}{T_h} \left( K^h \partial_x F^h - F^h \partial_x K^h \right) \left( N^h \partial_x G^h - G^h \partial_x N^h \right) \right] \, dx $$
\\
Then the poission bracket is thus:
\begin{align*}
    

\{F^h, H^h\}_h(u_h)= − (m_h \partial_x u_h, \phi_m)_{L^2} + (m_h u_h, \partial_x , \phi_m)_{L^2} - \\ (\rho_h \partial_x η_h, \phi_m)_{L^2}  + (\rho_h u_h, \partial_x , \phi_ρ)_{L^2} - (\sigma_h\partial_x T_h, \phi_m)_{L^2} + (\sigma_h u_h, \partial_x, \phi_σ)_{L^2} 
\end{align*}
Now, one thing I will say is that this is not technically a Poisson bracket, because it is fails to satisfy the Jacobi identity. This is because it is impossible(or at least nobody has figured out a way to discretize fluid poission brackets.

To then find equations of motion, adding them together $F^h=\{F^h, H^h\}_h+(F^h, H^h; S^h, H^h)_h$ where $F^h$ is the observable. 

In the case of momentum we must let $\phi_\rho = \phi_\sigma = 0$ so that only $\phi_\rho$ the momentum test function in the finite element(FE) test space $V_h$.
\\
\begin{multiline}
    (\phi_m, \partial_tm_h) + (m_h\partial_xu_h, \phi_m)_{L^2} - (m_h u_h, \partial_x, \phi_m)_{L^2}
    + (\rho_h \partial_x \eta_h, \phi_m)_{L^2} + (\sigma_h\partial_xT_h \phi_m)_{L^2} +
    \frac{1}{Re}(\partial_xu_h, \partial_x, \phi_m)_{L^2} = 0 
\end{multiline}
\\
Next, I will have $\phi_m=\phi_\sigma=0$ to get continuity equations:
\\
\begin{multiline}
    (\phi_\rho, \partial_g \rho_h)_{L^2}-(\rho_hu_h, \partial_x \phi_\rho)_{L^2}
\end{multiline}
\\
Finally the entropy equation is derived in a similar fashion:
\\
\begin{multiline}
    (\phi_\sigma, \partial_g \sigma_h)_{L^2}-(\sigma_h u_h, \partial_x \phi_\sigma)_{L^2}+\frac{1}{Re}[(\frac{(\partial_x u_h)^2}{T_h}, \phi_\sigma)_{L^2}-\frac{1}{Pr}\frac{\gamma}{\gamma-1}[(\frac{\partial_x T_h}{T_h}, \partial_x\phi_\sigma)_{L^2}-(\frac{(\partial_x T_h)^2}{T_h^2}, \phi_\sigma)_{L^2} ]]
\end{multiline}
\\
To prove energy conservation
$$(\frac{u^{n+1}_h-u^n_h}{\Delta t}, \delta h_h)=\frac{H(u^{n+1}_h)-H(U^n_h)}{\Delta t} = 0$$
Positive entropy production can be found through:
$$\frac{s^{n+1}_h-s^n_h}{\Delta t} = \frac{1}{Re}[(\frac{(\partial_x u^n_h)^2}{T^n_h},1)_{L^2}+\frac{1}{Pr}\frac{\gamma}{\gamma-1}\frac{(\partial_x T^n_h)^2}{T^n_h},1)_{L^2}]\geq 0$$

\\
\\
Let's figure out all we need to find for each equation, starting with momentum. 
Start with $M_{ii}=\phi_m$, this should be equal to dx, but I will confirm. Also need the time derivative of the momentum variational, $u_h$ variational, $\rho_h$ variational, $T_h$, $\eta_h$, and $\frac{1}{Re}$, reynolds number, decide later

$$M_{ii}=dx$$
$$u_h=(\rho_h, m_h, \sigma_h)$$
$\rho_h$, $T_h$, and $\eta_h$ are the respective fields, initialized simply and evolved through the equation.
Obviously $\rho_h, m_h,$ and $\sigma_h$ evolve through their metriplectic equations of "motion."
$T_h$, $\eta_h$, are evolved through these equations:
$$(\eta_h + \frac{m^2_h}{2\rho^2_h}-U(\rho_h, \frac{\sigma_h}{\rho_h})-\rho_h \partial_1 U(\\rho_h, \frac{\sigma_h}{\rho_h})+\frac{\rho_h}{\sigma_h}\partial_2U(\rho_h, \frac{\sigma_h}{\rho_h}, \phi_\eta)_{L^2}=0$$
$$(u_h-\frac{m_h}{\rho_h},\phi_u)=0$$
$$(T_h-\partial_2U(\rho_h, \frac{\sigma_h}{\rho_h})\phi_T)_{L^2}=0$$
\\
Nothing new is added in continuity
\\
Entropy adds: $Pr$ and $\gamma$
\paragraph{Coding}
The first thing is to use a Galerkin Method for projecting PDEs into a finite-dimensional function space. (can base off of Firedrake, FEniCS, Galerkin, or another) \[$M_{ii}$\] in mesh.f90

The second thing to do is to initialize the parameters and states(use mesh to calculate field) in states.f90
($Pr$, $Re$, $\gamma$ for constants; initialize ($\rho_h, \sigma_h, m_h, \eta_h, T_h, u_h$)


Then functionals: calculates pointwise functional derivatives of the Total Hamiltonian and Entropy. in functionals.f90
(Update $\rho_h, \sigma_h, m_h, u_h$)

EOS.f90 is used to update the thermodynamic points($\eta_h, T_h$)

Time step the program, by using Gauss-Legendre implicit Runge-Kutta methods in time\_integration.f90

Run the Program(using the driver and makefile)

Input the outputs in io.f90

Some notes: $(f,g)_{L^2}=\int_\Omega f(x) \cdot g(x)$

An example would be, for a point in the momentum equation $(\phi_m, \partial_t m_h)_{L^2}$, or the mass matrix times the derivative of momentum functional as a function of time. Thus it makes it so $\phi_m=M_{ii}=dx$ thus, $(\phi_m, \partial_t m_h)_{L^2}=dx\cdot\frac{m_h}{\partial t}$

\paragraph{Future Work}
The paper presents several ways to improve, or at least look into improving through future work.

Mainly, they mention that many different structure preserving methods exist and may be more suitable.

I also need to figure out how to make this codebase work for more examples


\subsubsection{Structure of Program}
%edit to dirtree later
.1 Omega\_X.
.2 makefile\_folder.
.3 Makefile.
.3 mod.
.3 obj.
.2 src.
.3 Core.
.4 EOS.
.5 Temperature.f90.
.5 Specific\_Entropy.f90
.4 functionals.
.5 Momentum.f90.
.5 Entropy.f90.
.5 Mass.f90.
.4 io.f90.
.4 mesh.
.5 1D\_Galerkin
.4 states.
.5 Constants.f90
.5 feilds.f90.
.4 time\_integrator.
.5 1D_integrator.f90.
.3 driver.
.4 1D\_Thermal\_driver.f90.
.3 visual.
.4 plot.py.


\subsubsection{Coding(1D fluid)}
I am starting out with developing the code for a 1D thermal-fluid. From there I will develop it further.
\paragraph{Makefile(1D fluid)}
Here is my current make file for a 1D fluid, I need to figure out how 

\subsubsection{Plans}
I need to figure out how to make this iterative so complexity can be added.
\subsection{*Mathematics}
\subsection{*Coding, Final}
\subsection{*Test}


\subsection{*Results}

\section{*Hamiltonian Neural Network}
\section{Assumptions of Physics}

\chapter{Random Sci-Fi Research Projects}
"The only way to discovering the limits of the possible is to venture a little way past them into the impossible." |Arthur C. Clarke
\section{Death Star}
\par Not to long ago(from when I wrote this section) I tried to simulate the death star using FLASH-X(which is really a testament to who I am as a person, getting my hands on government codes and the fist thing I do is play around with pop-culture based planetary destruction) anyways. It failed. The laser dynamics were based upon old code that was outdated and I couldn't figure it out. Though I did learn a lot about Flash-X as a system, which is good.
\par So for it... what am I doing, this doesn't make sense if I am trying to teach myself through practical action of writing but I already did the most practical action of all, doing it. This is kinda useless.
\\
\\
\par I later went back and ended up actually making the simulation.

\section{Interstellar travel}
\subsection{Speeds and forces}
\par Before we can get to all of the more complex stuff about design and the effects of extreme environments of specific interstellar space travel we must first look at how forces effect speeds at at relativistic speeds. Here is the math stuff in simplified form: \\
We all know that force is the derivative of the momentum over time:
$$F=\frac{dp}{dt}$$
Where momentum can be expand for relativistic momentum:
$$p=\gamma mv$$
You can then expand force to be(this is a constant mass, we will look at that in a second)
$$F=m \frac{d(\gamma v)}{dt}$$
One thing to keep in mind for future maths is that in simplifying $\dot \gamma$ is equal to $\frac{\gamma v}{c^2}\frac{dv}{dt}$
With the force you can calculate acceleration with
$$a=\frac{F}{m \gamma}\frac{1}{1+\frac{\gamma^2 v^2}{c^2}}$$
Due to the fact most rockets use propulsion which lowers the mass, we must include mass variations:
$$F=\frac{\gamma mv}{dt}$$
Where you must use the product rule to expand
$$F=\dot \gamma mv+\gamma \dot m v+ \gamma m \dot v$$
Which can be further expanded to
$$F=mv \frac{\gamma v}{c^2}\frac{dv}{dt}+\gamma v \frac{dm}{dt}+\gamma m \frac{dv}{dt}$$
With $\dot m$ being the infinitesimal change in mass over time.\\ 
To calculate acceleration you must find $\frac{dv}{dt}$. First subtract the middle equation without acceleration within it.
$$F-\gamma v \frac{dm}{dt}=mv \frac{\gamma v}{c^2}\frac{dv}{dt}+\gamma m \frac{dv}{dt}$$
Then factor for acceleration
$$F- \gamma v \frac{dm}{dt}= \frac{dv}{dt}(mv \frac{\gamma v}{c^2}+\gamma m)$$
Then divide the parentheses:
$$(F- \gamma v \frac{dm}{dt}) \frac{1}{mv \frac{\gamma v}{c^2}+\gamma m}=a$$
Now, this is all well and good, now let us find the position. To take position from acceleration you must take a double integral, but this integral can only be solved numerically(maybe it could be approximated) \\
 \\
Another possible way to solve this is through Hamiltonians/Lagrangians (I will choose Hamiltonians) \\
Let us first start with the Hamiltonian expressed for relativistic speeds in covariant form \\
First the Lagrangian must be calculated:
The four-momentum is
$$P^\mu=m \gamma (c,v)$$
Energy:
$$E= \gamma mc^2$$
Therefore the Lagrandian is:
$$L=-mc^2\sqrt{1-\frac{v^2}{c^2}}$$
Now to find generalized momentum:
$$p=\frac{\partial L}{\partial \boldsymbol{v}}$$
$$p=\gamma m v$$
Now the Hamiltonian
$$H=p \cdot v - L$$
Therefore a relativistic Hamiltonain can be represented as
$$H=\sqrt{(pc)^2+(mc^2)^2}$$
Now to include acceleration
$$H(r,p)=\sqrt{(pc)^2+(mc^2)^2}+V(r)$$
Now to make mass variable for the rockets:
$$H(r,p,t)=\sqrt{(p(t)c)^2+(m(t)c^2)^2}+H_{thrust}$$
Now to incorporate with equations of motion:
$$\frac{dr}{dt}=\frac{pc^2}{\sqrt{(p(t)c)^2+(m(t)c^2)^2}}$$
Now, this works for the time for most people, now let us look at proper time
$$H(r,p,\tau)=\sqrt{(p(\tau)c)^2+(m(\tau)c^2)^2}+H_{thrust}$$
For specific cases, these equations will be evolved to include them, but this is the general theoretical framework.
\\ 
\\
\par One thing I will add, is the force will likely be variable, a couple of days of 2-3 g, most of the time in g then reversing after the half-way point.
\subsection{Magnetic Fusion Plasma Drive}
\subsubsection{Theoretical Basis}
\par Magnetic Fusion Plasma Drive is a type of propulsion system specifically for interstellar travel.
\par The basis of this concept is that there is a fusion reactor undergoing Thermal Nuclear fusion:
$$D+T \xrightarrow{}^4He+n+17.6MeV$$
\par The spent fuel then being shot out the nozzle as propulsion.
\subsection{*Optimization}
\subsection{*Special Circumstances}
\subsubsection{*Strong Magnetic Fields}
\subsubsection{*Gravitational Fields}


\section{Kerr Black hole stuff}
\subsection{Energy (finish)}
\par Kerr Black holes, or spinning black holes are a great source of energy. They have extreme amounts of radial kinetic energy that we are actually capable of 'bleeding' off through several methods.
\par First, we must ask why we are able to do this. Well, because a rotating object with a strong gravitational field 'drags' spacetime, that spacetime can affect the direction, speed, and even waves in a way that can amplify them.
\par First, Penrose(really like that dude) found that if you break apart an object rotating a Kerr black hole, the object that escapes gains energy. This process is challenging given that the object that escapes must have a velocity of $v>\frac{c}{2}$. There is a whole lot of math involved in proving this, but I will skip ahead.
\par The next that is found is that electromagnetic waves can be amplified. That if they are reflected around the black hole, they will gain energy and be amplified(this process is different from the Doppler shift and does not change the frequency.
\par After a lot, and I mean a lot of very complected mathematics you can find the amplification factor as
$$Z = 8r^2_+T_{H(^{2l+1)}) }(r_+-r_-)^{2l}[\frac{\Gamma(1+l-s)\Gamma(1+l+s)}{2l+1)!!\Gamma(l+1)\Gamma(2l+1)}]^2 \times, $$
$$\sinh(\frac{m \Omega_H}{r_+ T_H})\Gamma(l-\frac{im\Omega_H}{\pi r_+ T_H}+1)\Gamma(l+\frac{im\Omega_H}{\pi r_+ T_H}+1)$$

\par For more information visit 
\href{https://arxiv.org/pdf/1501.06570}{here}

\par Now the point of this is to take strange theoretical and apply it, so I will apply it for actual application. For this I will assume the average mass of 10 $M_\odot $ or ten solar masses. A spin of 0.98(if 1 refers to an extremal Kerr black hole)

\par First we must find out what a spin 0.98 really means for a $10 M_\odot$ black hole. We can find this with a simple equation with a being the spin parameter, r being the radius, and M being the mass
$$\Omega=\frac{ac^3}{2GMr_h}$$
Though, for a kerr black hole the is slightly different to the swarzchild radius.
$$r_H=\frac{GM}{c^2}(1+\sqrt{1-\frac{ac}{GM}}$$

\\


\par You know, I wanted to optimize the black hole engine, but I don't want to do that anymore, so I choice to move on.

\subsection{*Multiversal Travel}
\subsection{*Other effects}
\section{*Extreme Theoretical Work}
\section{Astroid Game}
\subsection{Introduction}
\par I have recently created a Python game based on the Atari game Asteroid.
\subsection{Classical}
\par Just like the regular game, nothing special.
\subsection{Newtonian Gravity}
\par This adds that the asteroids and UFO's mass, so they attract each other and add more challenges.
\subsection{Dark Matter}
\par Add the challenge of invisible mass attractors, though when you are close by they light up.
\subsection{Relativistic}
\par This adds time dilatation(time skips or slow downs), length contractions(changing shapes of objects), black holes, and even doopler effect. 
%%%%%%%%%%%%%%%%%%%%%%%
\chapter{Current Physics Research}
\section{Introduction}
\par Here I will be writing about things that I am learning in physics, mathematics, coding, etc. This will be use to help me refine my own ideas and as a fun exercise. Also, as notes
\section{*Fusion Technology}
\subsection{*Introduction}
\section{Theoretical Physics by Georg Joos}
\par The vector analysis portion is intresting due to the fact that it takes something as simple as vector analysis and brings rigorous ideas to it like that $\oint ds=0 $ for close surfaces. 
\par I also never really thought about using vector analysis instead of tensors. 
\par The rest of curl, gauss, and such is very simple. Though, now it is getting into tensors though vectors.
$$dv_x=ds\nabla v_x$$
$$dv_y=ds\nabla v_y$$
$$dv_z=ds\nabla v_z$$
\par So therefore $dv=ds \nabla v$
\par To calculate v, three vectors or nine scaler must be known.
\par Another interesting addition is the fact that in physics, symmetric tensors can be represented as a surface to the second degree.
\par One thing I have noticed is I need a more intuitive grasp into the relationships between curl, div, laplance, and grad. They use it a lot to simplify the calculations. 
$$\nabla^2 f = \nabla \cdot (\nabla f)$$
The Laplace operator is equal to the divergence of the gradient.
$$\nabla \times (\nabla f) = 0$$
For smooth scaler fields. The curl of the gradient is zero.
$$\nabla \cdot (\nabla \times \mathbf{F}) = 0$$
The same is true for the divergence of the curl.
$\nabla^2 \mathbf{F} = \nabla(\nabla \cdot \mathbf{F}) - \nabla \times (\nabla \times \mathbf{F})$
For vector fields, the Laplace operator has different relations.
\\
\par Next, onto calculus of variations. I will derive Euler-Lagrange differential equation.
\par Let: $\tilde{y}$ be a neighboring function to $y$. Where $\in$ be a small quantity and $\eta(x)$ be a arbitrary function of x. so if $\tilde{y}=y+\epsilon \eta$ then $\tilde{y'}=y'+ \epsilon \eta'$. Here we stipulate that the two functions $\tilde{y}$ and $y$ converge at the beginning and end. Thus, $\eta$ must vanish at the ends. So if we substitute an integral $I$,, we find that it becomes a function of $\epsilon$. Then we require that $I(\epsilon)$ must have an extreme value of $\epsilon=0$. Here it is in mathematical terms:
$$I(\epsilon)=\int^{x_1}_{x_0}F(x,y+\epsilon \eta, y'+\epsilon\eta')dx=extremum \ for \ \epsilon=0$$
\par This gives us a simple way of determining the extreme value for a given integral. The condition is:
$$(\frac{dI}{D\epsilon})_{\epsilon=0}=0$$
\par We can then expand the integrad function $F$ in Taylor's series, according to the powers of $\epsilon$.
\par The differentiate with respect to $\epsilon$.
\par This expression then vanishes for $\epsilon=0$. Thus then simply remains the condition for the extremum.
\par Integrate this to get the Euler-Lagrange differential equation:
$$\frac{d}{dx}\frac{\partial F(x, y, y')}{\partial y'}-\frac{\partial F(x, y, y')}{\partial y}=0$$
\\
\par For writing constraining forces, we find it to be 
$$Z=\lambda \ grad \ G$$
\par Where $G(x,y,z)$ is the equation of the surface
\\

\section{Foundation of Mechanics by Ralph Abraham and Jerrold E. Marsden}
\section{Plasma Physics by Richard Fitzpatrick}
\section{Geometrical Methods in the Theory of Ordinary Differential Equations}
\section{Mathematical Methods for Physicists by Weber and Arfken}

\section{On Sympathetic Reduction in Classical Mechanics}
\section{Magnetic Fields and Magnetic Diagnostics for Tokamak Plasmas by Alan Wooton}
\section{Advanced MHD with Applications to Laboratory and Astrophysical Plasmas by Cambridge }
\par For wide variety of MHD instabilities operating in tokamaks, represented by normal modes of the form(assuming that  cylindrical approximation and the toroidal representation may be ignored):
$$f(\psi,\vartheta, \varphi, t)=\sum\nolimits_m \tilde{f}(\psi) e^{i(m\vartheta + \eta \varphi - \omega t)} $$
\par Is only unstable for perpendicular wave vectors.
\par The reason is the enormous field line bending energy of the Alfven waves

\section{Hamiltonian description of the ideal Fuid by
 P.J.Morrison}
 \section{Classical Dynamics a Modern Perspective}
 \par 
\section{Papers}
\subsection{Hamiltonian formulations for perturbed dissipationless plasma equations}
\par I can hardly understand this paper, but it was an interesting read.
\section{Practice}


%%%%%%%%%%%%%%%%%%%%%%%


